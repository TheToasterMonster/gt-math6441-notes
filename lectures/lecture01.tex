\chapter{Jan.~6 --- CW-Complexes}

\section{Introduction and Motivation}
Algebraic topology builds ``functions'' (actually
\emph{functors})
\[
  \{
    \text{topological spaces, continuous maps}
  \}
  \longrightarrow
  \{
    \text{algebraic things, algebraic maps}
  \},
\]
where ``algebraic things'' can be groups, vector
spaces, etc. The main objective of algebraic topology is
to \emph{distinguish topological spaces}, e.g.
showing that $\R^n \ncong \R^m$ for $n \ne m$.
More applications are:
\begin{enumerate}
  \item Studying maps between spaces.\footnote{All maps and functions in this class are continuous unless otherwise specified.}
    \begin{itemize}
      \item Does a given space $M$ embed in $N$? For
        instance, for what $m$ does $\R P^n$ embed
        in $\R^m$? (This is still not known in general.)
        Here $\R P^n$ is the real projective space.
      \item Lifting maps, i.e. given
        $f : A \to B$ and $g : E \to B$, does
        there exist a map $\widetilde{f} : A \to E$
        such that $g \circ \widetilde{f} = f$?
        In other words, is there a map $\widetilde{f}$
        such that the following diagram commutes:
        \begin{center}
        \begin{tikzcd}
          & E \arrow[d, "g"] \\
          A \arrow[ur, "\widetilde{f}", dashed] \arrow[r, "f", swap] & B
        \end{tikzcd}
        \end{center}
      \item Fixed point problems: Given $f : X \to X$,
        does $f$ have a fixed point, i.e. $x \in X$
        such that $f(x) = x$? Such theorems are used
        to prove the existence of solutions to
        ordinary differential equations, for instance.
    \end{itemize}
  \item Group actions, e.g. which finite groups act
    freely on $S^n$?
  \item Group theory, e.g. showing that every subgroup of a free
    group is free.
    Another example is that if $F_n$ is
    the free group on $n$ generators, then
    its \emph{commutator} $[F_n, F_n]$ is not
    finitely generated.
  \item Algebra, e.g. proving the fundamental theorem
    of algebra.
\end{enumerate}

This course will cover the following topics:
\begin{enumerate}
  \item The \emph{fundamental group}
    $\pi_1(X, x_0)$ of a space $X$ for $x_0 \in X$,
    and \emph{covering spaces}.
  \item The \emph{homology groups} $H_k(X)$ for
    $k = 0, 1, 2, \dots$. These groups are abelian.
  \item The \emph{cohomology ring}
    $H^*(X) = \bigoplus_{k = 0}^\infty H^k(X)$.
\end{enumerate}
But before getting to this, we need to develop some
important ideas.

\section{CW-Complexes}

\begin{definition}
  Let $D^n \subseteq \R^n$ be the unit disk and
  $S^{n - 1} = \partial D^n$. Given a topological
  space $Y$ and a continuous map $a : S^{n - 1} \to Y$,
  the space obtained from $Y$ by \emph{attaching} an
  \emph{$n$-cell} (\emph{via $a$}) is
  \[
    Y \cup_a D^n = (Y \sqcup D^n) / {\sim},
  \]
  where the equivalence relation $\sim$ is given by
  $x \sim a(x)$ for $x \in \partial D^n$.
  Here $\sqcup$ denotes disjoint union.
\end{definition}

\begin{definition}
  An \emph{$n$-complex} or an \emph{$n$-dimensional CW-complex}
  is defined inductively by:
  \begin{itemize}
    \item A $(-1)$-complex is the empty set $\varnothing$.
    \item An $n$-complex $X^n$ is a space obtained
      from an $(n - 1)$-complex by attaching $n$-cells.
  \end{itemize}
  An $n$-complex is \emph{finite} if it involves
  only a finite number of cells. The \emph{$k$-skeleton}
  of $X$ is the union of all $n$-cells in $X$
  with $n \le k$.
\end{definition}

\begin{remark}
  Any CW-complex is Hausdorff. See Hatcher for a proof.
\end{remark}

\begin{example}
  Here are some examples of CW-complexes:
  \begin{itemize}
    \item A $0$-complex is a union of points.
      This is because $D^0 = \{\text{pt}\}$ and
      $\partial D^0 = \varnothing$.
    \item A $1$-complex is a graph (points and lines
      connecting them).
    \item The torus $T$ (a square with opposite sides
      identified) is a $2$-complex. Here
      the $0$-skeleton $T^{(0)}$ is the common corner
      on the square and the $1$-skeleton $T^{(1)}$
      is two sides of the square after taking the
      quotient. The $2$-skeleton $T = T^{(2)}$ is the
      entire torus.
    \item Another example of a $2$-complex is the
      two-holed torus, which is obtained by identifying
      the edges of an octagon (pairs of every other
      edge identified with opposite orientation).\footnote{This CW-decomposition of the two-holed torus results in one $0$-cell,
      four $1$-cells, and one $2$-cell.}
    \item A third example of a $2$-complex is
      $X^{(1)} \cup_a D^2$ given an attaching map
      $D^2 \to X^{(1)}$.
    \item Consider the unit sphere $S^n \subseteq \R^{n + 1}$.
      One way to give $S^2$ a CW-complex structure is
      to see the sphere as two disks $D^2$ glued together,
      resulting in one $0$-cell, one $1$-cell, and two
      $2$-cells. Another way is to start with
      two points, attach two $1$-cells to get a circle,
      and then attaching two disks to get $S^2$. This
      results in two $0$-cells, two $1$-cells, and two
      $2$-cells.

      The second idea generalizes to $S^n$. We can
      write
      \[
        S^n = S^{n - 1} \cup_{a_1} D^n \cup_{a_2} D^n,
      \]
      where $S^{n - 1}$ inductively has a CW-complex
      structure. This yields two $k$-cells for each
      $k \le n$.

      Another way to put a CW-complex structure
      on $S^n$ is to attach $D^n$ to a point with
      $\partial D^n \to \{\text{pt}\}$.
      In particular, notice that a space
      can in general have several different CW-complex
      structures.
    \item Consider the \emph{$n$-dimensional real projective space}
      \[
        \R P^n = \{\text{lines through the origin in } \R^{n + 1}\}.
      \]
      Since each line through the origin passes through
      $S^n$ twice, we can equivalently think of $\R P^n$
      as the unit sphere $S^n$ with antipodal points
      identified.

      We can also think of this as $D^n$ with
      antipodal points on $\partial D^n$ identified.
      Since $\partial D^n = S^{n - 1}$, this is
      simply $\R P^{n - 1} \cup_{a} D^n$, where
      $a : \partial D^n \to \R P^{n - 1}$ is the
      quotient map. This gives $\R P^n$ a CW-complex
      structure with one $k$-cell for each $k \le n$.
    \item The complex projective space
      $\C P^n$ has a similar CW-complex structure with
      one $k$-cell for each even $k \le 2n$.
      One can verify this as an exercise.
    \item Any smooth manifold has a CW-complex
      structure. See Hatcher.
  \end{itemize}
\end{example}

\begin{exercise}
  Show the product of CW-complexes is a CW-complex.
\end{exercise}

\begin{definition}
  A \emph{subcomplex} of a CW-complex $X$ is a closed
  subset $A \subseteq X$ that is a union of cells in $X$.
  In particular, $A$ is also a CW-complex and
  $(X, A)$ is called a \emph{CW-pair}.
\end{definition}

\section{Homotopy}

\begin{definition}
  Let $X$ and $Y$ be topological spaces. Two maps
  $f, g : X \to Y$ are \emph{homotopic}, denoted
  $f \sim g$, if there exists a continuous map
  $\Phi : X \times [0, 1] \to Y$ such that
  \[
    \Phi(x, 0) = f(x) \quad \text{and} \quad \Phi(x, 1) = g(x)
  \]
  for all $x \in X$.
  In this case, $\Phi$ is called a \emph{homotopy}
  between $f$ and $g$.
\end{definition}

\begin{remark}
  We note the following:
  \begin{itemize}
    \item A homotopy $\Phi$ gives a family of maps
      $\phi_{t_0} : X \to Y$ given by
      $x \mapsto \phi(x, t_0)$ which is continuous
      in $t_0$. So maps are homotopic if there is a
      continuous family of maps between them.
    \item If $A \subseteq X$ then we say that
      $f$ is \emph{homotopic to $g$ rel $A$},
      denoted $f \sim_A g$, if there exists
      $\Phi$ as above with the additional property
      that $\Phi(x, t) = f(x)$ for all $x \in A$,
      i.e. points in $A$ are fixed.
    \item If $A \subseteq X$ and $B \subseteq Y$,
      then the notation $f : (X, A) \to (Y, B)$
      means that $f : X \to Y$ and
      $f(a) \in B$ for each $a \in A$.
      We say that $f$ is a \emph{map of pairs}.
      If $f, g : (X, A) \to (Y, B)$, then $f, g$
      are \emph{homotopic as maps of pairs} if each
      $\phi_t : (X, A) \to (Y, B)$ is a map of pairs.
  \end{itemize}
\end{remark}
