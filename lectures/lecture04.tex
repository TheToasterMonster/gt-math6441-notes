\chapter{Jan.~15 --- Fundamental Group}

\section{Fundamental Group}

\begin{remark}
  The basic idea of the \emph{fundamental group} is to
  study the topology of a space with loops mapped into
  the space. For instance, intuitively, any loop
  in $S^2$ can be ``pulled back'' to (i.e. is homotopic
  to) a constant loop. On the other hand, a loop
  wrapping around the hole in $T^2$ gets stuck and
  cannot be ``pulled back'' to a constant loop. The
  same issue happens for a loop in $T^2$ around the
  cylindrical part.
  The fundamental group is a way to make this intuition
  precise and measure the ``holes'' in a space.
\end{remark}

\begin{definition}
  The \emph{fundamental group} of a based space
  $(X, x_0)$ is
  \[
    \pi_1(X, x_0) = [(S^1, n), (X, x_0)]_0,
  \]
  i.e. the homotopy classes of loops in $X$ based
  at $x_0$. Here $n = (0, 1)$ is the north pole of
  $S^1$.
\end{definition}

\begin{exercise}
  Let $S^1 \subseteq \R^2$ be the unit circle and
  let $p : [0, 1] \to S^1$ be given by
  \[
    t \mapsto (\cos 2\pi t, \sin 2\pi t).
  \]
  Show that $p$ is a quotient map, so we can think of
  $S^1$ as $[0, 1]$ with $0, 1$ identified. Moreover,
  show that there is a one-to-one correspondence
  between maps of the form\footnote{Such a loop $\gamma$ is called a \emph{based loop}.}
  \[
    \gamma : ([0, 1], \{0, 1\}) \to (X, x_0)
    \quad \text{and} \quad
    \widetilde{\gamma} : (S^1, \{(1, 0)\}) \to (X, x_0)
  \]
  given by $\widetilde{\gamma} \mapsto \widetilde{\gamma} \circ p = \gamma$,
  and that homotopies of $\widetilde{\gamma}$
  rel $\{0, 1\}$ correspond to homotopies of $S^1$
  rel $\{(1, 0)\}$.
\end{exercise}

\begin{remark}
  Using the above exercise, we can think of
  $\pi_1(X, x_0) = [S^1, X]_0$ instead as
  \[
    \pi_1(X, x_0) = [([0, 1], \{0, 1\}), (X, x_0)]_0.
  \]
  Given a based loop $\gamma : [0, 1] \to X$,
  we denote its equivalence class in $\pi_1(X, x_0)$
  by $[\gamma]$.
\end{remark}

\begin{definition}
  If $\gamma_1, \gamma_2$ are loops in $X$ based
  at $x_0 \in X$, their \emph{concatenation}
  $\gamma_1 * \gamma_2 : [0, 1] \to X$ is
  \[
    t \mapsto
    \begin{cases}
      \gamma_1(2t) & \text{if } 0 \le t \le 1 / 2, \\
      \gamma_2(2t - 1) & \text{if } 1 / 2 \le t \le 1.
    \end{cases}
  \]
\end{definition}

\begin{remark}
  Concatenation of loops indeed yields another loop
  since $\gamma_1 * \gamma_2(0) = \gamma_1 * \gamma_2(1) = x_0$
  and $\gamma_1 * \gamma_2$ is continuous since
  the definitions agree on the closed
  set $\{1 / 2\}$.
\end{remark}

\begin{remark}
  We can clearly see that $\gamma_1 * \gamma_2$ is
  well-defined given $\gamma_1$ and $\gamma_2$,
  but can we define $[\gamma_1] * [\gamma_2]$
  for homotopy classes of loops in a well-defined
  manner? We need to check
  that if $\gamma_1 \sim \gamma_2$ and $\delta_1 \sim \delta_2$
  (i.e. $\gamma_1, \gamma_2 \in [\gamma_1]$ and
  $\delta_1, \delta_2 \in [\delta_1]$), then we
  also have
  $\gamma_1 * \delta_1 \sim \gamma_2 * \delta_2$.

  To do this, let $H : [0, 1] \times [0, 1] \to X$
  be the homotopy from $\gamma_1$ to $\gamma_2$ and
  $G : [0, 1] \times [0, 1] \to X$ be the homotopy
  from $\delta_1$ to $\delta_2$. We need to construct
  a homotopy $\widetilde{H} : [0, 1] \times [0, 1] \to X$
  from $\gamma_1 * \delta_1$ to $\gamma_2 * \delta_2$.

  Note that if we think of $[0, 1] \times [0, 1]$
  as the unit square, then $\widetilde{H}$ is
  already defined on the boundary: the left and right
  sides are constantly $x_0$, the top side is
  $\gamma_2 * \delta_2$, and the bottom side is
  $\gamma_1 * \delta_1$. So we only need
  to define it on the interior.
  For this, note that the vertical line in
  the middle of the square is also constantly
  $x_0$ by construction: This creates two rectangles
  on each half, which we can fill with $H$ and $G$.

  More formally, we can construct the homotopy
  $\widetilde{H} : [0, 1] \times [0, 1] \to X$
  explicitly via
  \[
    (t, s) \mapsto
    \begin{cases}
      H(2t, s) & \text{if } 0 \le t \le 1 / 2, \\
      G(2t - 1, s) & \text{if } 1 / 2 \le t \le 1.
    \end{cases}
  \]
  This is continuous since the definitions agree
  on the closed set $\{t = 1 / 2\}$. Thus, setting
  \[
    [\gamma_1] * [\delta_1] = [\gamma_1 * \delta_1]
  \]
  gives a well-defined binary operation by the above
  discussion.
\end{remark}

\begin{lemma}
  The pair $(\pi_1(X, x_0), *)$ is
  a group.
\end{lemma}

\begin{proof}
  For the identity, let $e : [0, 1] \to X$ be the
  constant loop $t \mapsto x_0$. We will show that
  \[
    [e] * [\gamma] = [\gamma] = [\gamma] * [e].
  \]
  The picture is that $[0, 1] \times [0, 1]$ has
  $\gamma$ on the top side and $\gamma * e$ on the
  bottom. By drawing a line from the midpoint of
  the bottom side and the top-right corner, we see
  that we can fill the right portion with just
  $x_0$ and the left portion with $\gamma$.
  The equation of this line is $s = 2t - 1$, so
  $t = (s + 1) / 2$.
  Thus from the picture, we can write the explicit
  homotopy $H : [0, 1] \times [0, 1] \to X$ via
  \[
    H(t, s) =
    \begin{cases}
      \gamma(2 / (s + 1), t) & \text{if } 0 \le t \le (s + 1) / 2, \\
      x_0 & \text{if } (s + 1) / 2 \le t \le 1.
    \end{cases}
  \]
  One can use a similar construction to show that
  $[e] * [\gamma] = [\gamma]$, so that
  $[e]$ is an identity element.

  Now we show the existence of inverses. Given
  a loop $\gamma$, define
  $\overline{\gamma}$ via $\overline{\gamma}(t) = \gamma(1 - t)$, i.e.
  $\gamma$ backwards. Set
  $\gamma_s(t) = \gamma(st)$. Note that
  as $t$ goes from $0$ to $1$, $\gamma_s$ goes
  from $\gamma(0)$ to $\gamma(s)$, and also that
  $\overline{\gamma}_s(t) = \gamma(s - st)$. So
  we can write the homotopy
  $H : [0, 1] \times [0, 1] \to X$ between
  $\gamma * \overline{\gamma}$ and $e$ by
  \[
    H(t, s) =
    \begin{cases}
      \gamma_s(2t) & \text{if } 0 \le t \le 1 / 2, \\
      \overline{\gamma}_s(2t - 1) & \text{if } 1 / 2 \le t \le 1
    \end{cases}
    =
    \begin{cases}
      \gamma(2st) & \text{if } 0 \le t \le 1 / 2, \\
      \gamma(s - s(2t - 1)) & \text{if } 1 / 2 \le t \le 1.
    \end{cases}
  \]
  Thus setting $[\gamma]^{-1} = [\overline{\gamma}]$
  gives us inverses.

  Finally, for associativity, we need to see that
  $(\gamma_1 * \gamma_2) * \gamma_3 \sim \gamma_1 * (\gamma_2 * \gamma_3)$. Again by drawing a picture, we
  see that we can draw two diagonal lines connecting the
  starting points of $\gamma_2$ on top and bottom and
  the ending points of $\gamma_1$ on top and bottom.
  Write an explicit formula for the homotopy
  as an exercise.
\end{proof}

\section{Induced Homomorphisms}

\begin{remark}
  If $f : X \to Y$ and $x_0 \in X$, let
  $y_0 = f(x_0)$. Then given a based loop
  $\gamma : [0, 1] \to X$, note that the composition
  $f \circ \gamma : [0, 1] \to Y$ is a based loop
  in $Y$. Also, if $\gamma \sim \delta$, then
  $f \circ \gamma \sim f \circ \delta$ (if
  $H$ is a homotopy from $\gamma \sim \delta$, then
  $f \circ H$ is a homotopy from $f \circ \gamma$
  to $f \circ \delta$). In particular,
  $f$ induces a map
  \[f_* : \pi_1(X, x_0) \to \pi_1(Y, y_0).\]
\end{remark}

\begin{lemma}
  The induced map $f_* : \pi_1(X, x_0) \to \pi_1(Y, y_0)$ is a homomorphism.
\end{lemma}

\begin{proof}
  Note that
  \[
    \gamma_1 * \gamma_2(t) =
    \begin{cases}
      \gamma_1(2t) & \text{if } 0 \le t \le 1 / 2, \\
      \gamma_2(2t - 1) & \text{if } 1 / 2 \le t \le 1
    \end{cases}
  \]
  and that
  \[
    (f \circ \gamma_1) * (f \circ \gamma_2)(t) =
    \begin{cases}
      f(\gamma_1(2t)) & \text{if } 0 \le t \le 1 / 2, \\
      f(\gamma_2(2t - 1)) & \text{if } 1 / 2 \le t \le 1,
    \end{cases}
  \]
  so $f \circ (\gamma_1 * \gamma_2) = (f \circ \gamma_1) * (f \circ \gamma_2)$, which implies
  $f_*([\gamma_1 * \gamma_2]) = f_*([\gamma_1]) * f_*([\gamma_2])$.
\end{proof}

\begin{exercise}
  Check the following as an exercise:
  \begin{enumerate}
    \item $(f \circ g)_* = f_* \circ g_*$;
    \item if $f : X \to Y$ is homotopic rel
      base point to $g : X \to Y$, then
      $f_* = g_* : \pi_1(X, x_0) \to \pi_1(Y, y_0)$.
  \end{enumerate}
\end{exercise}

\begin{remark}
  How does $\pi_1$ depend on the base point? Let
  $x_0, x_1 \in X$, and suppose that there
  exists a path $h : [0, 1] \to X$ with
  $h(0) = x_0$ and $h(1) = x_1$. Then if $\gamma$ is a
  loop based at $x_1$, we can get a loop based at $x_0$
  by going from $x_0$ to $x_1$ along $h$, taking
  $\gamma$, and then going back to $x_0$. More
  explicitly, this is
  \[
    h * \gamma * \overline{h}(t) =
    \begin{cases}
      h(3t) & \text{if } 0 \le t \le 1 / 3, \\
      \gamma(3t - 1) & \text{if } 1 / 3 \le t \le 2 / 3, \\
      \overline{h}(3t - 2) & \text{if } 2 / 3 \le t \le 1.
    \end{cases}
  \]
\end{remark}

\begin{lemma}
  The path $h$ induces an isomorphism
  $\phi_h : \pi_1(X, x_1) \to \pi_1(X, x_0)$
  by $[\gamma] \mapsto [h * \gamma * \overline{h}]$.
\end{lemma}

\begin{proof}
  Check as an exercise that $\phi_h$ is a well-defined
  homomorphism.
  To show that $\phi_h$ is an isomorphism, we claim
  that $\phi_{\overline{h}}$ is an inverse of $\phi_h$.
  To see this, let $[\gamma] \in \pi_1(X, x_0)$. Then
  \[
    \phi_h \circ \phi_{\overline{h}}([\gamma])
    = [h * \overline{h} * \gamma * h * \overline{h}]
    = [h * \overline{h}] * [\gamma] * [h * \overline{h}]
    = [e] * [\gamma] * [e]
    = [\gamma],
  \]
  where the second equality follows by the same
  proof for associativity of $*$. This proves the result.
\end{proof}

\begin{remark}
  Note the following based on the above lemma:
  \begin{enumerate}
    \item The isomorphism class of $\pi_1(X, x_0)$
      only depends on the path component of $X$
      containing $x_0$.
    \item The isomorphism \emph{depends on} $h$.
      One needs to be careful about using
      the correct identification.
  \end{enumerate}
\end{remark}
