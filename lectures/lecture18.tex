\chapter{Mar.~12 --- Mayer-Vietoris Theorem}

\section{Mayer-Vietoris Theorem}

\begin{theorem}[Mayer-Vietoris]\label{thm:mayer-vietoris}
  Let $A, B \subseteq X$ be subspaces such that
  $X = (\Int A) \cup (\Int B)$, and let
  \begin{center}
    \begin{tikzcd}
      A \cap B \ar[r, "i_A"] \ar[dr, swap, "i_B"] & A \ar[dr, "j_A"] \\
      & B \ar[r, swap, "j_B"] & X
    \end{tikzcd}
  \end{center}
  be the inclusions. Then the sequence
  \begin{center}
    \begin{tikzcd}
      \cdots \ar[r] & H_n(A \cap B) \ar[r, "\phi"] & H_n(A) \oplus H_n(B) \ar[r, "\psi"] & H_n(X) \ar[r, "\partial"] & H_{n - 1}(A \cap B) \ar[r] & \cdots
    \end{tikzcd}
  \end{center}
  is exact, where $\phi = (i_A)_* \oplus (i_B)_*$,
  $\psi([a], [b]) = (j_A)_*([a]) - (j_B)_*([b])$, and
  \[
    \partial [z] = [\partial a]
    \quad \text{where } z = a + b \text{ for } a \in C_*(A), b \in C_*(B).
  \]
\end{theorem}

\begin{remark}
  This is like an analogue of the Seifert-van Kampen
  theorem for homology.
\end{remark}

\section{Applications of the Mayer-Vietoris Theorem}
\begin{example}
  Consider $T^2 = S^1 \times S^1$. Let $c$ (e.g.
  $\{\pi / 4\} \times S^1$) be a
  circle around the center of the torus. Let
  $A$ be a small neighborhood of $c$
  (e.g. $(0, \pi / 2) \times S^1$) and
  $B = T^2 - c \cong S^1 \times I$. Note that
  $A \cap B$ is the union of two annuli.
  We have $A \simeq S^1$, $B \simeq S^1$, and
  $A \cap B \simeq S^1 \cup S^1$. Hence
  \[
    H_k(A) \cong H_k(B) \cong
    \begin{cases}
      \Z & \text{if } k = 0, 1, \\
      0 & \text{if } k \ne 0, 1,
    \end{cases}
    \quad \text{and} \quad
    H_k(A \cap B) \cong
    \begin{cases}
      \Z \oplus \Z & \text{if } k = 0, 1, \\
      0 & \text{if } k \ne 0, 1.
    \end{cases}
  \]
  By Mayer-Vietoris, we have the exact sequence:
  \begin{center}
    \begin{tikzcd}[row sep=small]
      0 \ar[d, equal] & & \Z \oplus \Z \ar[d, equal] & \Z \oplus \Z \ar[d, equal] \\
      H_2(A) \oplus H_2(B) \ar[r] & H_2(X) \ar[r, "\partial_2"] & H_1(A \cap B) \ar[r, "\phi_1"] & H_1(A) \oplus H_1(B) & \\ \ar[r, "\psi_1"] & H_1(X) \ar[r, "\partial_1"] & H_0(A \cap B) \ar[r, "\phi_0"] & H_0(A) \oplus H_0(B) \ar[r, "\psi_0"] & H_0(X) \ar[r] & 0 \\
      & & \Z \oplus \Z \ar[u, equal] & \Z \oplus \Z \ar[u, equal] & \Z \ar[u, equal]
    \end{tikzcd}
  \end{center}
  Note that $\phi_0$ is induced by the inclusion,
  so $\phi_0(1, 0) = \phi_0(0, 1) = (1, 1)$, and
  similarly one can also see that
  $\phi_1(1, 0) = \phi_1(0, 1) = (1, 1)$.
  Since $H_2(A) \cong H_2(B) \cong 0$, we have
  $\ker \partial_2 = 0$ and so
  \[
    \langle (1, -1) \rangle \cong \Z \cong \ker \phi_1
    = \im \partial_2 = \frac{H_2(X)}{\ker \partial_2} = H_2(X).
  \]
  Next, we have $\ker \psi_1 = \im \phi_1 \cong \Z$,
  which is generated by $(1, 1)$. Then
  \[
    \im \psi_1
    \cong \frac{\Z \oplus \Z}{\ker \psi_1}
    \cong \frac{\Z \oplus \Z}{\Z}
    \cong \frac{\Z \oplus \Z}{\langle (1, 1) \rangle}
    \cong \Z.
  \]
  We also have $\im \partial_1 = \ker \phi_) \cong \Z$
  (generated by $(1, -1)$, as before), so
  \[
    \Z \cong \im \partial_1 \cong \frac{H_1(X)}{\ker \partial_1}
    \cong \frac{H_1(X)}{\im \psi_1} \cong \frac{H_1(X)}{\Z}.
  \]
  Check as an exercise that this implies
  $H_1(X) \cong \Z \oplus \Z$. Also verify that
  $H_k(X) \cong 0$ for $k \ge 3$, so
  \[
    H_k(T^2) =
    \begin{cases}
      \Z & \text{if } k = 0, 2, \\
      \Z \oplus \Z & \text{if } k = 1, \\
      0 & \text{if } k \ge 3.
    \end{cases}
  \]
\end{example}

\section{Proof of the Mayer-Vietoris Theorem}

\begin{lemma}
  Consider two long exact sequences and maps between them
  as below:
  \begin{center}
    \begin{tikzcd}
      \cdots \ar[r] & C_{n + 1} \ar[r] \ar[d, "\gamma_{n + 1}"] & A_n \ar[r, "f_n"] \ar[d, "\alpha_n"] & B_n \ar[r, "g_n"] \ar[d, "\beta_n"] & C_n \ar[r, "h_n"] \ar[d, "\gamma_n"] & A_{n - 1} \ar[r] \ar[d, "\alpha_{n - 1}"] & \cdots \\
      \cdots \ar[r] & C'_{n + 1} \ar[r] & A'_n \ar[r, "f'_n"] & B'_n \ar[r, "g'_n"] & C'_n \ar[r, "h'_n"] & A'_{n - 1} \ar[r] & \cdots
    \end{tikzcd}
  \end{center}
  so that the diagram commutes.
  If $\gamma_n$ is an isomorphism, then the sequence
  \begin{center}
    \begin{tikzcd}
      \cdots \ar[r] & A_n \ar[r, "\Phi_n"] & A_n' \oplus B_n \ar[r, "\Psi_n"] & B_n' \ar[r, "\Gamma_n"] & A_{n - 1} \ar[r] & \cdots
    \end{tikzcd}
  \end{center}
  is exact, where $\Phi, \Psi, \Gamma$ are defined by
  \[
    \Phi_n(a) = (\alpha_n(a), f_n(a)) \quad
    \Psi_n(a', b) = \beta_n(b) - f_n'(a) \quad
    \Gamma_n(b') = h_n \circ \gamma_n^{-1} \circ g_n'(b').
  \]
\end{lemma}

\begin{proof}
  We check exactness at $B_n'$ (i.e. $\im \Psi_n = \ker \Gamma_n$), the rest is left
  as an exercise. We can see that
  \[
    \Gamma_n \circ \Psi_n(a', b)
    = \Gamma_n(\beta_n(b) - f_n'(a))
    = h_n \circ \gamma_n^{-1} \circ g_n'(\beta_n(b) - f_n'(a))
    = h_n(\gamma_n^{-1}(g_n'(\beta_n(b))))
  \]
  since $g_n' \circ f_n' = 0$ by assumption. We can
  then write ($\beta_n \circ \gamma_n' = \gamma_n \circ g_n$)
  \[
    \Gamma_n \circ \Psi_n(a', b)
    = h_n(\gamma_n^{-1}(\gamma_n(g_n(b))))
    = h_n(g_n(b))
    = 0
  \]
  since $h_n \circ g_n$ by assumption. So
  $\im \Psi_n \subseteq \ker \Gamma_n$. For the
  reverse inclusion, let $b' \in \ker \Gamma_n$, so
  $\gamma_n^{-1} \circ g_n'(b')$ lies in
  $\ker h_n$. Then there exists $b \in B_n$ such that
  $g_n(b) = \gamma_n^{-1} \circ g_n'(b')$, so
  (write $g_n'(b') = \gamma_n \circ g_n(b)$)
  \[
    g_n'(\beta_n(b) - b')
    = g_n'(\beta_n(b)) -  g_n'(b')
    = \gamma_n(g_n(b)) - g_n'(b')
    = 0.
  \]
  So there exists $a' \in A_n'$ such that
  $f_n'(a') = \beta_n(b) - b'$, and thus
  $b' = \beta_n(b) - f_n'(a') = \Psi_n(a', b)$.
\end{proof}

\begin{proof}[Proof Theorem \ref{thm:mayer-vietoris}]
  Consider the long exact sequence of the pairs
  $(A, A \cap B)$ and $(X, B)$:
  \begin{center}
    \begin{tikzcd}
      \cdots \ar[r] & H_n(A \cap B) \ar[r, "(i_A)_*"] \ar[d, "(i_B)_*"] & H_n(A) \ar[d, "(j_A)_*"] \ar[r, "(j_A)_*"] & H_n(A, A \cap B) \ar[r] \ar[d, "I_*"] & H_{n - 1}(A \cap B) \ar[r] \ar[d, "(i_B)_*"] & \cdots \\
      \cdots \ar[r] & H_n(B) \ar[r, "(j_B)_*"] & H_n(X) \ar[r] & H_n(X, B) \ar[r] & H_{n - 1}(B) \ar[r] & \cdots
    \end{tikzcd}
  \end{center}
  Verify as an exercise that $X / B \cong A / (A \cap B)$.
  So if $(A, A \cap B)$ and $(X, B)$ are good pairs, then
  $I_*$ is an isomorphism, since excision allows us to
  write
  \[
    H_n(A, A \cap B) \cong H_n(A / (A \cap B))
    \cong H_n(X / B) \cong H_n(X, B).
  \]
  From here, Mayer-Vietoris
  (in the case that $(A, A \cap B)$ and
  $(X, B)$ are good pairs) follows from the
  lemma. In general, $I_*$ is
  still an isomorphism even when
  $(A, A \cap B)$ and $(X, B)$ are not good pairs,
  but the argument is much more difficult (see Hatcher
  for the details).
\end{proof}

\section{Proof of Excision}
\begin{proof}[Proof of Theorem \ref{thm:excision}]
  The main idea is the following: For any
  $n$ and
  $\alpha = \sum m_i \sigma_i \in C_n(X, A)$, we can find
  $\beta \in C_{n + 1}(X, A)$ such that
  $\alpha + \partial \beta = \sum_{i = 1}^\ell m_i \tau_i$,
  where $\im \tau_1 \subseteq X - Z$ or
  $\im \tau_1 \subseteq A$.

  Call the above statement
  $(*)$, and we will prove it later. We will first
  prove excision assuming $(*)$.

  To see that $i_*$ is onto, let
  $[\alpha] \in H_n(X, A)$. Then $(*)$ implies that there
  is $\alpha' \in [\alpha]$ such that
  $\alpha' = \sum_i m_i \tau_i$, where
  the $\tau_i$ are as in $(*)$. Define
  $\alpha'' = \sum_{i, \im \tau_i \nsubseteq A} m_i \tau_i$.
  Note the following:
  \begin{enumerate}
    \item $\alpha'' = \alpha$ in
      $C_n(X) / C_n(A) = C_n(X, A)$.
    \item $\partial \alpha'' \in C_{n - 1}(A)$ (in
      fact is contained in $C_{n - 1}(A - Z)$).

      To see this, note that
      $- \partial \alpha'' = \partial(a' - \alpha'') - \partial \alpha'$, where
      \[
        \alpha'' \subseteq X - Z \quad \text{and} \quad
        \alpha' - \alpha'', \alpha' \subseteq A,
      \]
      so
      $\partial \alpha'' \subseteq X - Z$ and
      $\partial(a' - \alpha''), \partial \alpha' \subseteq A$,
      and thus $\partial \alpha'' \subseteq A - Z$.
    \item $\alpha'' \in C_n(X - Z) / C_n(A - Z)$
  \end{enumerate}
  Thus $\alpha''$ defines an element
  $[\alpha''] \in H_n(X - Z, A - Z)$.
  Clearly
  $i_*([\alpha'']) = [\alpha'] = [\alpha]$, so
  $i_*$ is onto.

  To see that $i_*$ is injective, suppose that
  $[\alpha] \in H_n(X - Z, A - Z)$ and
  $i_*([\alpha]) = 0$, so
  \[
    i \circ \alpha = \alpha = \partial \beta, \quad
    \text{for some $\beta \in C_{n + 1}(X, A)$}.
  \]
  Then $(*)$ implies that there exists
  $\gamma \in C_{n + 2}(X, A)$ such that
  $\beta + \partial \gamma = \sum m_i \tau_i$, where
  the $\tau_i$ are as in $(*)$. Clearly
  $\alpha = \partial(\beta + \partial \gamma)$.
  Again define $\beta' = \sum_{i, \im \tau_i \nsubseteq A} m_i \tau_i$, and
  note:
  \begin{enumerate}
    \item $\beta' \in C_n(X - Z)$.
    \item $\partial \beta' = \partial \beta + \text{terms with image in $A \cap (X - A) = A - Z$}$, so
      $\partial \beta' = \alpha$ in $C_n(X - Z, A - Z)$.
  \end{enumerate}
  Thus $[\alpha] = 0$ in $H_n(X - Z, A - Z)$, so
  $i_*$ is injective. This proves excision given $(*)$

  We now proceed to prove $(*)$. We will need
  \emph{barycentric subdivision}, defined as follows:
  \begin{enumerate}
    \item[0.] subdivision of $0$-simplex: do nothing;
    \item subdivision of $1$-simplex: divide into two
      equal pieces;
    \item subdivision of $2$-simplex:
      subdivide the edges $[e_0, e_1]$, $[e_1, e_2]$,
      $[e_2, e_0]$ of the triangle to get
      vertices $f_0, f_1, f_2$, add a vertex $e$
      at the center of the triangle, and add (6) edges
      between $e$ and the $e_i, f_i$:
      \[
        [e_0, e_1, e_2]
        = [e_0, f_0, e] \cup [f_0, e_1, e] \cup \dots;
      \]
    \item subdivision of $n$-simplex ($n \ge 3$):
      subdivide all faces, add center point, add all
      edges, add all $2$-simplices, \dots, add
      all $(n - 1)$-simplices.
  \end{enumerate}
  Show as an exercise that ``up to boundaries, we can
  subdivide simplices,'' i.e. if $\sigma : \Delta^n \to X$
  is a singular $n$-simplex and
  $\Delta^n = \Delta_1^n \cup \dots \cup \Delta_n^n$
  is its barycentric subdivision, then
  there is an $(n + 1)$-chain $\tau$ with
  \[
    \sigma + \partial \tau = \sum_{i = 1}^k (\pm \sigma|_{\Delta_i^n}).
  \]
  Note that if we continually barycentrically subdivide,
  the subsimplices have diameter $d \to 0$.
  Now given $\sigma : \Delta^n \to X$, note that
  $\{\sigma^{-1}(\Int A), \sigma^{-1}(X - \overline{Z})\}$
  is an open cover of $\Delta^n$. Since
  $\Delta^n$ is a compact metric space, there exists
  a Lebesgue number $\delta > 0$ for the cover.
  So if we subdivide enough, all subsimplices
  are either in $\sigma^{-1}(\Int A)$ or $\sigma^{-1}(X - \overline{Z})$, which
  proves $(*)$.
\end{proof}
