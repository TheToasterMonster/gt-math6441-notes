\chapter{Apr.~14 --- Poincar\'e Duality, Part 2}

\section{Applications of Poincar\'e Duality}

\begin{corollary}\label{cor:cup-product-pairing}
  Let $M$ be a closed oriented $n$-fold.
  Then the cup product pairing
  \begin{align*}
    \frac{H^p(M)}{\mathrm{tors}}
    \times \frac{H^{n - p}(M)}{\mathrm{tors}}
    &\longrightarrow \Z \\
    (\alpha, \beta)
    &\longmapsto \alpha \cup \beta([M])
  \end{align*}
  is non-degenerate, and if $\alpha$ is a
  generator of  $H^p(M) / \mathrm{tors}$, then there
  exists $\beta \in H^{n - p}(M) / \mathrm{tors}$ such that
  $\alpha \cup \beta$ is a generator of
  $\Z \cong H^n(M)$.
\end{corollary}

\begin{proof}
  Recall that the universal coefficient
  theorem says we have a short exact sequence:
  \begin{center}
    \begin{tikzcd}
      0 \ar[r] & \Tor(H_{p - 1}(M); \Z) \ar[r] & H^p(M; \Z) \ar[r, "\phi"] & \Hom(H_p(M); \Z) \ar[r] & 0
    \end{tikzcd}
  \end{center}
  where $\phi$ maps $\alpha \mapsto \phi(\alpha)$,
  where $\phi(\alpha) : H_p(M) \to \Z$ is
  given by $\sigma \mapsto \alpha(\sigma)$.
  Thus
  \[
    \frac{H^p(M)}{\mathrm{tors}} \cong
    \Hom(H_p(M); \Z)
    = \Hom(H_p(M) / \mathrm{tors}; \Z)
  \]
  since every homomorphism must send torsion
  elements to $0 \in \Z$. Poincar\'e
  duality says
  \[
    \frac{H^{n - p}(M)}{\mathrm{tors}} \cong
    \frac{H_p(M)}{\mathrm{tors}}
  \]
  by the map $\alpha \mapsto [M] \cap \alpha$,
  so we get a map $\Phi : H^p(M) / \mathrm{tors} \to \Hom(H_{n - p}(M) / \mathrm{tors}; \Z)$,
  where
  \begin{align*}
    \Phi(\alpha) : H^{n - p}(M) / \mathrm{tors}
    &\longrightarrow \Z \\
    \beta &\longmapsto \phi(\alpha)([M] \cap \beta)
    = \alpha([M] \cap \beta)
    = \alpha \cup \beta([M])
  \end{align*}
  This map is an isomorphism since it
  is the composition of two
  isomorphisms. So if
  $(\beta \cup \alpha)([M]) = 0$ for all
  $\beta$, then $\Phi(\alpha) = 0$, which
  implies $\alpha = 0$. Similarly, one can
  see that if
  $\beta \cup \alpha = 0$ for all $\alpha$, then
  $\beta = 0$. This shows that the pairing
  is non-degenerate.

  Now if $\alpha \in H^p(M) / \mathrm{tors}$
  is a generator, then there is a homomorphism
  $\psi : H^p(M) / \mathrm{tors} \to \Z$
  sending $\alpha$ to $1$ (since
  $H^p(M) / \mathrm{tors}$ is a free group), so
  from above, there exists an element
  $\beta \in H^{n - p}(M) / \mathrm{tors}$ such
  that $\Phi(\beta) = \psi$, i.e. $\psi(\gamma) = (\beta \cup \gamma)([M])$ for any $\gamma \in H^p(M) / \mathrm{tors}$. Then we get
  \[
    1 = \psi(\alpha) = (\beta \cup \alpha)([M]),
  \]
  so $\beta \cup \alpha$ must generate
  $H^n(M; \Z)$.
\end{proof}

\begin{corollary}
  The cohomology ring of $\C P^n$ is
  $H^*(\C P^n; \Z) \cong \Z[x] / \langle x^{n + 1} \rangle$,
  with $\deg x = 2$.
\end{corollary}

\begin{proof}
  Recall that $\C P^n$ has a CW structure with
  a single $0$-cell, $2$-cell, $4$-cell, \dots,
  $2n$-cell, so
  \[
    H^k(\C P^n; \Z) =
    \begin{cases}
      \Z & \text{if $k = 0, 2, 4, \dots, 2n$}, \\
      0 & \text{otherwise}.
    \end{cases}
  \]
  Note we have an inclusion map
  $i : \C P^{n - 1} \to \C P^n$. The
  long exact sequence of the pair
  $(\C P^n, \C P^{n - 1})$ is
  \begin{center}
    \begin{tikzcd}
      \cdots \ar[r] &
      H^k(\C P^n, \C P^{n - 1}) \ar[r] & H^k(\C P^n) \ar[r, "i^*"] & H^k(\C P^{n - 1}) \ar[r] & H^{k + 1}(\C P^n, \C P^{n - 1}) \ar[r] & \cdots
    \end{tikzcd}
  \end{center}
  Note that $\C P^n / \C P^{n - 1} \cong S^{2n}$.
  If $k < 2n - 1$, then we have
  \begin{center}
    \begin{tikzcd}
      0 \ar[r] & H^k(\C P^n) \ar[r, "i^*"] & H^k(\C P^{n - 1}) \ar[r] & 0
    \end{tikzcd}
  \end{center}
  so $i^* : H^k(\C P^n) \to H^k(\C P^{n - 1})$
  is an isomorphism for $k \le 2n - 1$ (when
  $k = 2n - 1$, both are $0$).

  Note that the corollary is clearly
  true for $\C P^1$, where
  $H^*(\C P^1; \Z) \cong \Z \oplus \Z$, where
  the first $\Z$ corresponds to $1$ with
  degree $0$ and the second $\Z$ corresponds
  to $x$ with degree $2$.

  Now we proceed by induction. Assume
  that the statement is true for $\C P^{n - 1}$,
  so we can find a generator $x \in H^2(\C P^{n - 1})$,
  and $x^k$ generates $H^{2k}(\C P^{n - 1}$)
  for $k \le n - 1$. We know
  $i^*$ induces an isomorphism from
  $H^2(\C P^n) \to H^2(\C P^{n - 1})$,
  so there exists a $y \in H^2(\C P^n)$ with
  $i^*(y) = x$. Therefore, $y^k$ must
  generate $H^{2k}(\C P^n)$ for $k \le n - 1$,
  since $i^*(y^k) = (i^*(y))^k = x^k$
  generates $H^{2k}(\C P^{n - 1})$.

  For $k = 2n$,
  Corollary \ref{cor:cup-product-pairing}
  says that there is a generator in
  $H^{2n - 2}(\C P^n)$ which cups with
  $y \in H^2(\C P^n)$ to be a generator of
  $H^{2n}(\C P^n)$, so $y \cup y^{n - 1}$
  generates $H^{2n}(\C P^n)$, which
  finishes the induction.
\end{proof}

\begin{corollary}
  Any homotopy equivalence $\C P^{2n} \to \C P^{2n}$
  preserves orientation.
\end{corollary}

\begin{proof}
  If $f : \C P^{2n} \to \C P^{2n}$ is a
  homotopy equivalence, then let
  $x$ generate $H^2(\C P^{2n})$ and we see
  \[
    f^*(x) = \pm x.
  \]
  Note that $x^{2n}$ generates $H^{4n}(\C P^{2n})$
  and $f^*(x^{2n}) = (f^*(x))^{2n} = (\pm x)^{2n} = x^{2n}$,
  so $f^*$ fixes the generator of $H^{4n}(\C P^{2n})$.
  Thus by the universal coefficient theorem,
  $f_*$ fixes the generator of $H_{4n}(\C P^{2n})$, i.e.
  it fixes a fundamental class, and hence
  an orientation.
\end{proof}

\begin{corollary}\label{cor:even}
  Let $M$ be a closed oriented $n$-manifold.
  Then
  \[
    \Free H_{n - k}(M) \cong \Free H_k(M) \quad
    \text{and} \quad \Tor H_{n - k}(M) \cong \Tor H_{k - 1}(M).
  \]
  If $n$ is odd, then $\chi(M) = 0$, and
  if $n = 4k + 2$, then $\chi(M)$ is even.
\end{corollary}

\begin{proof}
  The first part follows by the universal
  coefficient theorem and Poincar\'e duality.

  For the second part, recall the Euler
  characteristic is given by
  \[
    \chi(M)
    = \sum_{i = 0}^{n} (-1)^i
    = \sum_{i = 0}^{n} (-1)^i b_i,
  \]
  where the $b_i$ are called the \emph{Betti numbers}.
  If $n = 2m + 1$ is odd, we can write
  \begin{align*}
    \chi(M)
    = \sum_{i = 0}^m (-1)^i b_i
    + \sum_{i = m + 1}^{2m + 1} (-1)^i b_i
    &= \sum_{i = 0}^m (-1)^i b_i
    + \sum_{i = 0}^m (-1)^{2m + 1 - i} b_{2m + 1 - i} \\
    &= \sum_{i = 0}^m (-1)^i b_i
    + \sum_{i = 0}^m (-1)^{i + 1} b_i
    = 0.
  \end{align*}
  Now if $M$ is even-dimensional, then
  the above computation gives $\chi(M) = b_{n / 2} + \text{even number}$.
  So if $\dim M = 4m + 2$, then
  $\chi(M)$ is even if and only if
  $b_{2m + 1}$ is even. Corollary \ref{cor:cup-product-pairing}
  implies that
  \[
    \frac{H^{2m + 1}(M)}{\mathrm{tors}} \times
    \frac{H^{2m + 1}(M)}{\mathrm{tors}} \longrightarrow
    \Z
  \]
  is non-degenerate and skew-symmetric. It
  is now an exercise in linear algebra to
  show that if
  $V$ is a $k$-dimensional vector space
  and $q : V \times V \to \R$ is a
  non-degenerate
  skew-symmetric pairing, then $k$ is even.
  (Hint: If $W$ is a subspace of $V$ and
  $W^\perp = \{v \in V : q(v, w) = 0 \text{ for all } w \in W\}$, then
  $(W^\perp)^\perp = W$ and
  $\dim V = \dim W + \dim W^\perp$. Now given
  $v \in V$, there is some $v'$ such that
  $q(v, v') = 1$. Let $w = \Span\{v, v'\}$, and
  consider $W \oplus W^\perp$.)
  The corollary then follows.
\end{proof}

\begin{corollary}\label{cor:rank-even}
  Let $M^{2n} = \partial V^{2n + 1}$ with
  $V$ compact, orientable and $M$ connected.
  Then $\rank H^n(M)$ is even and we have
  the relation
  \[
    \dim (\ker (i_* : H_n(M) \to H_n(V)))
    = \dim (\im (i^* : H^n(V) \to H^n(M)))
    = \frac{1}{2} \dim H^n(M).
  \]
  Moreover, any two classes in
  $\im i^*$ cup to be zero.
\end{corollary}

\begin{proof}
  Consider the long exact sequence (by
  Poincar\'e-Lefschetz duality)
  \begin{center}
    \begin{tikzcd}
      H^n(V) \ar[r, "i^*"] & H^n(M) \ar[d, swap, "\cong"] \ar[d, "{[M] \cap \cdot}"] \ar[r, "\delta^*"] & H^{n + 1}(V, M) \ar[d, swap, "\cong"] \ar[d, "{[V, \partial V] \cap \cdot}"] \\
      & H_n(M) \ar[r, "i_*"] & H_n(V)
    \end{tikzcd}
  \end{center}
  So $[M] \cap (\im i^*) = [M] \cap (\ker \delta^*) = \ker i_*$, and we have
  \[
    \rank i^* = \dim \im i^*
    = \dim \ker i_*
    = \dim H_n(M) - \dim \im i_*
    = \dim H_n(M) - \dim \im i^*,
  \]
  where the second-to-last equality
  is by the rank-nullity theorem
  for modules. The last equality is by
  \[
    \langle i^* \alpha, c \rangle
    = (i^* \alpha)(c)
    = \alpha(i_* c)
    = \langle \alpha, i_* c \rangle,
  \]
  i.e. $i^*$ and $i_*$ are adjoints, and
  adjoints have the same rank. So
  $\dim H_n(M) = 2 \dim \im i^*$.
\end{proof}

\begin{corollary}
  If $M = \partial V$ is connected and
  $V$ is compact, orientable, then
  $\chi(M)$ is even.
\end{corollary}

\begin{proof}
  If $\dim M$ is odd or $\dim M = 4n + 2$, we have already
  seen that $\chi(M)$ is even by Corollary
  \ref{cor:even}.
  Now if $\dim M = 4m$, then the proof of
  Corollary \ref{cor:even} says that
  $\chi(M)$ is even if and only if
  $b_{2n}$ is even, which Corollary
  \ref{cor:rank-even} says that it is.
\end{proof}

\begin{corollary}
  $\C P^{2n}$ is not the boundary of a
  compact, oriented $(4n + 1)$-manifold.
\end{corollary}

\begin{proof}
  We have $\chi(\C P^{4n}) = 2n + 1$, which is
  not even.
\end{proof}
