\chapter{Jan.~13 --- Homotopy, Part 2}

\section{More on Homotopy Equivalence}
\begin{lemma}\label{lem:lemma-2}
  If $(X, A)$ is a CW pair and $A$ is contractible,
  then $X / A \simeq X$.\footnote{Here $X / A$ denotes
  the quotient of $X$ obtained by collapsing all of $A$
  to a single point.}
\end{lemma}

\begin{exercise}
  The following are some applications of this lemma:
  \begin{enumerate}
    \item Let $X$ be a connected graph (i.e. a
      $1$-complex), and let $A$ be an edge in $X$
      connecting distinct vertices. Then $A$ is
      contractible, so $X / A \simeq X$.
      Continuing this process, let $A$ be a maximal
      tree in $X$, which will also be contractible.
      Then $X \simeq X / A$, so any connected graph is
      homotopy equivalent to a \emph{wedge of circles}
      (with number of circles equal to the number of
      self-loops in the graph).\footnote{A \emph{wedge} of pointed spaces $(X, x_0) \lor (Y, y_0)$ is the space obtained from $X \sqcup Y$ by identifying $x_0$ and $y_0$.}
    \item Consider the space $X$ obtained by attaching
      a $1$-cell $A$ connecting the north and south
      poles on a sphere. Let $B$ be half of a great
      circle
      connecting the endpoints of $A$. Clearly $A$ and
      $B$ are both contractible. After collapsing
      $A$, we see that $X \simeq X / A$, which is
      $S^2$ with the north and south poles identified.
      On the other hand, by contracting $B$ instead
      we see that $X \simeq X / B$, which is
      $S^2 \lor S^1$.

      Note that after contracting $A$,
      the subset $B$ is actually no longer contractible.
    \item Let $X$ be a torus with attached disks
      $A_1, A_2, A_3$ in the tube of the torus.
      Then
      \[X \simeq ((X / A_1) / A_2) / A_3,\]
      which is three spheres lying in a circle, each
      attached to the next one at a single point.

      We can also obtain this space by considering
      the space $Y$ of
      three spheres attached in a line with an extra
      $1$-cell $B$ attached at the ends of the chain
      of circles. Also let $A$ be the union of halves
      of great circles going through the chain of
      circles, with the same endpoints as $B$. Then
      we can see that this creates the same space as
      before, so that $X \simeq Y / B \simeq Y$.
      On the other hand, by contracting $A$, we see
      that $Y \simeq Y / A = S^2 \lor S^2 \lor S^2 \lor S^1$.

      Of course, all of these spaces are then homotopy
      equivalent to each other by transitivity.
  \end{enumerate}
\end{exercise}

\begin{lemma}\label{lem:lemma-3}
  Let $(X, A)$ be a CW pair and $f, g : A \to Y$ 
  be homotopic maps. Then $X \cup_f Y \simeq X \cup_g Y$.
\end{lemma}

\begin{example}
  Let $Y = S^2$, and $X = D^2$, and $A = \partial D^2$.
  Let $g : A \to Y$ map $A$ to a great circle, and
  let $f : A \to Y$ map $A$ to the north pole.
  One can show as an exercise that $f \sim g$ (e.g. by
  pulling the equator towards the north pole).
  So the lemma says that $X \cup_g Y$, which is a
  sphere with a disk glued along its equator, is homotopy
  equivalent to $X \cup_f Y = S^2 \lor S^2$.
\end{example}

\section{Homotopy Extension Property}
\begin{remark}
  To prove both of these lemmas, we need the
  \emph{homotopy extension property} (HEP).
\end{remark}

\begin{definition}
  A space $X$ and a subspace $A \subseteq X$ have the
  \emph{homotopy extension property} if given
  $F_0 : X \to Y$ (for any $Y$ and $F_0$) and a
  homotopy $f_t : A \to Y$ such that $f_0 = F_0|_A$,
  then there is a homotopy $F_t : X \to Y$ such that
  $F_t|_A = f_t$ for every $t$.
\end{definition}

\begin{lemma}\label{lem:lemma-4}
  A pair $(X, A)$ has the homotopy extension property
  if and only if
  \[
    (X \times \{0\}) \cup (A \times [0, 1])
  \]
  is a retract of $X \times [0, 1]$.
\end{lemma}

\begin{proof}
  $(\Leftarrow)$ We will assume that $A$ is closed
  (not necessarily but makes the proof easier, and almost
  all examples satisfy this). By assumption, we
  have a retraction
  \[
    r : (X \times [0, 1]) \to (X \times \{0\}) \cup (A \times [0, 1]).
  \]
  Given $F_0 : X \to Y$ and $f_t : A \to Y$
  such that $f_0 = F_0|_A$, we can define a map
  \[
    \widetilde{F} : (X \times \{0\}) \cup (A \times [0, 1]) \to Y
  \]
  by $x \mapsto F_0(x)$ on $X \times \{0\}$ and
  $(a, t) \mapsto f_t(a)$ on $A \times [0, 1]$.
  This map $\widetilde{F}$ is continuous since the definitions of
  $\widetilde{F}$ agree on the intersection and the
  intersection $A \times \{0\}$ is closed. Now define
  \[F : X \times [0, 1] \to Y\]
  by $F = \widetilde{F} \circ r$,
  which is a homotopy of $F_0$ that extends $f_t$.

  $(\Rightarrow)$ Let $Y = (X \times [0, 1]) \cup (A \times [0, 1])$. Let
  \[F_0 : X \to (X \times \{0\}) \cup (A \times [0, 1])\]
  be given by $x \mapsto (x, 0)$, and
  \[
    f_t : A \mapsto (X \times \{0\}) \cup (A \times [0, 1])
  \]
  be given by $a \mapsto (a, t)$. Then the homotopy
  extension property yields an extension
  \[F : X \times [0, 1] \to (X \times \{0\}) \cup (A \times [0, 1]),\]
  which is a retraction, as desired.
\end{proof}

\begin{lemma}\label{lem:lemma-5}
  If $(X, A)$ is a CW pair, then
  $(X \times \{0\}) \cup (A \times [0, 1])$ is a
  (deformation) retract of $X \times [0, 1]$.
  In particular, $(X, A)$ satisfies the homotopy
  extension property.
\end{lemma}

\begin{proof}
  The main idea is that for any disk $D^n$, the space
  $(D^n \times \{0\}) \cup (\partial D^n \times [0, 1])$
  is a deformation retract of $D^n \times [0, 1]$.
  To see this, let $D^n \subseteq \R^n$ be the unit
  disk and $D^n \times [0, 1] \subseteq \R^{n + 1}$.
  Let
  \[
    p = (0, \dots, 0, 2).
  \]
  For any $x \in D^n \times [0, 1]$, let
  $\ell_x$ be the line through $p$ and $x$. Note that
  \[
    \ell_x \cap ((D^n \times [0, 1] \cup (\partial D^n \times [0, 1]))
  \]
  is a unique point. Define $\widetilde{r}(x)$ to be this point,
  which yields a map
  \[\widetilde{r} : D^n \times [0, 1] \to (D^n \times \{0\}) \cup (\partial D^n \times [0, 1])\]
  Note that for $x \in (D^n \times \{0\}) \cup (\partial D^n \times [0, 1])$,
  then $\widetilde{r}(x) = x$ since the point of intersection is
  unique and $x$ is already in the intersection.
  Show as an exercise that $\widetilde{r}$ is continuous.
  Then setting
  \[
    \widetilde{r}_t = t \widetilde{r} + (1 - t) \id_{D^n \times [0, 1]}
  \]
  gives a deformation retraction
  of $D^n \times [0, 1]$ onto
  $(D^n \times \{0\}) \cup (\partial D^n \times [0, 1])$.

  To show the general case that
  $X \times [0, 1]$ retracts to
  $(X \times \{0\}) \cup (A \times [0, 1])$, we induct
  on the dimension of cells. Define
  $r$ on $X^{(0)} \times [0, 1] \to (X \times \{0\}) \cup (A \times [0, 1])$
  by the following:
  \begin{itemize}
    \item if a $0$-cell $D^0 \subseteq A$, then
      let $r$ be the identity on $D^0 \times [0, 1]$;
    \item if a $0$-cell $D^0$ is not in $A$, then
      send $D^0 \times [0, 1]$ to
      $D^0 \subseteq X \times \{0\}$.
  \end{itemize}
  Now inductively assume $r$ is defined on
  $X^{(k - 1)} \times [0, 1] \to (X \times \{0\}) \cup (A \times [0, 1])$.
  For each $k$-cell $D^k$,
  \begin{itemize}
    \item if $D^k \subseteq A$, then
      let $r$ be the identity on $D^k \times [0, 1]$;
    \item if $D^k$ is not in $A$, then note that
      $\partial D^k \times [0, 1] \to X^{(k - 1)} \times [0, 1]$
      is defined by induction, and
      we have an ``inclusion'' (here
      $a : \partial D^k \to X^{(k - 1)}$ is the attaching map for $D^k$)
      \begin{center}
        \begin{tikzcd}
          {D^k} \arrow[r, "i"] \arrow[rr, bend right=10, swap, "j"] & {X^{(k - 1)} \cup D^k}
          \arrow[r, "q"] & {(X^{(k - 1)} \cup D^k) / \{(x \in \partial D^k) \sim (a(x) \in X^{k - 1})\}}
        \end{tikzcd}
      \end{center}
    So we let $D^k \times \{0\} \to (X \times \{0\}) \cup (A \times [0, 1])$
    be the map $j$ into $X \times \{0\}$. This
    defines $r$ on
    \[
      (D^k \times [0, 1]) \cup (\partial D^k \times [0, 1]),
    \]
    which extends to $D^k \times [0, 1]$ by composing
    with the map $\widetilde{r}$ from above.
  \end{itemize}
  This inductively defines the retraction
  $r$ on all of $X \times [0, 1]$.
  The last claim follows by Lemma
  \ref{lem:lemma-4}.
\end{proof}

\begin{remark}
  Now we can prove the first two lemmas from the
  beginning of the day.
\end{remark}

\begin{proof}[Proof of Lemma \ref{lem:lemma-2}]
  We prove that the result holds for any $(X, A)$
  which satisfies the homotopy exttension property.
  We show that the quotient map $q : X \to X / A$
  has a homotopy inverse. Since $A$ is contractible,
  we know there exists a homotopy $f_t : A \to A \subseteq X$,
  such that $f_0 = \id_A$ and $f_1$ is constant.
  Let $F_0 : X \to X$ be the identity (note that
  $F_0|_A = \id_A$), so that the homotopy extension
  property gives a homotopy
  $F_t : X \to X$ extending $f_t$.
  Since $F_t(A) \subseteq A$ for every $t$, we get maps
  \[
    \widetilde{F}_t : X / A \to X / A,
  \]
  which are well defined since points in $A$ are mapped
  to points in $A$. Furthermore, $F_1(A) = \{\text{pt}\}$,
  so $F_1$ factors through $X / A$ to give a map
  $h : X / A \to X$ satisfying $F_1 = h \circ q$. This
  gives the diagram:
  \begin{center}
    \begin{tikzcd}
      X \arrow[d, swap, "q"] \arrow[r, "F_1"] & X \arrow[d, "q"] \\
      X / A \arrow [ru, "h"] \arrow[r, swap, "\overline{F}_1"] & X / A
    \end{tikzcd}
  \end{center}
  It is easy to check that $h \circ q = F_1$ and
  $q \circ h = \overline{F}_1$, so that the
  diagram commutes. But then
  \[
    h \circ q = F_1 \sim F_0 = \id_X \quad \text{and} \quad
    q \circ h = \overline{F}_1 \sim \overline{F}_0 = \id_{X / A},
  \]
  so $q$ is a homotopy equivalence.
\end{proof}

\begin{proof}[Proof of Lemma \ref{lem:lemma-3}]
  Let $F : A \times [0, 1] \to Y$ be a homotopy,
  which extends to $F : X \times [0, 1] \to Y$
  by the homotopy extension property. Consider
  the mapping cylinder
  \[
    M_F = (X \times [0, 1]) \cup_F Y,
  \]
  and one can show that
  $M_F \simeq X \cup_g Y \simeq X \cup_f Y$.
\end{proof}
