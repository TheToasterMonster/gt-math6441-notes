\chapter{Feb.~12 --- Covering Spaces, Part 3}

\section{Galois Correspondence}

\begin{definition}
  A space $X$ is \emph{semi-locally simply connected} if
  each point $x \in X$ has a neighborhood $U$ such that
  $\pi_1(U, x) \to \pi_1(X, x)$ is trivial.
\end{definition}

\begin{example}
  Let $X = \bigcup_{n = 1}^\infty S_{1 / n}(1 / n, 0)$,
  where $S_{1 / n}(1 / n, 0)$ is a circle of radius $1 / n$
  about the point $(1 / n, 0)$. Note that any
  neighorhood of $(0, 0)$ has non-trivial fundamental group
  that injects into $\pi_1(X, (0, 0))$.
  So $X$ is not semi-locally simply connected. On the other
  hand,
  let $X'$ be the cone on $X$:
  \[
    X' = (X \times [0, 1]) / (X \times \{0\}).
  \]
  Then $X'$ is semi-locally simply connected but not
  locally simply connected.
\end{example}

\begin{remark}
  All CW-complexes and manifolds are semi-locally
  simply connected.
\end{remark}

\begin{theorem}
  Let $X$ be a path connected, locally path connected,
  semi-locally simply connected space, and fix $x_0 \in X$.
  Then there is a one-to-one correspondence
  \[
  \left\{\substack{\displaystyle\text{basepoint preserving isomorphism classes} \\ \displaystyle\text{of covering spaces $(\widetilde{X}, p, \widetilde{x_0})$}}\right\}
    \longleftrightarrow
    \{\text{subgroups of $\pi_1(X, x_0)$}\}
  \]
  given by $(\widetilde{X}, p, \widetilde{x_0}) \mapsto p_*(\pi_1(\widetilde{X}, \widetilde{x_0}))$
  with inverse $H \mapsto (\widetilde{X}_H, p_H, \widetilde{x_0})$, which
  satisfies:
  \begin{enumerate}
    \item If $H \le K$, then $(\widetilde{X}_H, p_H, \widetilde{x}_H)$ is also
      a covering space of $(\widetilde{X}_K, p_K, \widetilde{x}_K)$.
    \item If $p_1$ in $(\widetilde{X}_1, p_1, \widetilde{x}_1)$
      lifts to a cover of $(\widetilde{X}_2, p_2, \widetilde{x}_2)$,
      then $p_{1 *}(\pi_1(\widetilde{X}_1, \widetilde{x}_1)) \le p_{2 *}(\pi_1(\widetilde{X}_2, \widetilde{x}_2))$.
    \item
      $[\pi_1(X, x_0) : H] = n$ if and only if
      $(\widetilde{X}_H, p_H, \widetilde{x}_H)$ is degree $n$.
  \end{enumerate}
  In addition, there is a one-to-one correspondence
  \[
    \{\text{covering spaces $(\widetilde{X}, p)$ of $X$}\}
    \longleftrightarrow \{\text{conjugacy classes of subgroups of $\pi_1(X, x_0)$}\}.
  \]
\end{theorem}

\begin{proof}
  Note that $(1)$ and $(2)$ follow from
  Corollary \ref{cor:corollary-25} once we have the
  one-to-one correspondence, and
  $(3)$ follows from Lemma \ref{lem:lemma-21}. Also,
  once we have a well-defined map
  $H \mapsto (\widetilde{X}_H, p_H, \widetilde{x}_H)$ such
  that $(p_H)_*(\pi_1(\widetilde{X}_H, \widetilde{x}_H)) = H$,
  then the first one-to-one correspondence follows from
  Corollary \ref{cor:corollary-25}(b). The second
  one-to-one correspondence follows from
  Corollary \ref{cor:corollary-25}(c).

  So we are left to construct
  $(\widetilde{X}_H, p_H, \widetilde{x}_H)$ given
  $H \le \pi_1(X, x_0)$. Given $H$, we define an equivalence
  relation on paths based at $x_0$: Call two based paths
  $\gamma, \eta : [0, 1] \to X$ (i.e. $\gamma(0) = \eta(0) = x_0$) $H$-equivalent
  if:
  \begin{enumerate}
    \item $\gamma(1) = \eta(1)$, and
    \item $[\gamma * \overline{\eta}] \in H$.
  \end{enumerate}
  Denote $H$-equivalence by $\gamma \sim_H \eta$ and its
  equivalence classes by $\langle \gamma \rangle$.
  Show as an exercise that this is an equivalence relation, and
  if $H = \{e\}$, then $\gamma \sim_H \eta$ if and only
  if $\gamma, \eta$ are homotopic rel endpoints. Let
  \begin{enumerate}
    \item $\widetilde{X}_H = \{\langle \gamma \rangle : \gamma \text{ if a path in $X$ based at $x_0$}\}$.

      Note that this is a set: We will need to put a
      topology on $\widetilde{X}_H$ later.
    \item $p_H : \widetilde{X}_H \to X : \langle \gamma \rangle \mapsto \gamma(1)$.
    \item $\widetilde{x}_H = \langle e_{x_0} \rangle$,
      where $e_{x_0}$ is the constant path $x \mapsto x_0$.
      Clearly $p_H(\widetilde{x}_H) = x_0$.
  \end{enumerate}
  Note that $p_H$ is onto, since $X$ is path connected.
  In order to put a topology on $\widetilde{X}_H$, we need to
  study the topology of $X$ in terms of paths. We claim that
  \[
    \mathcal{U} = \left\{
    U \subseteq X : \substack{\displaystyle\text{$U$ is open, path connected, and} \\ \displaystyle\text{$\pi_1(U, x) \to \pi_1(X, x)$ is trivial for some $x \in U$}}
    \right\}
  \]
  is a basis for the topology on $X$.\footnote{Recall that a collection of open sets in $X$ is a \emph{basis for the topology on $X$} if for every $x \in X$ and open set $U$ containing $X$, there exists a set $O$ in the collection such that $x \in O \subseteq U$.}
  To see this, note that if $\pi_1(U, x) \to \pi_1(X, x)$
  is trivial for some $x \in U$, then this is true
  for all $y \in U$ since $U$ is path-connected:
  \begin{center}
    \begin{tikzcd}
      \pi_1(U, x) \ar[d, "\phi_h"] \ar[r, "\text{triv}"] & \pi_1(X, x) \ar[d, "\phi_h"] \\ 
      \pi_1(U, y) \ar[r, dashed, "i_*"] & \pi_1(X, y) \ar[u]
    \end{tikzcd}
  \end{center}
  Also if $U \in \mathcal{U}$ and $V \subseteq U$ is open
  and path connected, then:
  \begin{center}
    \begin{tikzcd}
      \pi_1(V, x) \ar[r] \ar[rr, dashed, swap, bend right=20, "\text{triv}"] & \pi_1(U, x) \ar[r, "\text{triv}"] & \pi_1(X, x)
    \end{tikzcd}
  \end{center}
  so $V \in \mathcal{U}$ too. Now let $x \in X$ and
  $U$ be any open set in $X$ containing $x$. Since
  $X$ is semi-locally simply connected, there exists an
  open set $V$ containing $x$ such that
  $\pi_1(V, x) \to \pi_1(X, x)$ is trivial. Then
  $U \cap V$ is an open set containing $x$ and
  $\pi_1(U \cap V, x) \to \pi_1(X, x)$ is trivial.
  Now $X$ being locally path connected implies that there
  exists an open set $W$ such that $x \in W \subseteq U \cap V$
  and $W$ is path connected. Clearly
  $\pi_1(W, x) \to \pi_1(X, x)$ is trivial, so
  $W \in \mathcal{U}$. Thus $\mathcal{U}$ is a basis
  for the topology on $X$.

  Now for each $U \in \mathcal{U}$ and a path
  $\gamma$ from $x_0 $ to a point in $U$, set
  \[
    U_{\gamma} = \{\langle \gamma * \eta \rangle : \eta \text{ a path in $U$ such that $\eta(0) = \gamma(1)$}\}.
  \]
  Note that $U_\gamma$ is a subset of $\widetilde{X}_H$.
  We claim that $\{U_\gamma\}_{U \in \mathcal{U}, \gamma \text{ a path from $x_0$ to $x \in U$}}$ forms a basis
  for a topology on $\widetilde{X}_H$.\footnote{Recall that a collection of subsets of a set $X$ form a \emph{basis for a topology} on $X$ if for each pair of sets $U, V$ in the collection and $x \in U \cap V$, there is another set $W$ in the collection such that $x \in W \subseteq U \cap V$, and $X$ is the union of the sets in the collection.} To see this, note the following:
  \begin{enumerate}
    \item If $\langle \gamma \rangle = \langle \delta \rangle$, then $U_\gamma = U_\delta$ (so we can write $U_{\langle \gamma \rangle}$).

      This is because if $\gamma \sim_H \delta$, then
      $\gamma * \eta \sim_H \delta * \eta$
      for all paths $\eta$ in $U$ with $\eta(0) = \eta(1)$:
      \[
        \delta * \eta * \overline{(\gamma * \eta)}
        = \gamma * \eta * \overline{\eta} * \overline{\delta}
        \sim \gamma * \overline{\delta}.
      \]
    \item $p_H : U_{\langle \gamma \rangle} \to U$ is
      onto since $U$ is path connected.
    \item $p_H : U_{\langle \gamma \rangle} \to U$ is injective.

      This is because if $p_H(\langle \gamma * \eta \rangle) = p_H(\langle \gamma * \eta' \rangle)$,
      then $\gamma * \eta(1) = \gamma * \eta'(1)$, so
      $\eta * \overline{\eta'}$ is a loop in $U$ based
      at $x = \gamma(1)$. So $\eta * \overline{\eta'}$ is
      homotopic to a constant loop in $X$. Thus
      $\gamma * \eta \sim \gamma * \eta'$ rel endpoints, so
      \[
        [(\langle * \eta) * \overline{(\gamma * \eta')}]
        = [e_{x_0}] \in H.
      \]
      This gives $\langle \gamma * \eta \rangle = \langle \gamma * \eta' \rangle$, i.e. $p_H$ is injective.
    \item If $\langle \gamma' \rangle \in U_{\langle \gamma \rangle}$, then
      $U_{\langle \gamma' \rangle} = U_{\langle \gamma \rangle}$.

      This is because by hypothesis, there exists
      a path $\eta$ in $U$ such that
      $\langle \gamma' \rangle = \langle \gamma * \eta \rangle$,
      so we can take $\gamma * \eta$ to represent
      $\gamma'$ by note $(1)$. So if
      $\langle \delta \rangle \in U_{\langle \gamma' \rangle}$, then
      \[
        \delta = (\gamma * \eta) * \eta'
       \sim \gamma * (\eta * \eta'),
      \]
      so $\langle \delta \rangle \in U_{\langle \gamma \rangle}$.
      This shows that $U_{\langle \gamma' \rangle} \subseteq U_{\langle \gamma \rangle}$, and
      similarly one can prove that $U_{\langle \gamma \rangle} \subseteq U_{\langle \gamma' \rangle}$.
  \end{enumerate}
  Now if $\langle \delta \rangle \in U_{\langle \gamma \rangle} \cap V_{\langle \gamma' \rangle}$, then
  $U_{\langle \gamma \rangle} = U_{\langle \delta \rangle}$
  and $V_{\langle \gamma' \rangle} = V_{\langle \delta \rangle}$.
  by $(4)$. So if
  $W$ is any element of $\mathcal{U}$ such that
  $\delta(1) \in W \subseteq U \cap V$, then
  \[\langle \delta \rangle = W_{\langle \delta \rangle} \subseteq U_{\langle \delta \rangle} \cap V_{\langle \delta \rangle} = U_{\gamma} \cap U_{\gamma'}.\]
  Clearly $\widetilde{X}_H$ is the union of the sets
  in $\{U_{\gamma}\}$, so we have a basis for a basis
  on $\widetilde{X}_H$.

  We claim that with this topology,
  $(\widetilde{X}_H, p_H)$ is a covering
  space of $(X, x_0)$. For any
  $U \in \mathcal{U}$ and a path $\gamma$ from
  $x_0$ to a point in $U$,
  $p|_{U_{\langle \gamma \rangle}} : U_{\langle \gamma \rangle} \to U$ is a homeomorphism.
  To see this, first note $p_H$ is a bijection.
  Next, $p_H|_{U_{\langle \gamma \rangle}}$ is continuous
  since for any basic open set $V \subseteq U$ and
  path $\delta$ from $x_0$ to a point in $V$,
  \[
    (p_H|_{U_{\langle \gamma \rangle}})^{-1}(V) = V_{\langle \delta \rangle}
  \]
  which is open. This also shows that
  the image of a basic open set $V_{\langle \delta \rangle}$
  is an open set $V$, so $p_H|_{U_{\langle \gamma \rangle}}$
  is a homeomorphism. Note that this implies
  $p_H : \widetilde{X}_H \to X$ is continuous.
  Now if $x \in X$ is any point, let
  $U \in \mathcal{U}$ be an open set containing $x$.
  Then we have
  \[
    p^{-1}(U) = \bigcup_{\substack{\gamma \text{ a path from $x_0$} \\ \text{to a point in $U$}}} U_{\langle \gamma \rangle},
  \]
  and $p_H|_{U_{\langle \gamma \rangle}} : U_{\langle \gamma \rangle} \to U$ is a homeomorphism, so
  $U$ is evenly covered, i.e.
  $(\widetilde{X}_H, p_H)$ is a covering space.

  It only remains to show
  $(p_H)_*(\pi_1(\widetilde{X}_H, \widetilde{x}_H)) = H$
  in $\pi_1(X, x_{0})$.
  If $[\gamma] \in H$, let $\gamma_t(s)$ be the path
  \[
    \gamma_t : [0, 1] \to X : s \mapsto \gamma(ts).
  \]
  Note that $\widetilde{\gamma} : [0, 1] \to \widetilde{X}_H$
  given by $t \mapsto \langle \gamma_t \rangle$ is a
  loop in $\widetilde{X}_H$ since $\widetilde{\gamma}_0 = \gamma_0$
  (the constant path in $\widetilde{X}_H$)
  and $\widetilde{\gamma}(1) = \langle \gamma \rangle
  = \langle e_{x_0} \rangle$ since $[\gamma] \in H$.
  We also have $(p_H \circ \widetilde{\gamma})(t) = \gamma_t(1) = \gamma(t)$,
  so $p_H \circ \widetilde{\gamma} = \gamma$.
  So $\widetilde{\gamma}$ is a lift of $\gamma$ 
  and a loop, therefore $[\gamma] \in \im((p_H)_*)$
  by Lemma \ref{lem:lemma-21}(b). Now if
  $\gamma \notin H$, then the path
  $\widetilde{t}$ is not a loop since
  $\widetilde{\gamma}(1) = \langle \gamma \rangle \notin H$,
  so $\widetilde{\gamma}(1) \ne \langle e_{x_0} \rangle$.
  So $[\gamma] \notin \im((p_H)_*)$ by Lemma \ref{lem:lemma-21}(b).
  These two properties imply that
  $(p_H)_*(\pi_1(\widetilde{X}_H, \widetilde{x}_H)) = H$,
  which completes the proof of the theorem.
\end{proof}

\begin{remark}
  There is a ``lattice'' of subgroups of
  $\pi_1(X, x_0)$ and a ``lattice'' of covering
  spaces of $X$. The above theorem says that these are
  the same (similar as in Galois theory).
\end{remark}

\begin{remark}
  By the above correspondence, there is a unique covering
  space of $X$ corresponding to the trivial subgroup $\{e\} \subseteq \pi_1(X, x_0)$. This is
  called the \emph{universal cover} of $X$.
\end{remark}
