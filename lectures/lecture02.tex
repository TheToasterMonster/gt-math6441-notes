\chapter{Jan.~8 --- Homotopy}

\section{More on Homotopy}

\begin{example}
  For any space $X$, any map $f : X \to [0, 1]$ is
  homotopic to the map $g : X \to [0, 1]$ given by
  $x \mapsto 0$. To see this, we have the homotopy
  $\Phi : X \times [0, 1] \times [0, 1]$ defined by
  \[
    (x, t) \mapsto (1 - t) f(x).
  \]
  We can see that $\Phi(x, 0) = f(x)$ and
  $\Phi(x, 1) = 0 = g(x)$.
\end{example}

\begin{exercise}
  Show that homotopy is an equivalence relation on
  maps $X \to Y$.
\end{exercise}

\begin{definition}
  Let $C(X, Y) = \{\text{continuous maps from $X$ to $Y$}\}$.
  Let $[X, Y] = C(X, Y) / {\sim}$, i.e. homotopic
  maps are identified with each other.
\end{definition}

\begin{example}
  We have the following:
  \begin{enumerate}
    \item $[X, [0, 1]] = \{g\}$ for any space $X$, where
      $g$ is the map $x \mapsto 0$ as above.
    \item $[\{*\}, X] = \{\text{path components of $X$}\}$.
  \end{enumerate}
\end{example}

\section{Homotopy Groups}
\begin{definition}
  We call a space $X$ \emph{pointed} if there
  is a designated ``base point'' $x_0 \in X$.
  Given two pointed spaces $(X, x_0)$ and
  $(Y, y_0)$, we define
  \[
    [X, Y]_0 = \{\text{homotopy classes of maps of pairs
    $(X, \{x_0\}) \to (Y, \{y_0\})$}\}.
  \]
\end{definition}

\begin{definition}
  Let $y_0$ be the north pole in $S^n$, i.e.
  $S^n \subseteq \R^{n + 1}$ is the unit sphere and
  $y_0 = (0, \dots, 0, 1)$. The
  \emph{$n$th homotopy group}
  of a pointed space $(X, x_0)$ is
  $\pi_n(X, x_0) = [S^n, X]_0$.
\end{definition}

\begin{remark}
  The homotopy group $\pi_n(X, x_0)$ is in fact a
  group. We will study $\pi_1(X, x_0)$ next and it is
  called the \emph{fundamental group} of $(X, x_0)$.
\end{remark}

\begin{remark}
  For which $(Y, y_0)$ is $[Y, X]_0$ ``naturally'' a
  group for all $(X, x_0)$? Similarly, for which
  $(Y, y_0)$ is $[X, Y]_0$ a group for all $(X, x_0)$?
  Here, given a map $f : (X_1, x_1) \to (X_2, x_2)$,
  there is an obvious \emph{induced map}
  $f_* : [Y, X_1]_0 \to [Y, X_2]_0$ given by
  $[g] \mapsto [f \circ g]$. Similarly, there is a map
  $f^* : [X_2, Y]_0 \to [X_1, Y]_0$ given by
  $[g] \mapsto [g \circ f]$. In the questions above,
  ``naturally'' means that $f_*$ and $f^*$ are
  homomorphisms.
  for any $(X_1, x_1)$ and $(X_2, x_2)$.
  The (perhaps unsatisfying) answer is that
  a space satisfying the first condition is called
  an \emph{$H$-space}, and a space satisfying the second
  is called an \emph{$H'$-space}.
\end{remark}

\section{Homotopy Equivalence}
\begin{definition}
  We say that $f : X \to Y$ is the \emph{homotopy inverse}
  to a function $g : Y \to X$ if
  $f \circ g \sim \id_Y$ and $g \circ f \sim \id_X$,
  where $\id_X$ and $\id_Y$ are the identity maps
  on $X$ and $Y$.
  If $g$ has a homotopy inverse, then
  we call $g$ a \emph{homotopy equivalence} from $Y$ to
  $X$ and we call $X, Y$ \emph{homotopy equivalent}.\footnote{We will denote homotopy equivalence by $X \simeq Y$ or simply $X \sim Y$.}
\end{definition}

\begin{exercise}
  Show that homotopy equivalence is an equivalence
  relation.
\end{exercise}

\begin{lemma}
  The following are equivalent:
  \begin{enumerate}
    \item $X$ and $Y$ are homotopy equivalent.
    \item For any space $Z$, there is a one-to-one
      correspondence $\phi_Z : [X, Z] \to [Y, Z]$
      such that for all continuous maps
      $h : Z \to Z'$, the following diagram commutes:
      \begin{center}
        \begin{tikzcd}
          {[X, Z]} \ar[r, "\phi_Z"] \ar[d, "h_*"] & {[Y, Z]} \ar[d, "h_*"] \\
          {[X, Z']} \ar[r, "\phi_{Z'}"] & {[Y, Z']}
        \end{tikzcd}
      \end{center}
    \item For any space $Z$, there is a one-to-one
      correspondence $\phi^Z : [Z, X] \to [Z, Y]$
      such for that all continuous maps $h : Z \to Z'$,
      the following diagram commutes:
      \begin{center}
        \begin{tikzcd}
          {[Z', X]} \ar[r, "\phi^{Z'}"] \ar[d, "h^*"] & {[Z', Y]} \ar[d, "h^*"] \\
          {[Z, X]} \ar[r, "\phi^{Z}"] & {[Z, Y]}
        \end{tikzcd}
      \end{center}
  \end{enumerate}
\end{lemma}

\begin{proof}
  This is left as an exercise.
\end{proof}

\begin{remark}
  Based on the previous lemma, two spaces are
  homotopy equivalent if and only if homotopy classes
  of maps to and from the space are ``naturally
  equivalent.''
\end{remark}

\begin{example}
  We have the following:
  \begin{itemize}
    \item Homeomorphic spaces are homotopy equivalent.
    \item Let $X = S^1$ and $Y = S^1 \times [0, 1]$.
      We claim that $X$ is homotopy equivalent to $Y$.

      Define the maps $f : S^1 \to S^1 \times [0, 1]$ by
      $x \mapsto (x, 0)$ and $g : S^1 \times [0, 1] \to S^1$
      by $(x, t) \mapsto x$. Then we can see
      that $g \circ f : S^1 \to S^1$ maps
      $x \mapsto x$, so $g \circ f = \id_{S^1}$. On the
      other hand, the composition
      $f \circ g : S^1 \times [0, 1] \to
      S^1 \times [0, 1]$ maps $(x, t) \mapsto (x, 0)$.
      Now $f \circ g \sim \id_{S^1 \times [0, 1]}$ by
      homotopy. For instance, define
      $\Phi : (S^1 \times [0, 1]) \times [0, 1] \to
      (S^1 \times [0, 1])$ by
      $((x, t), s) \mapsto (x, st)$, so
      \[
        \Phi((x, t), 1) = (x, t) = \id_{S^1 \times [0, 1]}(x, t)
        \quad \text{and} \quad
        \Phi((x, t), 0) = (x, 0) = f \circ g.
      \]
      Thus $f$ is a homotopy equivalence from $S^1$ to
      $S^1 \times [0, 1]$. Note that
      $S^1 \times [0, 1]$ is the annulus.
  \end{itemize}
\end{example}

\begin{definition}
  A space is called \emph{contractible} if it is
  homotopy equivalent to a point.
\end{definition}

\begin{example}
  The spaces $[0, 1]$ and $\R^n$ are contractible
  (exercise).
\end{example}

\begin{definition}
  If $A \subseteq X$, then a \emph{retraction of $X$ to $A$}
  is a map $r : X \to A$ such that $r(a) = a$ for
  every $a \in A$. A \emph{deformation retraction} of
  $X$ to $A$ is a retraction $r : X \to A$
  that is homotopic rel $A$ to the identity map $\id_X$,
  i.e. we can find $\phi_t : X \to X$ for
  $t \in [0, 1]$ such that
  $\phi_0(x) = x$ and $\phi_1(X) \subseteq A$ and
  $\phi_t(x) = x$ for all $x \in A$ and $t \in [0, 1]$.
\end{definition}

\begin{remark}
  If $X$ deformation retracts to $A$, then $X$ is
  homotopy equivalent to $A$. To see this, suppose we
  have a homotopy $\phi_t : X \to X$ as above, and let
  $i : A \to X$ be the inclusion map. Then
  $\phi_1 \circ i = \id_A$ and $i \circ \phi_1 = \phi_1 \sim \phi_0 = \id_X$, so
  $\phi_1$ is a homotopy equivalence from $X$ to $A$.
\end{remark}

\begin{definition}
  Given two spaces $X, Y$ and a map $f : X \to Y$,
  the \emph{mapping cylinder} of $f$ is the space
  \[
    M_f = ((X \times [0, 1]) \cup Y) / {\sim},
  \]
  where the equivalence relation $\sim$ is given by
  by $(x, 1) \sim f(x)$ for $x \in X$.
\end{definition}

\begin{remark}
  The mapping cylinder $M_f$ deformation retracts to $Y$.
  To see this, consider the map $\widetilde{\phi}_t$
  given by
  $(x, s) \mapsto (x, (1 - t)s + t)$ on
  $X \times [0, 1]$ and $y \mapsto y$ on $Y$. Since
  $\widetilde{\phi}_t$ respects the equivalence
  relation, it descends to a map $\phi_t : M_f \to M_f$
  on the quotient space. Note that
  $\phi_0 = \id_{M_f}$ and $\phi_1(M_f) = Y \subseteq M_f$,
  and $\phi_t = \id_Y$ for all $t$. Thus $\phi_1$ is a
  deformation retraction. In particular,
  this means that $M_f \simeq Y$.
\end{remark}

\begin{remark}
  There are obvious inclusions
  $i : X \to M_f$ given by $x \mapsto (x, 0)$
  and $j : Y \to M_f$ given by $y \mapsto y$.
  Note that $\phi_1$ defined above is the homotopy
  inverse to $j$. Now we have the diagram
  \[
    \begin{tikzcd}
      X \ar[r, "f"] \ar[dr, "i"] & Y \ar[d, "j"] \\
      & M_f
    \end{tikzcd}
  \]
  where $j$ is a homotopy equivalence and
  $j \circ f \sim i$ (exercise).
\end{remark}

\begin{remark}
  The above remark shows the following ``slogan''
  of algebraic topology:
  \begin{quote}
    Any map is an inclusion up to homotopy.
  \end{quote}
\end{remark}

\begin{example}
  Let $X$ be three circles with two enclosed in a third
  bigger one, and let $Y$ be two circles enclosing the
  inner two circles of $X$
  connected by a line segment. Let $Z$ be the region
  inside by the outer circle of $X$ and outside
  the inner two circles of $X$.

  Define $f : X \to Y$
  to be the map which sends $x \in X$ to the point in
  $Y$ at the other end of an interval (points on the
  inner circles of $X$ are mapped by radial lines to the
  circles in $Y$, and points on the outer circle of $X$
  are mapped radially to either the circles or
  the line segment in $Y$). One can write an explicit
  formula for $f$ as an exercise.

  Then $Z$ is homeomorphic to $M_f$, and in particular
  $Z \simeq Y$. Similarly, $M_f$ is homotopy equivalent
  to two circles joined at a point, or a circle
  with a diameter.
\end{example}
