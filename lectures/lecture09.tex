\chapter{Feb.~5 --- Covering Spaces}

\section{Examples of Covering Spaces}

\begin{example}\label{example:covering}
  The following are more examples of covering maps:
  \begin{enumerate}
    \setcounter{enumi}{2}
    \item Define $p_n : S^1 \to S^1$ by
      $\theta \mapsto n \theta$.
      These are covering maps for each $n \in \Z$.
    \item Let $X$ be a wedge of two circles
      (corresponding to $a, b$), and
      let $\widetilde{X}$ be a circle with
      three outer circles attached, evenly spaced (the inner
      circle corresponding to $a_1, a_2, a_3$ and
      the three outer circles corresponding to
      $b_1, b_2, b_3$). Let $p$ map
      $a_i$ to $a$ and $b_i$ to $b$. This is a covering
      map.

      Write out a formula for $p$ and really check that
      $p$ is a covering map as an exercise.
    \item Again let $X$ be a wedge of two circles,
      labeled $a, b$. Let $\widetilde{X}$ be a wedge of
      two circles, with an extra circle attached on
      either side. Let $a_3, b_3$ be the extra circles
      on the left and right, let $b_1, b_2$ be
      the top and bottom halves of the circle on the
      left, and let $a_1, a_2$ be the top and bottom
      halves of the circles on the right. Let
      $p$ map $a_i$ to $a$ and $b_i$ to $b$. Then
      $p$ is a covering map.
    \item Consider the quotient map
      $p : S^2 \to \R P^2$.
      This map is a covering map.
      Note that each point
      in $\R P^2$ has a neighborhood whose preimage
      is two open sets on opposite sides of $S^2$.
  \end{enumerate}
\end{example}

\section{Covering Spaces and Lifting}
\begin{lemma}
  Let $(\widetilde{X}, p)$ be a covering space of a
  connected space $X$. Then the cardinality
  $|p^{-1}(x)|$ is independent of $x \in X$.
\end{lemma}

\begin{proof}
  Fix $x_0 \in X$ and let $n = |p^{-1}(x_0)|$. Let
  $A = \{x \in X : |p^{-1}(X)| = n\}$, and note that
  $A \ne \varnothing$ since $x_0 \in A$. We will
  show that $A$ is both open and closed, which
  implies that $A = X$ by connectedness.

  To see that $A$ is open, let $x \in A$.
  By the definition of a covering space, there exists
  an open set $U$ in $X$ such that $x \in U$ and
  $p^{-1}(U) = \{U_1, \dots, U_n\}$. Thus for
  any $x' \in U$, we have
  \[p^{-1}(x') \subseteq p^{-1}(U) = \{U_1, \dots, U_n\},\]
  and $p^{-1}(x') \cap U_i = \{\text{pt}\}$ since
  $p|_{U_i} : U_i \to U$ is a homeomorphism.
  Thus $|p^{-1}(x')| = n$, so $x' \in A$. This holds
  for each $x' \in U$, so $U \subseteq A$, i.e. $A$
  is an open set.

  One can make a similar argument to check that
  if $x \notin A$, then there exists a open set $U$ about
  $x$ such that $U \cap A = \varnothing$.
  This shows that $X \setminus A$ is open, i.e.
  $A$ is closed, which completes the proof.
\end{proof}

\begin{definition}
  We call $|p^{-1}(x)|$ the \emph{degree} of the
  covering space.
\end{definition}

\begin{definition}
  If $(\widetilde{X}, p)$ is a covering space
  for $X$ and $f : Y \to X$ is a continuous map,
  then a \emph{lift} of $f$ to $\widetilde{X}$ is
  a map $\widetilde{f} : Y \to \widetilde{X}$
  such that $p \circ \widetilde{f} = f$, i.e.
  the following diagram commutes:
  \begin{center}
    \begin{tikzcd}
      & \widetilde{X} \ar{d}{p} \\
      Y \ar{r}{f} \ar[dashed]{ur}{\widetilde{f}} & X
    \end{tikzcd}
  \end{center}
  If $f(y_0) = x_0$ and $\widetilde{x}_0 \in \widetilde{X}$
  with $p(\widetilde{x}_0) = x_0$, then
  $\widetilde{f}$ is a \emph{lift of $f$ based at
  $\widetilde{x}_0$} if $\widetilde{f}$ is a lift
  and $\widetilde{f}(y_0) = \widetilde{x}_0$.
\end{definition}

\begin{lemma}\label{lem:general-lift}
  Let $(\widetilde{X}, p)$ be a covering space of
  $X$, $x_0 \in X$, and $\widetilde{x}_0 \in p^{-1}(x_0)$.
  Then
  \begin{enumerate}[(a)]
    \item each path $\gamma : [0, 1] \to X$ based
      at $x_0$ has a unique lift $\widetilde{\gamma} : [0, 1] \to X$
      based at $\widetilde{x}_0$.
    \item if $H : Y \times [0, 1] \to X$ is a homotopy
      with $h_0(y) = H(y, 0)$ and
      $\widetilde{h}_0 : Y \to \widetilde{X}$ a lift of
      $h_0$, then there is a unique lift
      $\widetilde{H} : Y \times [0, 1] \to \widetilde{X}$
      of $H$ such that $\widetilde{H}(y, 0) = h_0(y)$.
  \end{enumerate}
  The above properties are called \emph{path lifting}
  and \emph{homotopy lifting}.
\end{lemma}

\begin{proof}
  $(a)$ The proof of this part is exactly the proof of
  part $(a)$ in Lemma \ref{lem:lifting}.

  $(b)$ The proof from Lemma \ref{lem:lifting} works
  if $Y = [0, 1]$. For the general case, see proof
  of Theorem \ref{thm:theorem-23}.
\end{proof}

\section{Connections to the Fundamental Group}
\begin{lemma}\label{lem:lemma-21}
  If $(\widetilde{X}, p)$ is a path connected
  covering space of $X$ and $x_0 \in X$,
  $\widetilde{x}_0 \in p^{-1}(x_0)$, then the map
  $p_* : \pi_1(\widetilde{X}, \widetilde{x}_0) \to
  \pi_1(X, x_0)$ satisfies the following:
  \begin{enumerate}
    \item $p_*$ is injective;
    \item the image of $p_*$ is the set of loops
      in $\pi_1(X, x_0)$ that when lifted are
      loops in $\widetilde{X}$ based at $\widetilde{x}_0$;
    \item the index $[\pi_1(X, x_0) : p_*(\pi_1(\widetilde{X}, \widetilde{x}_0))]$ is the degree of $(\widetilde{X}, p)$.
  \end{enumerate}
\end{lemma}

\begin{proof}
  $(1)$ Suppose that $p_*([\gamma]) = [e]$, so
  there exists a homotopy $H$ in $X$ between $x_0$ and
  $p \circ \gamma$. Note that $\gamma$ is a lift
  of $H(t, 0)$, so Lemma \ref{lem:general-lift} says
  $H$ lifts to a homotopy $\widetilde{H}$ starting
  at $\gamma$ in $\widetilde{X}$.
  Note that $H|_{\{0\} \times [0, 1]}$ is a constant
  loop and the loop $t \mapsto \widetilde{x}_0$ is a
  lift of $H|_{\{0\} \times [0, 1]}$ based at
  $\widetilde{x}_0$, so by uniqueness we see that
  $\widetilde{H}|_{\{0\} \times [0, 1]} = \widetilde{x}_0$.
  Similarly, we see that
  $\widetilde{H}_{\{1\} \times [0, 1]} = \widetilde{x}_0$
  and $\widetilde{H}_{[0, 1] \times \{1\}} = \widetilde{x}_0$.
  Thus $\widetilde{H}$ is a homotopy of loops based
  at $\widetilde{x}_0$ from $\gamma$ to the constant
  loop, i.e. $[\gamma] = [e_{x_0}]$. So
  $p_*$ is injective.

  $(2)$ Clearly if $[\gamma] \in \pi_1(X, x_0)$ lifts
  to a loop $\widetilde{\gamma}$ based at
  $\widetilde{x}_0$, then $[\gamma] = p_*(\widetilde{\gamma})$, so
  $[\gamma]$ is in the image of $p_*$. Now if
  $[\eta] = p_*([\gamma])$, then
  $\eta \sim p \circ \gamma$ in $X$. Let
  $\widetilde{\eta}$ be the lift of $\eta$ based
  at $\widetilde{x}_0$. By Lemma \ref{lem:general-lift},
  the homotopy $\eta \sim p \circ \gamma$ lifts
  to a homotopy $\widetilde{\eta} \sim \gamma$ rel
  endpoints. But $\gamma$ is a loop, so
  $\widetilde{\eta}$ must be a loop.

  $(3)$ Let $H = p_*(\pi_1(\widetilde{X}, \widetilde{x}_0)) \le \pi_1(X, x_0)$.
  If $[\gamma] \in \pi_1(X, x_0)$ and
  $[\delta] \in H$, then note that by part (2),
  $\delta$ lifts to a loop $\widetilde{\delta}$ based
  at $\widetilde{x}_0$. Let $\widetilde{\delta * \gamma}$
  be a lift of $\delta * \gamma$ based at
  $\widetilde{x}_0$, and note that
  $\widetilde{\gamma}(1) = \widetilde{\delta * \gamma}(1)
  = \widetilde{\delta} * \widetilde{\gamma}(1)$.
  This allows us to define
  \[
    \phi : \{\text{right cosets of $H$}\} \to p^{-1}(x_0)
  \]
  by $H[\gamma] \mapsto \widetilde{\gamma}(1)$, which
  is well-defined by the above arguments.

  If $\widetilde{x}_1 \in p^{-1}(x_0)$, then let
  $\widetilde{\gamma}$ be a path in $\widetilde{X}$
  from $\widetilde{x}_0$ to $\widetilde{x}_1$. Let
  $\gamma = p \circ \widetilde{\gamma}$, which is
  a loop in $X$ based at $x_0$. Clearly
  $\phi(H[\gamma]) = \widetilde{\gamma}(1) = \widetilde{x}_1$,
  so $\phi$ is onto.
  Now suppose that
  $\phi(H[\gamma]) = \phi(H[\eta])$. If
  $\widetilde{\gamma}, \widetilde{\eta}$ are
  lifts of $\gamma$ based at $\widetilde{x}_0$, then
  $\widetilde{\gamma}(1) = \widetilde{\eta}(1)$.
  Thus $\widetilde{\gamma} * \overline{\widetilde{\eta}}$
  is a loop in $\widetilde{X}$, so
  \[
    p_*([\widetilde{\gamma} * \overline{\widetilde{\eta}}])
    = [\gamma] * [\overline{\eta}]
    = [\gamma] * [\eta]^{-1} \in H.
  \]
  This gives $H[\gamma] = H[\eta]$, so $\phi$ is injective.
  Thus $\phi$ is a bijection, so
  $[\pi_1(X, x_0) : H] = |p^{-1}(x_0)|$.
\end{proof}

\begin{example}
  Recall the following examples from before:
  \begin{enumerate}
    \item For the covering map $p : \R \to S^1$, we
      have $p_* : \pi_1(\R, 0) \to \pi_1(S^1, (1, 0))$
      which sends $e \mapsto 0$, if we view
      $p_*$ as $p_* : \{e\} \to \Z$. One can see
      all three properties of the above lemma in this
      example, in
      particular that
      $[\pi_1(S^1) : \pi_1(\R)] = \infty = |p^{-1}((1, 0))|$.
    \item Let $p_n : S^1 \to S^1$
      send $\theta \mapsto n\theta$. Then
      $(p_n)_* : \pi_1(S^1, (1, 0)) \to \pi_1(S^1, (1, 0))$, which we can view
      as a map $(p_n)_* : \Z \to \Z$, or $(p_n)_* : \langle g | \ \rangle \to \langle h | \ \rangle$.
      Then $(p_n)_*$ maps $g \mapsto h^n$, so the subgroup
      of $\Z$ corresponding to this covering space
      is $n\Z$. Again we can see all three properties
      of the above lemma, in particular that
      we have $[\Z : n\Z] = n = |p_n^{-1}((1, 0))|$.
  \end{enumerate}
\end{example}

\begin{exercise}
  Check the properties of the above lemma explicitly
  for Example \ref{example:covering}(4). Note that
  picking different base points can yield different
  images of $p_*$.
\end{exercise}
