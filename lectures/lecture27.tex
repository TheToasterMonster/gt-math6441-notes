\chapter{Apr.~21 --- Poincar\'e Duality, Part 3}

\section{Cohomology with Compact Support}

\begin{definition}
  If $M$ is an $n$-manifold, then let
  $I$ be the set of all compact subsets of
  $M$, directed by inclusion. Note that
  $K \le K'$ implies that we have
  an inclusion $i : (M, M - K') \to (M, M - K)$, so
  \[
    i^* : H^q(M, M - K; R)
    \longrightarrow H^q(M, M - K'; R).
  \]
  The collection $\{H^q(M, M - K; R)\}_{K \in I}$
  is a directed system of $R$-modules.
  Define
  \[
    H^q_c(M; R) = \varinjlim_{K \in I} H^q(M, M - K; R),
  \]
  which is called the
  \emph{cohomology with compact support}.
\end{definition}

\begin{remark}
  Note the following:
  \begin{enumerate}
    \item If $M$ is compact, then
      $M$ is final in $I$, so
      $H^q(M; R) \cong H^q_c(M; R)$.
    \item One can think of elements
      in $H^q_c(M; R)$ as cochains
      that vanish off of some compact
      subset of $M$.
  \end{enumerate}
\end{remark}

\section{Poincar\'e Duality, Revisited}
\begin{remark}
  Fix an $R$-orientation on $M$ (recall
  this means that we have a section
  $\sigma : M \to M_R$ such that $\sigma(x)$
  generates $H_n(M, M - \{x\})$ for all
  $x$), and let $K$ be a compact
  set in $M$.

  Then Lemma \ref{lem:direct-limit}
  gives a class $\alpha_K \in H_n(M, M - K; R)$
  such that $i^x_*(\alpha_K) = \sigma(x)$,
  where $i^x : (M, M - K) \to (M, M - \{x\})$. The cap product gives
  a map
  \[
    H_n(M, M - K; R) \times H^p(M, M - K; R)
    \longrightarrow
    H_{n - p}(M; R).
  \]
  So $\alpha_K \cap \cdot$ gives a map
  $H^p(M, M - K; R) \to H_{n - p}(M; R)$,
  and if $K \subseteq K'$, then
  \begin{center}
    \begin{tikzcd}
      H^n(M, M - K; R) \ar[d] \ar[r, "\alpha_K \cap \cdot"] & H_{n - p}(M; R) \\
      H^p(M, M - K'; R) \ar[ur, swap, "\alpha_{K'} \cap \cdot"]
    \end{tikzcd}
  \end{center}
  commutes.
  Thus we get a map $H^p_c(M; R) \overset{D_M}{\longrightarrow} H_{n - p}(M; R)$.
\end{remark}

\begin{theorem}[Poincar\'e duality, revised]\label{thm:poincare-duality-revised}
  If $M$ is an $R$-oriented $n$-manifold,
  then
  \[D_M : H^p_c(M; R) \to H_{n - p}(M; R)\]
  is an isomorphism.
\end{theorem}

\begin{remark}
  If $M$ is compact, then this
  proves Theorem \ref{thm:poincare-duality}
  since $H^p_c(M; R) \cong H^p(M; R)$.
\end{remark}

\begin{proof}[Proof of Theorem \ref{thm:poincare-duality-revised}]
  We prove the theorem in three steps:
  \begin{enumerate}[(i)]
    \item If the theorem is true for
      open sets $U$, $V$, and $U \cap V$
      in $M$, then it is true for $U \cup V$.
    \item Let $\{U_i\}$ be a collection
      of open sets which is
      totally ordered by inclusion. If
      the theorem is true for each $U_i$,
      then it is true for $U = \bigcup_i U_i$.
    \item The theorem is true for any
      open set $U$ contained in a coordinate
      chart of $M$.
  \end{enumerate}
  Once we know (i)-(iii), the we can
  finish the proof as follows: Recall
  \emph{Zorn's lemma}, which says that
  if $P$ is a partially ordered set
  such that every chain (i.e. a totally
  ordered subset) has an upper bound,
  then $P$ has a maximal element.
  By step (ii) and Zorn's lemma, there is
  a maximal element $U \subseteq M$
  for which the theorem is true. If
  $U \ne M$, then let $x \in M - U$.
  There exists an open set $V$ such that
  $x \in V$ and $V$ is in a coordinate
  chart of $M$, so the theorem is
  true for $V$ and $U \cap V$ by (iii).
  So by (i), the theorem is true for
  $U \cup V$, which contradicts the
  assumption that $U$ was maximal.
  Therefore, $U = M$.

  (iii) It suffices to prove the statement
  for open sets in $\R^n$. We consider
  the following steps:
  \begin{enumerate}[(a)]
    \item Let $U$ be a convex open set in
      $\R^n$.

      Check as an exercise that $U$ is
      homeomorphic to $\R^n$. Therefore,
      in this case,
      we just need to prove the result
      for $\R^n$. Let $K_r$ be the closed
      (hence compact) ball of radius $r$
      in $\R^n$, centered at the origin.
      The collection $\{K_r\}_{r \in (0, \infty)}$
      is final in the directed set of
      all compact sets. So
      \[
        H^p_c(\R^n; R)
        = \varinjlim_{K} H^p(\R^n, \R^n - K)
        = \varinjlim_{K_r} H^p(\R^n, \R^n - K_r).
      \]
      Note that $H^p(\R^n, \R^n - K_r) = 0$
      unless $p = n$ (since
      $\R^n / (\R^n - K_r) \cong S^n$),
      so $H^p_c(\R^n) = 0$ for $p \ne n$.
      Also, $H_{n - p}(\R^n) = 0$
      for all $p \ne n$, so the theorem
      holds for $p \ne n$.

      If $p = n$, then
      $H^n(\R^n, \R^n - K_r) \cong R$ for
      all $r$. Thus the inclusions are
      isomorphisms, so
      \[H^n_c(\R^n; R) \cong R \quad \text{and} \quad
      H_0(\R^n) \cong R.\]
      Now consider $\alpha_{K_r} \cap \cdot : H^n(\R^n, \R^n - K_r) \to H_0(\R^n)$.
      Recall the formula
      \begin{align*}
        H^n(\R^n, \R^n - K_r) \times
        H^n(\R^n, \R^n - K_r)
        &\longrightarrow H_0(\R^n) \\
        (\alpha, \beta)
        &\longmapsto \beta({}_n\alpha) \alpha_0 = \beta(\alpha)
      \end{align*}
      where ${}_n\alpha$ denotes the the
      front $n$-face of $\alpha$. So
      the map $\alpha_{K_r} \cap \cdot$
      is given by $\beta \mapsto \beta(\alpha_{K_r})$.
      We know $\alpha_{K_r}$ is a generator
      of $H_n(\R^n, \R^n - K_r)$ (else it
      could not map to generators of
      $H_n(\R^n, \R^n - \{x\})$ for all
      $x \in K_r$). Let
      $\beta \in \Hom(H^n(\R^n, \R^n - K_r), R) \cong H^n(\R, \R^n - K_r)$
      that evaluates to $1$ on $\alpha_{K_r}$,
      so $\beta$ generates $H^n(\R^n, \R^n - K_r)$
      and $\alpha_{K_r} \cap \beta = \beta(\alpha_{K_r}) = 1 \in R$.

      Thus $\alpha_{K_r} \cap \cdot$ is an
      isomorphism, so the
      theorem is true for convex open sets
      in $\R^n$.
    \item Let $U$ be a general open set
      in $\R^n$.

      Let $\{b_i\}$ be a countable dense
      set in $U$. For each $i$, let $U_i \subseteq U$ be an
      open ball centered at $b_i$, so
      $U = \bigcup_i U_i$.
      Set $V_1 = U_1$ and
      $V_i = V_{i - 1} \cup U_i$ for
      $i > 1$. The theorem is true for the
      $V_i$ by the following claim, (from
      which the theorem will follow for $U$
      by (ii)):
      \begin{quote}
        \textbf{Claim.} The theorem is
        true for finite unions of convex
        open sets in $\R^n$.
      \end{quote}
      To prove the claim, we induct on the
      number of convex sets. The claim is
      true for only one convex set by (a).
      Now assume the claim is true
      for unions of $i$ convex sets, with
      $i < k$. Let $A_1, \dots, A_k$
      be $k$ convex open sets in $\R^n$.
      The result is true for each $A_i$
      by (a), and the result is true
      for
      \[
        (A_1 \cup \dots \cup A_{k - 1})
        \quad \text{and} \quad
        A_k \cap (A_1 \cup \dots \cup A_{k - 1})
        = (A_k \cap A_1) \cup \dots \cup (A_k \cap A_{k - 1})
      \]
      by induction, since
      $A_k \cap A_i$ is convex.
      So the theorem holds for
      $A_1 \cup \dots \cup A_k$ by (i).
  \end{enumerate}

  (ii) From Lemma \ref{lem:direct-limit},
  we have $\varinjlim H_{n - p}(U_i) \cong H_{n - p}(U)$, induced by inclusion.
  Now if $U_i \subseteq U_j$ and
  $K \subseteq U_i$ is compact, then
  excision gives an isomorphism
  \[H^p(U_j, U_j - K) \longrightarrow H^p(U_i, U_i - K).\]
  The inverse of this isomorphism
  gives the composition
  \[
    H^p(U_i, U_i - K) \longrightarrow H^p(U_j, U_j - K) \longrightarrow H^p_c(U_j).
  \]
  Taking direct limits, we obtain a map
  $H^p_c(U_i) \to H^p_c(U_j)$. We claim
  the following:
  \begin{quote}
    \textbf{Claim.}
      $\varinjlim_i H^p_c(U_i) = H^p_c(U)$
      and $D_U = \varinjlim_i D_{U_i}$.
  \end{quote}
  Given the claim, note that
  $D_{U_i} : H^p_c(U_i) \to H_{n - p}(U_i)$
  is an isomorphism by hypothesis. Since
  the direct limit of isomorphisms is
  an isomorphism, $D_U : H^p_c(U) \to H_{n - p}(U)$
  is an isomorphism.

  Now we prove the claim. As above, we get
  maps $H^p_c(U_i) \to H^p_c(U)$, so we
  get a map
  \[
    \lim_i H^p_c(U_i) \overset{G}\longrightarrow H^p_c(U)
  \]
  on the direct limit. For any
  $K \subseteq U$ compact, there exists
  some $j$ such that $K \subseteq U_i$
  for any $i \ge j$ (since the $U_i$ are
  increasing and an open cover of $U$).
  Thus, for $i \le j$, we get a map
  \[
    H^p(U, U - K) \underset{\text{excision}}{\overset{\cong}{\longrightarrow}} H^p(U_i, U_i - K)
    \longrightarrow H^p_c(U_i),
  \]
  so we get a map $H^p_c(U) \overset{H}{\longrightarrow} H^p_c(U_i)$.
  Show as an exercise that $G, H$ are
  inverses of each other.

  (i) This is an involved Mayer-Vietoris
  argument,
  see the \href{https://etnyre.math.gatech.edu/class/6441Spring21/SectionIV.pdf}{course notes}.
\end{proof}

\section{Next Steps in Algebraic Topology}

The following are some further topics to
study in algebraic topology:
\begin{enumerate}
  \item \emph{Homotopy groups}: Recall that
    $\pi_n(X, x_0) = [S^n, X]_0$, and
    for $f : X \to Y$, we get a map
    \[f_* : \pi_n(X, x_0) \to \pi_n(Y, f(x_0)).\]
    \begin{itemize}
      \item One has the following theorem:
        \begin{theorem}[Whitehead]
          If $f : X \to Y$ is a map of
          CW complexes and $f_* : \pi_n(X) \to \pi_n(Y)$
          is an isomorphism for all $n$,
          then $f$ is a homotopy equivalence.
        \end{theorem}
      \item For $n \ge 2$, the $n$th
        homotopy
        group $\pi_n(X, x_0)$ is abelian.
      \item $\pi_n$ is very, very hard
        to compute in general, e.g.
        \[
          \pi_n(S^2) \cong
          \begin{cases}
            0 & \text{if $n = 1$}, \\
            \Z & \text{if $n = 2, 3$}, \\
            \Z / 2\Z & \text{if $n = 4, 5, 6, 7, 8$}, \\
            \Z / 3\Z & \text{if $n = 9$}, \\
            \Z / 15\Z & \text{if $n = 10$}, \\
            \dots.
          \end{cases}
        \]
      \item Given any abelian group $G$
        and integer $n$, there exists a
        space $K(g, n)$ such that
        \[
          \pi_k(K(G, n)) =
          \begin{cases}
            G & \text{if $k = n$}, \\
            0 & \text{otherwise}.
          \end{cases}
        \]
      \item One has the following two theorems:
        \begin{theorem}[Brown]
          $H^n(X; G) \cong [X, K(G, n)]$.
        \end{theorem}

        \begin{theorem}[Hurewicz]
          Let $n \ge 2$.
          If $\pi_k(X) = 0$ for all
          $k < n$, then
          $\widetilde{H}_k(X) = 0$ for all
          $k < n$ and $H_n(X) \cong \pi_n(X)$.
        \end{theorem}
    \end{itemize}
  \item \emph{Spectral sequences}:
    A map $p : E \to B$ is a \emph{fibration}
    if it has the \emph{homotopy lifting property}, i.e. if
    $f_t : X \to B$ is a homotopy and
    $\widetilde{f}_0$ is a lift of
    $f_0$ to $E$ (i.e. $\widetilde{f}_0 : X \to E$
    and $p \circ \widetilde{f}_0 = f_0$),
    then there exists a lift of
    $\widetilde{f}_t$ for all $t$.
    Note that fibrations are everywhere,
    but computing the homology of a
    fibration is very difficult in general.

    A group is \emph{bigraded} if
    $E = \{E_{s, t}\}_{s, t \in \Z}$.
    A map $d : E \to E$ is
    of \emph{bidegree} $(a, b)$ if
    $d(E_{s, t}) = E_{s + a, t + b}$, and
    if $d^2 = 0$ also, then we call it a
    \emph{differential.}
    A \emph{spectral sequence} is a sequence
    $\{(E^n, d^n)\}$ where each
    $E^r$ is a bigraded group, $d^n$ is a
    differential of degree $(-r, r - 1)$,
    and $E^{r + 1} = H(E^r, d^r)$.

    \begin{theorem}[Leray-Serre]
      If $p : E \to B$ is a fibration, where
      $B$ is a simply connected CW complex,
      then there exists a spectral
      sequence with
      $E_{s, t}^2 = H_s(B; H_t(F))$,
      where $F = p^{-1}$ and
      ``$E^\infty$'' more or less
      gives us $H_*(E)$.
    \end{theorem}
  \item \emph{Obstruction theory and
    characteristic classes}:
    See the \href{https://etnyre.math.gatech.edu/class/6441Spring21/SectionIV.pdf}{course notes}.
\end{enumerate}
