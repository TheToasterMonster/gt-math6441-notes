\chapter{Feb.~17 --- Covering Spaces, Part 4}

\section{Deck Transformations}

\begin{definition}
  Let $p : \widetilde{X} \to X$ be a covering
  space. A \emph{deck transformation}, or
  a \emph{covering transformation}, is a
  covering space isomorphism, i.e. a
  homeomorphism $f : \widetilde{X} \to \widetilde{X}$
  such that $p \circ f = p$. The set
  $G(\widetilde{X})$ of deck transformations
  is clearly a group under composition.
\end{definition}

\begin{example}
  Consider the covering map $p_4 : S^1 \to S^1$
  given by $\theta \mapsto 4\theta$.
  Then each $\phi_k : S^1 \to S^1$ given by
  $\theta \mapsto \theta + (2\pi k) / 4$ satisfies
  $p_4 \circ \phi_k = p_4$, so the $\phi_k$ are deck
  transformations. Now suppose $f : S^1 \to S^1$
  is another deck transformation, so
  $f(\widetilde{x}_1) = \widetilde{x}_i$ for some
  $i$ (let the $\widetilde{x}_i$ be the four preimages
  of $\widetilde{x}_1$ under $p_4$).
  But there exists $k$ such that
  $\phi_k(\widetilde{x}_1) = \widetilde{x}_i$,
  and these are both lifts of $p$ which agree at
  $\widetilde{x}_1$, so
  $f = \phi_k$ by Lemma \ref{lem:lemma-24}.
  Thus $G(S^1 \xrightarrow{p_4} S^1) = \{\phi_0, \phi_1, \phi_2, \phi_3\} \cong \Z / 4\Z$ (a generator is $\phi_1$).
\end{example}

\begin{example}
  Let $p : \widetilde{X} \to X$ be as
  in Example \ref{example:covering}(5).
  Let $\widetilde{x}_1$ be the center intersection
  point in
  $\widetilde{X}$ and $\widetilde{x}_2, \widetilde{x}_3$
  be the two intersection points at the sides.
  Let $x_0$ be the intersection point in $X$. Note that
  if we lift $a$ to $\widetilde{X}$ based at
  $\widetilde{x}_1$, then we get a path,
  but if we lift $a$ to $\widetilde{X}$
  based at $\widetilde{x}_3$, then we get a loop.
  So if $\phi : \widetilde{X} \to \widetilde{X}$
  is a deck transformation, then it cannot take
  $\widetilde{x}_1$ to $\widetilde{x}_3$, since
  $\phi$ applied to a lift is a lift. Similarly, there
  are no deck transformations taking $\widetilde{x}_1$
  to $\widetilde{x}_2$. So any deck transformation
  takes $\widetilde{x}_1$ to $\widetilde{x}_1$.
  Thus Lemma \ref{lem:lemma-24} says that any deck
  transformation is the identity, i.e.
  $G(\widetilde{X}) = \{1\}$.
\end{example}

\begin{remark}
  In the first example, a covering map of degree
  $4$ gave a group of order $4$, whereas in
  the second example, a covering map of degree $3$
  gave a group of order $1$.
\end{remark}

\begin{definition}
  A covering space $p : \widetilde{X} \to X$ is called
  \emph{normal} if for every $x \in X$ and
  $\widetilde{x}, \widetilde{x}' \in p^{-1}(x)$,
  there exists $\phi \in G(\widetilde{X})$ such that
  $\phi(\widetilde{x}) = \widetilde{x}'$.
\end{definition}

\begin{remark}
  The first example is normal, while the second one is
  not.
\end{remark}

\begin{theorem}
  Let $p : (\widetilde{X}, \widetilde{x}_0) \to (X, x_0)$
  be a path connected, locally path connected covering
  space of $X$. Let
  $H = p_*(\pi_1(\widetilde{X}, \widetilde{x}_0)) \le \pi_1(X, x_0)$. Then
  \begin{enumerate}
    \item $(\widetilde{X}, p)$ is normal if and only
      if $H$ is a normal subgroup of $\pi_1(X, x_0)$;
    \item $G(\widetilde{X}) \cong N(H) / H$, where
      $N(H)$ is the normalizer of $H$, i.e.
      the largest subgroup of $\pi_1(X, x_0)$
      in which $H$ is normal.
  \end{enumerate}
\end{theorem}

\begin{proof}
  $(1)$ $(\Rightarrow)$ Let
  $[\gamma] \in \pi_1(X, x_0)$ and let
  $\widetilde{\gamma}$ be a lift of $\gamma$
  based at $\widetilde{x}_0$. Set
  $\widetilde{x}_1 = \widetilde{\gamma}(1)$.
  From the proof of Lemma \ref{lem:lemma-22},
  we have
  \[
    [\gamma] p_*(\pi_1(\widetilde{X}, \widetilde{x}_0)) [\gamma]^{-1} = p_*(\pi_1(\widetilde{X}, \widetilde{x}_1)).
  \]
  Now $\widetilde{X}$ normal implies that there exists
  $\phi \in G(\widetilde{X})$ such that
  $\phi_1(\widetilde{x}_0) = \widetilde{x}_1$, so we
  get an isomorphism
  \[
    \phi_* : \pi_1(\widetilde{X}, \widetilde{x}_1) \to \pi_1(\widetilde{X}, \widetilde{x}_0).
  \]
  Then since $p_* \circ \phi_*$ (as $\phi$ is a
  deck transformation), we have
  \[
    p_*(\pi_1(\widetilde{X}, \widetilde{x}_0))
    = p_* \circ \phi_* (\pi_1(\widetilde{X}, \widetilde{x}_1))
    = p_*(\pi_1(\widetilde{X}, \widetilde{x}_1)),
  \]
  so that
  $[\gamma] p_*(\pi_1(\widetilde{X}, \widetilde{x}_0)) [\gamma]^{-1} = p_*(\pi_1(\widetilde{X}, \widetilde{x}_0))$, i.e.
  $p_*(\pi_1(\widetilde{X}, \widetilde{x}_0))$ is normal.

  $(\Leftarrow)$ Let $\widetilde{x}_0, \widetilde{x}_1$
  be two points in $p^{-1}(x_0)$, and
  set $H_1 = p_*(\pi_1(\widetilde{X}, \widetilde{x}_1))$
  Let $h$ be a path from $\widetilde{x}_0$ to
  $\widetilde{x}_1$, and let $\gamma = p \circ h$.
  By Lemma \ref{lem:lemma-22}, we have
  $[\gamma] H_1 [\gamma]^{-1} = H$. Since $H$ is normal,
  we have $H = H_1$, i.e.
  \[
    p_*(\pi_1(\widetilde{X}, \widetilde{x}_1)) = p_*(\pi_1(\widetilde{X}, \widetilde{x}_0)).
  \]
  Now by Theorem \ref{thm:theorem-23}, there are
  lifts of $p$ to $\phi_1, \phi_2$ such that
  \begin{center}
    \begin{tikzcd}
      (\widetilde{X}, \widetilde{x}_1) \ar[r, "\phi_1"] \ar[dr, "p"'] & (\widetilde{X}, \widetilde{x}_0) \ar[d, "p"] \ar[r, "\phi_2"] & (\widetilde{X}, \widetilde{x}_0) \ar[dl, "p"] \\
      & (X, x_0)
    \end{tikzcd}
  \end{center}
  so $\phi_1, \phi_2$ are deck transformations
  taking $\widetilde{x}_0$ to $\widetilde{x}_1$
  or the reverse.

  $(2)$ From above, if $[\gamma] H [\gamma]^{-1} = H$,
  there exists $\phi \in G(\widetilde{X})$ such that
  $\phi(\widetilde{x}_0) = \widetilde{\gamma}(1)$,
  where $\widetilde{\gamma}$ is the lift of $\gamma$
  based at $\widetilde{x}_0$. So we get a map
  $\Phi : N(H) \to G(\widetilde{X})$. First,
  we claim that $\Phi$ is a homomorphism.
  Suppose $\phi_i = \Phi([\gamma_i])$ for
  $i = 1, 2$ and $[\gamma_i] \in N(H)$. Then
  $\phi_i(\widetilde{x}_0) = \widetilde{x}_i$, so
  $\widetilde{\phi}_i$ is a path from
  $\widetilde{x}_0$ to $\widetilde{x}_i$. Note
  $\phi_1(\widetilde{x}_2)$ is a lift
  of $\gamma$ based at $\widetilde{x}_1$, and
  $\widetilde{\gamma}_1 * \phi_{1}(\widetilde{\gamma}_2)$
  is a path from $\widetilde{x}_0$ to $\phi_1(\widetilde{x}_2)$.
  Then
  \[
    [p \circ (\widetilde{\gamma}_1 * (\phi_1 \circ \widetilde{\gamma}_2))]
    = [\gamma_1 * \gamma_2]
    = [\gamma_1] \cdot [\gamma_2],
  \]
  so $\phi_1 \circ \phi_2 = \Phi([\gamma_1] \cdot [\gamma_2])$, i.e.
  $\Phi$ is a homomorphism.
  It only remains to show (as an exercise)
  that $\Phi$ is surjective and that $\ker \Phi = H$,
  so $G(\widetilde{X}) \cong N(H) / H$
  by the first isomorphism theorem.
\end{proof}

\begin{remark}
  If $H$ is normal in $\pi_1(X, x_0)$, then
  $G(\widetilde{X}) = \pi_1(X, x_0) / H$.
  In particular, if $\widetilde{X}$ is the universal
  cover (so $\pi_1(\widetilde{X})$ is trivial), then
  $G(\widetilde{X}) = \pi_1(X, x_0)$.
\end{remark}
