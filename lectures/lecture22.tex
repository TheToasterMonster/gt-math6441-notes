\chapter{Apr.~2 --- Cohomology, Part 2}

\section{Cohomology of Spaces}

\begin{definition}
  Let $X$ be a topological space and
  $(C_n(X), \partial)$ be the singular chain groups
  of $X$. The cohomology of this complex, denoted
  $H^n(X; G)$, is
  called the \emph{cohomology of $X$ (with coefficients in $G$)}.
  Similarly for a pair $(X, A)$, we get $H^n(X, A; G)$
  from the chain complex $(C_n(X, A), \partial)$.
\end{definition}

\begin{remark}
  From Corollary \ref{cor:iso-cohom}, we know that
  if $X$ is a CW complex, then we get the same
  cohomology groups if we use
  $(C_*^\CW(X), \partial^\CW)$ in place of
  $(C_*(X), \partial)$.
\end{remark}

\begin{remark}
  We state the following facts that carry over from
  homology:
  \begin{enumerate}
    \item If $f : X \to Y$ is a map, then
      we get a chain map $f_* : C_n(X) \to C_n(Y)$,
      and thus a homomorphism
      \[
        f^* : H^n(Y; G) \to H^n(X; G).
      \]
    \item If $f, g : X \to Y$ are homotopic, then
      $f_*, g_*$ are chain homotopic, i.e. there
      exists a map $p_n : C_n(X) \to C_{n + 1}(Y)$
      such that $\partial_{n + 1} \circ p_n + p_{n - 1} \circ \partial_n = f_n - g_n$.
      Dualizing, this becomes
      \[p^* \circ \delta + \delta \circ p^* = f^* - g^*.\]
      This implies that if $f \simeq g$, then
      $f^* = g^*$ on $H^n(Y; G)$.
  \end{enumerate}
\end{remark}

\begin{remark}
  Exactly as we did for homology, we can also prove:
  \begin{enumerate}
    \item \emph{Exact sequence of a pair:}
      We have the long exact sequence
      \begin{center}
        \begin{tikzcd}
          \cdots \ar[r] & H^n(X, A) \ar[r, "j^*"] & H^n(X) \ar[r, "i^*"] & H^n(A) \ar[r, "\delta"] & H^{n + 1}(X, A) \ar[r] & \cdots
        \end{tikzcd}
      \end{center}
      and if $f : (X, A) \to (Y, B)$, then
      the following diagram commutes:
      \begin{center}
        \begin{tikzcd}
          H^n(A) \ar[r, "\delta"] & H^{n + 1}(X, A) \\
          H^n(B) \ar[u, "f^*"] \ar[r, "\delta"] & H^{N + 1}(Y, B) \ar[u, "f^*"]
        \end{tikzcd}
      \end{center}
    \item \emph{Excision}: If
      $Z \subseteq \overline{Z} \subseteq \Int A \subseteq A \subseteq X$, then
      the inclusion map
      \[
        (X - Z, A - Z) \longrightarrow (X, A)
      \]
      induces an isomorphism
      \[
        H^n(X, A) \longrightarrow H^n(X - Z, A - Z).
      \]
    \item For a point, we have
      \[
        H^n(\{\mathrm{pt}\}; G)
        \cong
        \begin{cases}
          G & \text{if } n = 0, \\
          0 & \text{otherwise}.
        \end{cases}
      \]
    \item \emph{Mayer-Vietoris}: If
      $X = A \cup B$ with $A, B$ open, then
      \begin{center}
        \begin{tikzcd}
          \cdots \ar[r] & H^n(X) \ar[r] & H^n(A) \oplus H^n(B) \ar[r] & H^n(A \cap B) \ar[r] & H^{n + 1}(X) \ar[r] & \cdots
        \end{tikzcd}
      \end{center}
  \end{enumerate}
\end{remark}

\begin{exercise}
  Show directly (i.e. without appealing to the
  universal coefficient theorem) that
  \[
    H^k(D^n; G) =
    \begin{cases}
      G & \text{if } k = 0, \\
      0 & \text{otherwise}
    \end{cases}
    \quad \text{and} \quad
    H^k(S^n; G)
    \cong H^k(D^n, \partial D^n; G)
    \cong
    \begin{cases}
      G & \text{if } k = 0, n, \\
      0 & \text{otherwise}.
    \end{cases}
  \]
\end{exercise}

\section{Products}

\begin{remark}
  We will define two products on cohomology:
  The first is the \emph{cross product}
  \begin{align*}
    H^p(X) \times H^q(Y)
    &\longrightarrow H^{p + q}(X \times Y) \\
    (\alpha, \beta)
    &\longmapsto \alpha \times \beta,
  \end{align*}
  which satisfies the following properties:
  It is \emph{bilinear}:
  \begin{align*}
    (\alpha_1 + \alpha_2) \times \beta
    = (\alpha_1 \times \beta) + (\alpha_2 \times \beta), \\
    \alpha \times (\beta_1 + \beta_2)
    = (\alpha \times \beta_1) + (\alpha \times \beta_2),
  \end{align*}
  and \emph{natural}: If $f : X' \to X$ and $g : Y' \to Y$,
  then
  \[
    (f^* \alpha) \times (g^* \beta)
    = (f \times g)^*(\alpha \times \beta).
  \]
  The second is the \emph{cup product}:
  \begin{align*}
    H^p(X) \times H^q(X)
    &\longrightarrow H^{p + q}(X) \\
    (\alpha, \beta)
    &\longmapsto \alpha \cup \beta,
  \end{align*}
  which is also \emph{bilinear}:
  \begin{align*}
    (\alpha_1 + \alpha_2) \cup \beta
    = (\alpha_1 \cup \beta) + (\alpha_2 \cup \beta), \\
    \alpha \cup (\beta_1 + \beta_2)
    = (\alpha \cup \beta_1) + (\alpha \cup \beta_2),
  \end{align*}
  and \emph{natural}: If $f : X' \to X$, then
  \[
    f^*(\alpha \cup \beta)
    = (f^* \alpha) \cup (f^* \beta).
  \]
\end{remark}

\begin{remark}
  We will see that the cup product is more
  useful, but the cross product is usually easier
  to compute. However, they are \emph{logically equivalent}.
  To see this, let
  $p_1 : X \times Y \to X$ and $p_2 : X \times Y \to Y$
  be the projection maps and
  $\Delta : X \to X \times X : p \mapsto (p, p)$
  be the diagonal map. Suppose we have a
  cup product $\cup$, then we can define
  a cross product via
  \begin{align*}
    \times_{\cup} : H^p(X) \times H^q(Y)
    &\longrightarrow H^{p + q}(X \times Y) \\
    (\alpha, \beta)
    &\longmapsto (p_1^* \alpha) \cup (p_2^* \beta).
  \end{align*}
  Check as an exercise that
  $x_{\times}$ is bilinear and natural.
  Conversely, given a cross product $\times$,
  we can define
  \begin{align*}
    \cup_\times : H^p(X) \times H^q(X)
    &\longrightarrow H^{p + q}(X) \\
    (\alpha, \beta)
    &\longmapsto \Delta^*(\alpha \times \beta),
  \end{align*}
  which one can check is a cup product.
  Note that $\Delta^* : H^*(X \times X) \to H^*(X)$
  is the reason we need to work with
  cohomology for this construction.
\end{remark}

\begin{remark}
  Note that given $\cup$, we have
  $\cup_{X_{\cup}} = \cup$. Indeed, we have
  \begin{align*}
    \alpha \cup_{\times_{\cup}} \beta
    = \Delta^*(\alpha \times_{\cup} \beta)
    &= \Delta^*(p_1^*(\alpha) \cup p_2^*(\beta))
    = (\Delta^* \circ p_1^*)(\alpha) \cup (\Delta^* \circ p_2^*)(\beta) \\
    &= (p_1 \circ \Delta)^*(\alpha) \cup (p_2 \circ \Delta)^*(\beta)
    = \alpha \cup \beta.
  \end{align*}
  Similarly, one can verify that
  given $\times$, we have $\times_{\cup_{\times}} = \times$.
\end{remark}

\section{Tensor Products}
\begin{definition}
  Let $G, H$ be two abelian groups, and let
  $F(G \times H)$ be the free abelian group
  generated by $G \times H$, i.e.
  finite formal sums $\sum_i n_i (g_i, h_i)$
  with $n_i, \Z$, $g_i \in G$, and $h_i \in H$.
  Let
  \begin{align*}
  S = \text{subgroup of $F(G \times H)$ generated by }
  &(g + g', h) - (g, h) - (g', h), \\
  &(g, h + h') - (g, h) - (g, h'), \\
  &(ng, h) - n(g, h), \\
  &(g, nh) - n(g, h), \\
  &\hspace{-3em} \text{for all $g, g' \in G$, $h, h', \in H$, and $n \in \Z$}.
  \end{align*}
  The \emph{tensor product} of $G$ and $H$ is
  the group $G \otimes H = F(G \times H) / S$.
  Denote the coset of $(g, h)$ by $g \otimes h$,
  so that elements of $G \otimes H$
  are of the form $\sum_{i = 1}^k n_i (g_i \otimes h_i)$.
\end{definition}

\begin{remark}
  In $G \otimes H$, we have the following
  properties:
  \begin{align*}
    (g + g') \otimes h &= g \otimes h + g' \otimes h, \\
    g \otimes (h + h') &= g \otimes h + g \otimes h', \\
    (ng) \otimes h &= n(g \otimes h) = g \otimes (nh).
  \end{align*}
\end{remark}

\begin{exercise}
  Verify the following properties of
  the tensor product:
  \begin{enumerate}
    \item $G \otimes H \cong H \otimes G$.
    \item $(\bigoplus_i G_i) \otimes H \cong \bigoplus_i (G_i \otimes H)$.
    \item $(G \otimes H) \otimes K \cong G \otimes (H \otimes K)$.
    \item $\Z \otimes G \cong G$.
    \item $\Z / n \otimes G \cong G / nG$.
    \item Given homomorphisms
      $f : G \to G'$ and $g : H \to H'$, the map
      \begin{align*}
        f \otimes g : G \otimes H
        &\longrightarrow G' \otimes H' \\
        x \otimes y
        &\longmapsto f(x) \otimes g(y)
      \end{align*}
      is a well-defined homomorphism.
    \item A bilinear map $\phi : G \times H \to K$
      induces a homomorphism
      $G \otimes H \to K$ by $g \otimes h \mapsto \phi(g, h)$.
  \end{enumerate}
\end{exercise}

\begin{remark}
  Part $(7)$ of the previous exercise is
  part of the reason why we define the tensor
  product: It turns a bilinear map into a linear
  map, which is easier to work with.
\end{remark}

\begin{remark}
  We can define tensor products more generally
  for $R$-modules, where $R$ is a
  commutative ring with unit. We will not
  need this, however.
\end{remark}

\section{Cross and Cup Products}
\begin{definition}
  Define the \emph{tensor product} of
  chain complexes $(C_*, \partial)$ and
  $(C_*', \delta)$ to be the chain complex
  $C \otimes C'$ with
  \[
    (C \otimes C')_n = \bigoplus_{i + j = n} (C_i \otimes C_j)
  \]
  and boundary map $\partial^\otimes : C \otimes C' \to C \otimes C'$
  given by
  \[
    \partial^\otimes(a \otimes b)
    = (\partial a) \otimes b + (-1)^i a \otimes (\partial' b).
  \]
\end{definition}

\begin{exercise}
  Verify that $(\partial^\otimes)^2 = 0$.
\end{exercise}

\begin{definition}
  Define the \emph{algebraic cross product} by
  \begin{align*}
    \times_{\mathrm{alg}} : H_p(C) \times H_q(C')
    &\longrightarrow H_{p + q}(C \otimes C') \\
    [z] \otimes [w] &\longmapsto [z \otimes w].
  \end{align*}
\end{definition}

\begin{remark}
  This is well-defined, e.g. if
  $z = \overline{z} + \partial \tau$ (i.e.
  $z, \overline{z} \in [z]$), then
  \[
    z \oplus w = (\overline{z} + \partial \tau) \otimes w
    = \overline{z} \otimes w + \partial(\tau \otimes w)
    = \overline{z} \otimes w \otimes \partial^\otimes(\tau \otimes w)
  \]
  since $\partial' w = 0$, so
  $[z \otimes w] = [\overline{z} \otimes w]$.
  One can also check $z \otimes w$ is
  independent of the choice of
  $w \in [w]$.
\end{remark}

\begin{exercise}
  Show that $\times_{\mathrm{alg}}$ is natural
  with respect to chain maps.
\end{exercise}

\begin{theorem}[$1 / 2$ K\"unneth Sequence]
  The following sequence is exact:
  \begin{center}
    \begin{tikzcd}
      0 \ar[r] & \bigoplus_{p + q = n}(H_p(C) \otimes H_q(C')) \ar[r, "\times_{\mathrm{alg}}"] & H_n(C \otimes C')
    \end{tikzcd}
  \end{center}
\end{theorem}

\begin{proof}
  This is purely a fact of algebra (we will
  also not need it), see Hatcher.
\end{proof}

\begin{remark}
  If $X, Y$ are CW complexes, we get a CW structure
  on $X \times Y$ by taking products of cells: If
  $e^i_j$ is an $i$-cell of $X$ and
  $\widehat{e}_{j'}^{i'}$ is an $i'$-cell of $Y$,
  then $e^i_j \times \widehat{e}^{i'}_{j'}$
  is an $(i + i')$-cell of $X \times Y$.
  \[
    \alpha^i_j : \partial e^i_j \to X^{i - 1}
    \quad \text{and} \quad
    \widehat{\alpha}^{i'}_{j'} : \partial \widehat{e}^{i'}_{j'} \to Y^{i' - 1}
  \]
  are the attaching maps for
  $e^i_j$ and $\widehat{e}^{i'}_{j'}$,
  then the attaching map of $e^i_j \times \widehat{e}^{i'}_{j'}$
  is
  \begin{align*}
    \partial(e^i_j \times \widehat{e}^{i'}_{j'})
    = [\partial e^i_j \times \widehat{e}^{i'}_{j'}] \cup [e^i_j \times \partial \widehat{e}^{i'}_{j'}]
    &\longrightarrow
    (X^{i - 1} \times Y^i) \cup (X^i \times Y^{i - 1})
    \subseteq (X \times Y)^{i + i' - 1} \\
    [\partial e^i_j \times \widehat{e}^{i'}_{j'}] \ni (x, y)
    &\longmapsto (\alpha^i_j(x), y) \\
    [e^i_j \times \partial \widehat{e}^{i'}_{j'}] \ni (x, y)
    &\longmapsto (x, \widehat{\alpha}^{i'}_{j'}(y)).
  \end{align*}
\end{remark}

\begin{remark}
  For $a = \sum \alpha^k e^i_k \in C^\CW_i(X)$ and
  $b = \sum \beta^\ell \widehat{e}^{i'}_{\ell} \in C^\CW_{i'}(Y)$ (with $\alpha^k, \beta^\ell \in \Z$),
  define
  \[
    a \otimes b = \sum \alpha^k \beta^\ell (e^i_k \otimes \widehat{e}^{i'}_{\ell}).
  \]
  From above, the boundary map
  is given by
  \begin{align*}
    \partial^\CW(a \times b)
    &= \sum \alpha^k \beta^\ell (\partial^\CW e^i_k \times \widehat{e}^{i'}_{\ell} + (-1)^i e^i_k \times \partial^\CW \widehat{e}^{i'}_{\ell}) \\
    &= \partial^\CW a \times b + (-1)^i a \times \partial^\CW b.
  \end{align*}
  Thus we get chain maps
  \begin{align*}
    \bigoplus_{p + q = n} (C^\CW_p(X) \otimes C^\CW_q(Y))
    &\overset{B}{\longrightarrow} C^\CW_n(X \times Y) \\
    C^\CW(X \times Y)
    &\overset{A}{\longrightarrow} \bigoplus_{p + q = n} (C^\CW_p(X) \otimes C^\CW_q(Y)),
  \end{align*}
  where $A$ is given by
  \[
    A\Bigg(\sum_{p_i + q_j = n} a^{ij} (e_i^{p_i} \times e_j^{q_j})\Bigg)
    = \sum_{p_i + q_j = n} a^{ij} (e_i^{p_i} \otimes e_j^{q_j}).
  \]
  One can easily check $A \circ B(a \otimes b) = a \otimes b$ and
  $B \circ A(a) = a$, so we get an isomorphism
  on homology:
  \[
    H_n^\CW(X \times Y) \overset{\cong}{\longrightarrow}
    H_n(C^\CW(X) \otimes C^\CW(Y))
  \]
\end{remark}

\begin{remark}
  In singular homology, do the following:
  \begin{align*}
    B : C_p(X) \otimes C_q(Y)
    &\longrightarrow C_{p + q}(X \times Y) \\
    (\sigma : \Delta^p \to X, \tau : \Delta^q \to Y)
    &\longmapsto (\sigma \times \tau : \Delta^{p} \times \Delta^q \to X \times Y),
  \end{align*}
  where we can break $\Delta^p \times \Delta^q$ into
  several $(p + q)$-simplices.
  Now given $\sigma : \Delta^n \to X$, set
  \begin{align*}
    {}_p\sigma : \Delta^p \to X : (t_0, \dots, t_p)
    &\longmapsto \sigma(t_0, \dots, t_p, 0, \dots, 0) \\
    \sigma_q : \Delta^q \to X : (t_0, \dots, t_q)
    &\longmapsto \sigma(0, \dots, 0, t_0, \dots, t_q).
  \end{align*}
  We can then define
  \begin{align*}
    A : C_n(X \times Y)
    &\longrightarrow \bigoplus_{p + q = n} (C_p(X) \otimes C_q(Y)) \\
    \sigma
    &\longmapsto \sum_{p + q = n} {}_p(p_X \circ \sigma) \otimes (p_Y \circ \sigma)_q,
  \end{align*}
  where $p_X : X \times Y \to X$ and $p_Y : X \times Y \to Y$ are the projection maps.
\end{remark}

\begin{theorem}[Eilenberg-Zilber]
  The maps $A, B$ induce isomorphisms on
  singular homology.
\end{theorem}

\begin{definition}
  The \emph{homological cross product}
  $\times$ is given by the composition
  \begin{center}
    \begin{tikzcd}
      H_p(X) \otimes H_q(Y) \ar[r, "\times_{\mathrm{alg}}"] & H_{p + q}(C(X) \otimes C(Y))\ar[r, "B"] & H_{p + q}(X \times Y)
    \end{tikzcd}
  \end{center}
\end{definition}
