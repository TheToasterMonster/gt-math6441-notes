\chapter{Jan.~29 --- Some Group Theory}

\section{Group Presentations}

\begin{definition}
  The \emph{free group} on $n$ generators, denoted
  $F_n$, is defined inductively via
  \[
    F_n = F_{n - 1} * \Z,
  \]
  where $F_1 = \Z$. (One can also consider
  $F_\infty$.)
\end{definition}

\begin{remark}
  Note that any homomorphism $\phi : \Z \to G$
  (for any group $G$) is determined by $\phi(1)$.
  Moreover, given any $g \in G$, there is a unique
  homomorphism which maps $1 \mapsto g$. Thus
  by the universal property of the free product,
  a homomorphism $F_n \to G$ is determined
  uniquely by a choice of
  $g_1, \dots, g_n \in G$.
\end{remark}

\begin{definition}
  A \emph{group presentation} is a group
  $\langle X | R \rangle$ defined as follows:
  \begin{itemize}
    \item $X$ is some set (of generators);
    \item $R$ is a set of
      words (relations) in $X \cup X^{-1}$ (formally
      denote $x \in X$ as $x^{-1} \in X^{-1}$);
    \item let $n = |X|$ and $F_n$ be the free group
      on $n$ generators, so that we can think of
      $R \subseteq F_n$;
    \item let $\langle R \rangle$ be the smallest
      normal subgroup of $F_n$ containing $R$;
    \item define the group
      $\langle X | R \rangle = F_n / \langle R \rangle$.
  \end{itemize}
  We say that $\langle X | R \rangle$ is a \emph{presentation} of
  a group $G$ if $G \cong \langle X | R \rangle$.
\end{definition}

\begin{example}
  The group $\langle g | g^n \rangle$ is all the words
  in $g, g^{-1}$:
  \[
    \dots, g^{-2}, g^{-1}, e, g, g^2, g^3, \dots,
  \]
  but $g^n = e$, so $g^{n + 1} = g^n g = eg = g$
  and thus we have $g^{-1} = eg^{-1} = g^n g^{-1} = g^{n - 1} g g^{-1} = g^{n - 1}$.
  So there is a one-to-one correspondence between
  elements of $\langle g | g^n \rangle$ and
  $g^k$ for $k = 0, \dots, n - 1$.
\end{example}

\begin{exercise}
  Show that $\langle g | g^n \rangle \cong \Z / n$, so that $\langle g | g^n \rangle$ is a
  presentation of $\Z / n$.
\end{exercise}

\begin{lemma}
  Every group has a presentation.
\end{lemma}

\begin{proof}
  Let $G$ be a group. Let $X \subseteq G$ be a
  collection of elements of $G$ that generate $G$
  (e.g. take $X = G$ itself). Let $n = |X|$, so
  there exists a unique $\phi : F_n \to G$
  sending the generators of $F_n$ to the $g_i \in X$.
  Let $N = \ker \varphi$, so the first isomorphism
  theorem says that $G \cong F_n / N$ (note that
  $\phi$ is clearly surjective). Let
  $R$ be a subset of $N$ that generates $N$
  (e.g. take $R = N$). Then
  $G \cong \langle X | R \rangle$.
\end{proof}

\begin{definition}
  We say that $G$ is \emph{finitely generated}
  if $G \cong \langle X | R \rangle$ such that
  $|X| < \infty$. We say that $G$ is
  \emph{finitely presented} if both
  $|X|, |R| < \infty$.
\end{definition}

\begin{exercise}
   Show the following:
   \begin{enumerate}
     \item If $G = \langle g_1, \dots, g_n | r_1, \dots, r_m \rangle$, then for any group $H$ and
       any map $h : \{g_1, \dots, g_n\} \to H$
       satisfying $h(r_i) = e_H$ (the notation
       $h(r_i)$ means to replace
       any letters $g_j$ in $r_i$ by $h(g_j)$),
       there exists a unique homomorphism
       $\phi_h : G \to H$ such that
       $\phi_h(g_i) = h(g_i)$.
     \item If $G_1 = \langle g_1, \dots, g_n | r_1, \dots, r_m \rangle$ and
       $G_2 = \langle h_1, \dots, h_k | s_1, \dots, s_\ell \rangle$, then
       \[
         G_1 * G_2 = \langle g_1, \dots, g_n, h_1, \dots, h_k | r_1, \dots, r_m, s_1, \dots, s_\ell \rangle.
       \]
   \end{enumerate}
\end{exercise}

\begin{definition}
  Given groups $G_1, G_2$ and $K$, and homomorphisms
  $\psi_i : K \to G_i$, the \emph{free product with amalgamation}
  is
  \[
  G_1 *_K G_2 = \frac{G_1 * G_2}{\langle \{\psi_1(k) \psi_2(k)^{-1}\}_{k \in K} \rangle},
  \]
  where $\langle \{\psi_1(k) \psi_2(k)^{-1}\}_{k \in K} \rangle$ is the
  smallest normal subgroup of $G_1 * G_2$ containing
  $\{\psi_1(k) \psi_2(k)^{-1}\}_{k \in K}$.
\end{definition}

\begin{remark}
  The idea is that $G_1 *_K G_2$ is the set of all
  words in $G_1 \cup G_2$ but if we see
  $\psi_1(k)$ in a word, we can replace it
  by $\psi_2(k)$ and vice versa. In terms of
  group presentations, if
  \begin{align*}
    G_1 &= \langle g_1, \dots, g_n | r_1, \dots, r_m \rangle, \\
    G_2 &= \langle g_1', \dots, g_{n'}' | r_{1}', \dots, r_{m'}' \rangle, \\
    K &= \langle h_1, \dots, h_k | s_1, \dots, s_\ell \rangle,
  \end{align*}
  then we can write
  \[
    G_1 *_K G_2 = \langle g_1, \dots, g_n, g_1', \dots, g_{n'}' | r_1, \dots, r_m, r_{1}', \dots, r_{m'}', \psi_1(h_1) (\psi_2(h_1))^{-1}, \dots, \psi_1(h_k)(\psi_2(h_k))^{-1} \rangle.
  \]
  Note that the $s_i$ show up in the above since
  $\psi_1, \psi_2$ are homomorphisms.
\end{remark}

\begin{exercise}
  Show the following:
  \begin{enumerate}
    \item Check that the above presentation for
      $G_1 *_K G_2$ is correct.
    \item Let $\iota_i : G_i \to G_1 * G_2$ be the inclusions
      and $j_i : G_i \to G_1 *_K G_2$
      be the induced maps. Then given any homomorphisms
      $\phi_i : G_i \to H$ (where $H$ is any group)
      such that
      \[
        \phi_1 \circ \psi_1(k) = \phi_2 \circ \psi_2(k)
        \quad \text{ for all } k \in K,
      \]
      then there exists a unique homomorphism
      $\Phi : G_1 *_K G_2 \to H$ such that
      $\Phi \circ j_i = \phi_i$, i.e.
      \begin{center}
        \begin{tikzcd}
          & G_1 \ar[dr, "j_1"] \ar[drr, "\phi_1", bend left=30] \\
          K \ar[ur, "\psi_1"], \ar[dr, "\psi_2", swap] & & G_1 *_K G_2 \ar[r, "\Phi", dashed] & H \\
                                                 & G_2 \ar[ur, "j_2", swap] \ar[urr, "\phi_2", bend right=30]
        \end{tikzcd}
      \end{center}
      Show that this
      is the \emph{universal property} for the
      free product with amalgamation.
  \end{enumerate}
\end{exercise}

\section{Seifert-van Kampen Theorem}
\begin{theorem}[Seifert-van Kampen]
  Let $X$ be a topological space with base point $x_0$.
  Let $A, B \subseteq X$ be open sets with
  $X = A \cup B$ such that $A, B, A \cap B$ are
  path-connected and $x_0 \in A \cap B$. Let
  \[
    \psi_A : \pi_1(A, x_0) \to \pi_1(X, x_0)
    \quad \text{ and } \quad
    \psi_B : \pi_1(B, x_0) \to \pi_1(X, x_0)
  \]
  be the homomorphisms induced from the inclusions
  $A \cap B \to A$ and $A \cap B \to B$. Then
  \[
    \pi_1(X, x_0) \cong \pi_1(A, x_0) *_{\pi_1(A \cap B, x_0)} \pi_1(B, x_0).
  \]
\end{theorem}

\begin{remark}
  There is a more general version where
  $X = \bigcup_{\alpha \in A} U_\alpha$. See Hatcher
  for more details.
\end{remark}

\begin{example}
  Consider $W_2 = S^1 \lor S^1$, a wedge of two circles.
  Let $x_0$ be the point of intersection of the circles.
  Let $A$ be an open neighborhood of the left circle
  and $B$ be an open neighborhood of the right circle.
  Note that $A \simeq B \simeq S^1$ and
  $A \cap B \simeq \{\text{pt}\}$. So we see that
  \[
    \pi_1(A, x_0) \cong \Z \cong
    \langle g_1 |\ \rangle, \quad
    \pi_1(B, x_0) \cong \Z \cong \langle g_2 |\ \rangle, \quad
    \pi_1(A \cap B, x_0) = \{e\},
  \]
  and so
  $\psi_A : \pi_1(A \cap B, x_0) \to \pi_1(A, x_0)$ and
  $\psi_B : \pi_1(A \cap B, x_0) \to \pi_1(B, x_0)$
  must both map $e \mapsto e$. Thus
  \[
    \pi_1(W_2, x_0) \cong \pi_1(A, x_0) *_{\pi_1(A \cap B, x_0)} \pi_1(B, x_0)
    \cong \langle g_1, g_2 | \psi_A(e)(\psi_B(e))^{-1} \rangle
    \cong \langle g_1, g_2 | \ \rangle
    \cong F_2,
  \]
  by the Seifert-van Kampen theorem.
\end{example}

\begin{exercise}
  Show the following:
  \begin{enumerate}
    \item If $W_n$ is a wedge of $n$ circles,
      then $\pi_1(W_n, x_0) \cong F_n$.
    \item We have
      $\pi_1(\text{any connected graph}) = F_n$
      for some $n$.
  \end{enumerate}
\end{exercise}

\begin{example}
  Consider the torus $T^2 = S^1 \times S^1$. Recall
  that for a product, we know
  \[
    \pi_1(T^2) \cong \pi_1(S^1) \times \pi_1(S^1) \cong \Z \times \Z.
  \]
  We can also use van Kampen's theorem to see this.
  Think of $T^2$ as a square with opposite sides
  identified. Let $A$ be the square with a circle
  missing in the circle, which deformation retracts
  to the boundary of the square. Let $B$ be a big disk
  in the middle, so that $A \cap B$ deformation
  retracts to a circle. So
  \[
    A \simeq W_2, \quad B \simeq \{\text{pt}\}, \quad
    A \cap B \simeq S^1.
  \]
  Thus by our previous computations, we have
  \[
    \pi_1(A, x_0) \cong \langle g_1, g_2 | \ \rangle, \quad
    \pi_1(B, x_0) \cong \{e\}, \quad
    \pi_1(A \cap B, x_0) \cong \langle h | \ \rangle.
  \]
  The inclusion $\psi_B : \pi_1(A \cap B, x_0) \to \pi_1(B, x_0)$
  sends $h \to e$, and $\psi_A : \pi_1(A \cap B, x_0) \to \pi_1(A, x_0)$
  sends $h \mapsto g_1 g_2 g_1^{-1} g_2^{-1}$. To see the
  last claim, push a loop in $A \cap B$ to the boundary
  of the square (note that under the homotopy equivalence
  of $A$ and $W_2$, we can assume $x_0$ maps to a corner
  point), where it follows the four edges of the
  square. These are the loops
  $g_1, g_2, g_1^{-1}, g_2^{-1}$, where the last two
  loops are oriented in the opposite direction of
  the first two. Thus by van Kampen's theorem,
  \[
    \pi_1(T^2, x_0) \cong \pi_1(A, x_0) *_{\pi_1(A \cap B, x_0)} \pi_1(B, x_0)
    \cong \langle g_1, g_2 | \psi_A(h)(\psi_B(h))^{-1} \rangle
    = \langle g_1, g_2 | g_1 g_2 g_1^{-1} g_2^{-1} \rangle.
  \]
\end{example}

\begin{exercise}
  Check the following:
  \begin{enumerate}
    \item Show that $\Z \times \Z \cong \langle g_1, g_2 | g_1 g_2 g_1^{-1} g_2^{-1} \rangle$, so that
      $\pi_1(T^2) \cong \Z \times \Z$ again.
    \item Compute $\pi_1(\Sigma_g, x_0)$, where
      $\Sigma_g$ is the surface of genus $g$.
      Show that
      $\pi_1(\Sigma_g)$ is not abelian if $g > 1$.
  \end{enumerate}
\end{exercise}
