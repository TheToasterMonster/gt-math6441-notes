\chapter{Apr.~16 --- Fundamental Class and Algebra}

\section{Fundamental Class of a Manifold}

\begin{definition}
  Let $M$ be a manifold of dimension $n$, and
  let $R$ be a ring with identity (usually
  $\Z$ or $\Z / 2\Z$). If $x \in M$, let
  $U$ be an open set containing $x$ and
  homeomorphic to $\R^n$, then by excision,
  \[
    H_n(M, M - \{x\}; R)
    \cong H_n(U, U - \{x\}; R)
    \cong H_n(\R^n, \R^n - \{0\}; R).
  \]
  Using the long exact sequence of the pair
  $(\R^n, \R^n - \{0\})$, we have (note that
  $\R^n - \{0\} \simeq S^{n - 1}$)
  \begin{center}
    \begin{tikzcd}[row sep=small]
      H_n(\R^n; R) \ar[r] & H_n(\R^n, \R^n - \{0\}; R)
      \ar[r, "\cong"]
      & H_{n - 1}(\R^n - \{0\}; R)
      \ar[r] & H_{n - 1}(\R^n; R) \\
      0 \ar[u, equal] & & R \ar[u, equal] & 0 \ar[u, equal]
    \end{tikzcd}
  \end{center}
  so $H_n(M, M - \{x\}; R) \cong R$ for
  all $x \in M$. Call a generator of
  $H_n(M, M - \{x\}; R)$ a \emph{local $R$-orientation}
  on $M$ at $x$, and denote it by $\mu_x$.
\end{definition}

\begin{remark}
  If $R = \Z / 2\Z$, then there is a unique
  choice for $\mu_x$, and if $R = \Z$, then
  there are two choices.
\end{remark}

\begin{exercise}
  Show that the usual definition of
  orientability (e.g. from differential topology) at a point $x \in M$ is
  equivalent to $\Z$-orientability at $x$.
\end{exercise}

\begin{definition}
  If $B$ is an open ball in a coordinate
  chart $U$, then the above argument says that
  \[
    H_n(M, M - B; R)
    \cong R.
  \]
  Moreover, the inclusion map
  $i : (M, M - B) \to (M, M - \{x\})$
  induces an isomorphism
  \begin{center}
    \begin{tikzcd}
      H_n(M, M - B; R)
      \ar[r, "i_*"]
      \ar[r, swap, "\cong"]
      & H_n(M, M - \{x\}; R)
    \end{tikzcd}
  \end{center}
  so a choice of generator for one group
  determines a generator for the other. Thus if
  $x, y \in B$, then
  \[
    H_n(M, M - \{y\}; R)
    \cong H_n(M, M - B; R)
    \cong H_n(M, M - \{x\}; R),
  \]
  where both isomorphisms are induced by
  inclusions. So a local orientation at $x$
  induces one for $y$ for all $y \in B$.
  An \emph{$R$-orientation} on $M$ is a choice
  of local $R$-orientation $\mu_x$ for
  each $x \in M$ such that for all open balls
  $B \subseteq M$ contained in a coordinate
  chart, there exists a generator $\mu_B$ for
  $H_n(M, M - B; R)$ such that
  $i_*(\mu_B) = \mu_x$ for all $x \in B$.
  If $M$ has an $R$-orientation, then we
  say that $M$ is \emph{$R$-orientable}.
  If $R = \Z$, we just say that $M$ is
  \emph{orientable}.
\end{definition}

\begin{remark}
  One can think of the $i_*(\mu_B) = \mu_x$
  condition as saying that the choices
  at each $x$ must be compatible with each
  other.
\end{remark}

\begin{lemma}
  All manifolds are $\Z / 2\Z$-orientable.
\end{lemma}

\begin{proof}
  This follows since
  $\Z / 2\Z$ has a unique generator.
\end{proof}

\begin{lemma}
  If $M$ is $R$-orientable and connected,
  then any two $R$-orientations that
  agree at a point are the same, i.e. an
  $R$-orientation is determined by a local
  $R$-orientation at any point.
\end{lemma}

\begin{proof}
  We argue using connectedness.
  Suppose $\{\mu_x\}_{x \in M}, \{\widetilde{\mu}_x\}_{x \in M}$
  are two $R$-orientations, and let
  \[
    S = \{x \in M : \mu_x = \widetilde{\mu}_x\}.
  \]
  By assumption, $S \ne \varnothing$.
  Show as an exercise that $S$ is open and
  closed, which implies $S = M$.
\end{proof}

\begin{corollary}
  If $M$ is connected and orientable,
  then $M$ has exactly two orientations.
\end{corollary}

\begin{proof}
  There are exactly two choices for a
  local $\Z$-orientation.
\end{proof}

\begin{theorem}\label{thm:orientation-map}
  Let $M$ be a closed, connected
  $n$-manifold. Then
  \begin{enumerate}
    \item If $M$ is $R$-orientable, then the
      map $i : (M, \varnothing) \to (M, M - \{x\})$
      induces an isomorphism
      \[
        i_* : H_n(M; R) \longrightarrow
        H_n(M, M - \{x\}; R) \cong R
      \]
      for every $x \in M$.
    \item If $M$ is not $R$-orientable, then
      the inclusion map is injective with
      image
      $\{r \in R : 2r = 0\}$.
    \item $H_i(M; R) = 0$ for all $i > n$.
  \end{enumerate}
\end{theorem}

\begin{definition}
  An element $[M] \in H_n(M; R)$ that maps
  to a generator of $H_n(M, M - \{x\}; R) \cong R$
  for all $x$ is called a \emph{fundamental class}
  of $M$.
\end{definition}

\begin{corollary}
  We have the following:
  \begin{enumerate}
    \item If $M$ is a closed, oriented, connected
      $n$-manifold, then
        $H_n(M; \Z) \cong \Z$
        and
        $H_n(M; \Z / 2\Z) \cong \Z / 2\Z$.
    \item If $M$ is not orientable, then
      $H_n(M; \Z) = 0$ and $H_n(M; \Z / 2\Z) \cong \Z / 2\Z$.
  \end{enumerate}
\end{corollary}

\begin{proof}
  This is clear from Theorem \ref{thm:orientation-map}
  (note that $\Z$ has no $2$-torsion).
\end{proof}

\begin{definition}
  Let
  \[
    M_R = \{\alpha_x : x \in M, \alpha_x \in H_n(M, M - \{x\}; R)\}.
  \]
  We can put a topology on $M_R$ as follows:
  For each open ball $B$ contained in a
  coordinate chart and each $\alpha \in H_n(M, M - B; R)$,
  let $U(\alpha, B) = \{i^x_*(\alpha)\}_{x \in B}$,
  where $i^x : (M, M - B) \to (M, M - \{x\})$.
\end{definition}

\begin{exercise}
  Check the following:
  \begin{enumerate}
    \item The sets $U(\alpha, B)$ form a
      basis for a topology on $M_R$.
    \item The map $\pi : M_R \to M$ given by
      $\alpha_x \mapsto x$ is a covering map.
    \item If $\sigma : M \to M_R$ is
      continuous such that $\pi \circ \sigma = \id_M$
      (such a map $\sigma$ is called a
      \emph{section} of $M_R \overset{\pi}{\to} M$)
      and $\sigma(x)$ is a generator of
      $H_n(M, M - \{x\}; R)$ for all
      $x$, then $\sigma$ defines an $R$-orientation.
  \end{enumerate}
\end{exercise}

\begin{lemma}\label{lem:section}
  Let $M$ be an $n$-manifold and $A \subseteq M$
  a compact subset. Then
  \begin{enumerate}
    \item If $\sigma : M \to M_R$ is a
      section of $M_R$, then there exists a
      unique class $\alpha_A \in H_n(M, M - A; R)$
      whose image in $H_n(M, M - \{x\}; R)$ is $\sigma(x)$
      for all $x \in A$.
    \item $H_n(M, M - A; R) = 0$ for all
      $i > n$.
  \end{enumerate}
\end{lemma}

\begin{proof}
  The lemma is proved through the following
  sequence of claims:
  \begin{enumerate}
    \item If the lemma is true for $A, B, A \cap B$, then
      it is true for $A \cup B$.
    \item If the lemma is true for $M = \R^n$,
      then it is true for all $M$.
    \item The lemma is true for $M = \R^n$.
  \end{enumerate}
  The first two claims are exercises with
  Mayer-Vietoris, but the third is
  difficult.
  See the \href{https://etnyre.math.gatech.edu/class/6441Spring21/SectionIV.pdf}{course notes}.
\end{proof}

\begin{proof}[Proof of Theorem \ref{thm:orientation-map}]
  This follows from Lemma \ref{lem:section},
  see the \href{https://etnyre.math.gatech.edu/class/6441Spring21/SectionIV.pdf}{course notes}.
\end{proof}

\section{Algebraic Limits}

\begin{definition}
  A set $I$ is a \emph{directed set}
  if there exists a partial order
  $i \le i'$ defined on certain pairs in
  $I$ such that for all $i, i' \in I$, there
  exists $i'' \in I$ such that
  $i \le i''$ and $i' \le i''$.
\end{definition}

\begin{example}
  The following are examples of directed sets:
  \begin{enumerate}
    \item $I = \Z$ with the usual order $\le$.
    \item $I = \text{all subsets of a set $X$}$,
      where $\le$ is given by inclusion.
  \end{enumerate}
\end{example}

\begin{definition}
  Suppose $\{M_i\}_{i \in I}$ is a family of
  $R$-modules indexed by a directed set $I$,
  such that:
  \begin{enumerate}
    \item For all $i \le i'$, there is a
      homomorphism $\phi_{i' i} : M_i \to M_{i'}$.
    \item If $i \le i' \le i''$, then
      $\phi_{i'' i} = \phi_{i'' i'} \circ \phi_{i' i}$.
    \item $\phi_{ii} = \id_{M_i}$.
  \end{enumerate}
  Such a family is called a \emph{directed system of modules}.
  The \emph{direct limit} of a system
  $\{M_i\}_{i \in I}$
  is a module
  \[
    M = \varinjlim M_i
  \]
  together with
  homomorphisms $\phi_i : M_i \to M$
  such that
  $\phi_{i'} \circ \phi_{i' i} = \phi_i$
  and for any module $N$ with maps $\psi_i : M_i \to N$
  satisfying $\psi_{i'} \circ \phi_{i' i} = \psi_i$,
  there is a unique homomorphism
  $\psi : M \to N$ with
  $\psi_i = \psi \circ \phi_i$.
\end{definition}

\begin{exercise}
  Show that direct limits are unique up to
  isomorphism.
\end{exercise}

\begin{lemma}
  Direct limits exist.
\end{lemma}

\begin{proof}
  Let $M^+ = \bigoplus M_i$ and
  $\phi_i^+ : M_i \to M^+$ send
  $x$ to the $I$-tuple with $x$ in the
  $i$th component and $0$ elsewhere. Let
  $J$ be the submodule of $M^+$ generated by
  \[\{\phi_{i'}^+ \circ \phi_{i' i}(x) - \phi_i^+(x)\}_{x \in M}\]
  for all $i, i' \in I$. Set
  $M = M^+ / J$ and
  $\phi_i = \pi \circ \phi_i^+$, where
  $\pi : M^+ \to M$ is the quotient map.
  Check as an exercise that $(M, \phi_i)$ is
  indeed the direct limit.
\end{proof}

\begin{exercise}
  Show the following:
  \begin{enumerate}
    \item If the $M_i$ are all submodules of $M$
      and for all $i \le i'$, the map
      $\phi_{i' i} : M_i \to M_{i'}$ is
      just the inclusion map, then
      $\varinjlim M_i = \bigcup M_i$.
    \item If there exists $m \in I$ such that
      $i \le m$ for all $i \in I$, then
      $\phi_m : M_m \to \varinjlim M_i$ is an
      isomorphism.
    \item Suppose that for all $i \in I$,
      we have $M_i = N_i \oplus P_i$ and
      $\phi_{i' i} = \psi_{i' i} \oplus \rho_{i' i}$
      for all $i \le i'$.
      Let $N = \varinjlim N_i$,
      $P = \varinjlim P_i$, and $M = \varinjlim M_i$.
      Then there are maps
      $\psi : N \to M$ and $\rho : P \to M$
      which satisfy $\psi \circ \psi_i = \phi_I|_N$,
      $\rho \circ \rho_i = \phi_i|_P$, and
      $\psi \oplus \rho : N \oplus P \to M$
      is an isomorphism.
    \item A subset $J \subseteq I$ is called
      \emph{final} if for all $i \in I$,
      there exists $j \in J$ with
      $i \le j$. Applying the definition of
      a direct limit to $\phi_j : M_j \to M$,
      we get a homomorphism
      \[
        \lambda : \varinjlim_J M_j \longrightarrow \varinjlim_I M_i.
      \]
      Show that $\lambda$ is an isomorphism.
    \item Let $\{A_i\}_{i \in I}, \{B_i\}_{i \in I}, \{C_i\}_{i \in I}$ be directed systems
      and suppose that for all $i \in I$, we have $(*)$
      \begin{center}
        \begin{tikzcd}
          A_i \ar[r, "\lambda_i"]
          & B \ar[r, "\rho_i"]
          & C_i
        \end{tikzcd}
      \end{center}
      such that for all $i \le i'$, the
      following diagram
      \begin{center}
        \begin{tikzcd}
          A_i \ar[r, "\lambda_i"]
          \ar[d, swap, "\phi^A_{i' i}"]
          & B \ar[r, "\rho_i"]
          \ar[d, "\phi^B_{i' i}"]
          & C_i
          \ar[d, "\phi^C_{i' i}"] \\
          A_{i'} \ar[r, swap, "\lambda_{i'}"]
          & B \ar[r, swap, "\rho_{i'}"]
          & C_{i'}
        \end{tikzcd}
      \end{center}
      commutes. Then we get homomorphisms ($**$)
      \begin{center}
        \begin{tikzcd}
          \varinjlim A_i \ar[r, "\lambda"]
          & \varinjlim B \ar[r, "\rho"]
          & \varinjlim C_i
        \end{tikzcd}
      \end{center}
      Show that if $(*)$ is exact, then
      $(**)$ is also exact.
  \end{enumerate}
\end{exercise}

\begin{lemma}\label{lem:direct-limit}
  Let $\{U_\alpha\}$ be a directed system
  of subsets of $X$ such that any compact
  set $K \subseteq X$ is contained in some
  $U_\alpha$. Then
  \[
    \varinjlim H_i(U_\alpha; R)
    \cong H_i(X; R).
  \]
\end{lemma}

\begin{proof}
  Clearly we have inclusion maps
  $H_i(U_\alpha; R) \to H_i(X; R)$ for all
  $\alpha$, so we get a homomorphism
  \[
    \varinjlim H_i(U_\alpha; R)
    \longrightarrow H_i(X; R).
  \]
  If $[\sigma] \in H_i(X; R)$, then
  $\im \sigma \subseteq U_{\alpha'}$ for
  some $\alpha'$, so
  $H_i(U_{\alpha'}; R) \to H_i(X; R)$
  hits $[\sigma]$. But we have
  \begin{center}
    \begin{tikzcd}
      H_i(U_{\alpha}; R) \ar[d] \ar[r] & H_i(X; R) \\
      \varinjlim H_i(U_\alpha; R) \ar[ur]
    \end{tikzcd}
  \end{center}
  so the map $\varinjlim H_i(U_\alpha; R) \to H_i(X; R)$
  is onto.
  A similar argument shows injectivity.
\end{proof}
