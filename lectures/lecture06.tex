\chapter{Jan.~27 --- Fundamental Group of the Circle}

\section{Proof of Path Lifting}

\begin{proof}[Proof of Lemma \ref{lem:lifting}]
  $(a)$ Let $A = S^1 \setminus \{(1, 0)\}$, and note that
  \[
    p^{-1}(A) = \bigcup_{i \in \Z} (i, i + 1) = \bigcup_{i \in \Z} A_i.
  \]
  Notice that each restriction
  $p|_{A_i} : A_i \to A$ is a homeomorphism.
  Now let $B = S^1 \setminus \{(-1, 0)\}$, so
  \[
    p^{-1}(B) = \bigcup_{i \in \Z} \left(i - \frac{1}{2}, i + \frac{1}{2}\right)
    = \bigcup_{i \in \Z} B_i.
  \]
  Similarly, each $p|_{B_i} : B_i \to B$ is a
  homeomorphism. Now if $\gamma : [0, 1] \to S^1$ is
  contained in $A$ (or $B$), we can choose any $i \in \Z$
  and let $\widetilde{\gamma} = (p|_{A_i})^{-1} \circ \gamma$,
  giving a lift of $\gamma$. Then for a general
  $\gamma : [0, 1] \to S^1$ with $\gamma(0) = (1, 0)$,
  the set $\{\gamma^{-1}(A), \gamma^{-1}(B)\}$ is
  an open cover of the compact metric space $[0, 1]$,
  so there exists a Lebesgue number $\delta > 0$
  such that any interval $[a, b]$ with $b - a < \delta$
  lies in either $\gamma^{-1}(A)$ or $\gamma^{-1}(B)$.
  Choose $n$ such that $1 / n < \delta$. If
  $t_n = i / n$ for $i = 0, \dots, n$, then
  \[
    \gamma([t_i, t_{i + 1}]) \subseteq A \text{ or } B
  \]
  for every $i$. Again for convenience, if
  $[t_{i - 1}, t_i]$ and $[t_i, t_{i + 1}]$ are
  both in $\gamma^{-1}(A)$ or $f^{-1}(B)$, then
  discard $t_i$. So we have a partition
  $0 = t_0 < t_1 < \dots < t_k = 1$ such that
  (note that $\gamma$ starts at $(1, 0) \notin A$)
  \[
    \gamma([t_i, t_{i + 1}]) \subseteq
    \begin{cases}
      A & \text{if } i \text{ is odd}, \\
      B & \text{if } i \text{ is even}.
    \end{cases}
  \]
  Then we want to build $\widetilde{\gamma}_n$.
  Define $\widetilde{\gamma}_n$ on $[t_0, t_1]$
  to be $(p|_{B_n})^{-1} \circ \gamma|_{[t_0, t_1]}$.
  Now $\widetilde{\gamma}_n(t_1) \in A_i$ for a
  unique $i$, so define $\widetilde{\gamma}_n$
  on $[t_1, t_2]$ by $(p|_{A_i})^{-1} \circ \gamma|_{[t_1, t_2]}$.
  Note that $\widetilde{\gamma}_n$ is continuous
  on $[t_0, t_2]$ since the two definitions agree
  at $t = t_1$.
  Inductively continue to define the lift
  $\widetilde{\gamma}_n$ on all of $[0, 1]$.

  $(b)$ The proof is very similar to path lifting.
  Given a homotopy $H : [0, 1] \times [0, 1] \to S^1$,
  we can find a Lebesgue number $\delta > 0$
  for $\{H^{-1}(A), H^{-1}(B)\}$. Pick $n$
  such that $\sqrt{2} / n < \delta$ and break
  $[0, 1] \times [0, 1]$ into $n^2$ squares
  of side length $1 / n$. The diameter of each
  square is at most $\sqrt{2} / n$, so each square
  can be lifted as above. Finish the construction
  as an exercise to lift $H$ on all of
  $[0, 1] \times [0, 1]$.
\end{proof}

\section{Applications of the Fundamental Group of \texorpdfstring{$S^1$}{S1}}

\begin{corollary}
  There is no retraction $D^2 \to \partial D^2$.
\end{corollary}

\begin{proof}
  Suppose there was a retraction $r : D^2 \to \partial D^2$,
  and let $i : S^1 \to D^2$ be the inclusion of $S^1$
  as the boundary of $D^2$. Then we have the composition:
  \begin{center}
    \begin{tikzcd}
      S^1 \ar[r, "i"] & D^2 \ar[r, "r"] & S^1
    \end{tikzcd}
  \end{center}
    Noting that $r \circ i = S^1 \to S^1$ is the
    identity, so $(r \circ i)_* : \pi_1(S^1, (1, 0)) \to \pi_1(S^1, (1, 0))$
    is the identity map. In particular,
    $r_* \circ i_* = (r \circ i)_*$ is the identity map,
    hence $i_*$ must be injective.
    But
    \[
      i_* : \pi_1(S^1, (1, 0)) \to \pi_1(D^2, (1, 0))
    \]
    where $\pi_1(S^1, (1, 0)) \cong \Z$ and
    $\pi_1(D^2, (1, 0)) = \{1\}$, so $i_*$ cannot be
    injective. Contradiction.
\end{proof}

\begin{corollary}
  Any map $f : D^2 \to D^2$ has a fixed point, i.e.
  $x \in D^2$ such that $f(x) = x$.
\end{corollary}

\begin{proof}
  Suppose otherwise that $f : D^2 \to D^2$ has no fixed
  points.
  Then for each $x \in D^2$, there is a unique ray
  $R_x$ starting at $f(x)$ and going through $x$.
  Note that $R_x \cap \partial D^2$ in a unique point
  (on the interior of $R_x$). Define
  $r : D^2 \to S^1$ by $x \mapsto R_x \cap \partial D^2$.
  Show that $r$ is continuous as an exercise (e.g.
  parametrize the line).
  But then $r$ is a retraction $D^2 \to \partial D^2$,
  a contradiction.
\end{proof}

\begin{remark}
  There are more applications such as the
  fundamental theorem of algebra, the ham sandwich
  theorem, and the Borsuk-Ulam theorem. See Hatcher
  for more details.
\end{remark}

\section{Free Products of Groups}

\begin{definition}
  Let $G_1$ and $G_2$ be groups. A \emph{word} in
  $G_1 \sqcup G_2$ is a finite sequence
  \[
    x = (x_1, x_2, \dots, x_n)
  \]
  for some $n$, where each $x_i$ is in $G_1$ or $G_2$.
  Define an equivalence relation on words in
  $G_1 \sqcup G_2$ which is generated by (show as
  an exercise that this is in fact an equivalence
  relation):
  \begin{enumerate}
    \item replace $a, b$ in a sequence by $ab$ if
      $a, b$ are in the same group (or the reverse of this), and
    \item if $e_i$ (the identity in $G_i$) is in
      a sequence, then remove it (or add it in any
      place in a sequence).
  \end{enumerate}
  Denote the equivalence class of a word $x$ by $[x]$.
  Call a word $x = (x_1, \dots, x_n)$ \emph{reduced} if
  \begin{enumerate}
    \item $x_j \ne e_i$ for any $j$ or $i$, and
    \item $x_i$ and $x_{i + 1}$ are from different
      groups.
  \end{enumerate}
  Show that each $[x]$ contains a unique reduced word
  (note: uniqueness is hard).
  The \emph{free product} of $G_1$ and $G_2$ is
  the group $G_1 * G_2$ of all equivalence classes of
  words in $G_1 \sqcup G_2$, with multiplication
  \[
    [x_1, \dots, x_n] \cdot [y_1, \dots, y_m]
    = [x_1, \dots, x_n, y_1, \dots, y_m].
  \]
\end{definition}

\begin{remark}
  Note that in $G_1 * G_2$, the identity
  $e$ is the empty word and the inverse is given by
  \[[x_1, \dots, x_n]^{-1} = [x_n^{-1}, \dots, x_1^{-1}].\]
  Check as an exercise that $G_1 * G_2$ is in
  fact a group (really only need to check associativity).
\end{remark}

\begin{prop}
  Let $j_i : G_i \to G_1 * G_2$ be the inclusion
  of $G_i$ into $G_1 * G_2$. Given any homomorphisms
  $\phi_i : G_i \to H$ where $H$ is any group, there
  exists a unique homomorphism
  $\phi : G_1 * G_2 \to H$ such that
  $\phi \circ j_i = \phi_i$, i.e.
  the following diagram commutes:
  \begin{center}
    \begin{tikzcd}
      & G_1 \ar[d, "j_1"] \ar[ddr, "\phi_1", bend left=30] \\
      G_2 \ar[r, "j_2"] \ar[drr, "\phi_2", bend right=20, swap] & G_1 * G_2 \ar[dr, dashed, "\phi"] \\
        & & H
    \end{tikzcd}
  \end{center}
\end{prop}

\begin{proof}
  If the $x_i$ are reduced and $x_1 \in G_1$, then define
  \[
    \phi_1(x_1, x_2, \dots, x_n) = \phi_1(x_1) \cdot \phi_2(x_2) .\cdot \phi_1(x_3) \cdot \dots.
  \]
  Then check the following as an exercise:
  \begin{enumerate}
    \item Show such that $\phi$ exists and is unique.
    \item Show this property \emph{defines} the
      free product, i.e. if $D$ is another group
      satisfying the property in the proposition, then
      $D \cong G_1 * G_2$.
  \end{enumerate}
  The second part above says that this is
  the \emph{universal property} of the free product.
\end{proof}

\begin{example}
  Represent $\Z$ in product notation via $\{x^n\}$, where
  $x^n x^m = x^{n + m}$. Then
  \[
    \Z * \Z = \{x^n\} * \{y^m\}
    = \{e,
      x^{n_1} y^{m_1} \dots x^{n_k},
      x^{n_1} y^{m_1} \dots y^{m_k},
      y^{m_1} x^{n_1} \dots y^{m_k},
      y^{m_1} x^{n_1} \dots x^{n_k}
    \}.
  \]
  This group is called the \emph{free group on
  two generators}, and $\Z$ is the
  \emph{free group on one generator}.
\end{example}
