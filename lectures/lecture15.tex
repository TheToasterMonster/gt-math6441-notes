\chapter{Mar.~3 --- Homological Algebra}

\section{Introduction to Homological Algebra}

\begin{definition}
  A sequence of abelian groups $C_*$ and group
  homomorphisms
  $\partial_n : C_n \to C_{n - 1}$ is called a
  \emph{chain complex} if $\partial_{n - 1} \circ \partial_n = 0$
  for all $n$. The \emph{homology} of a chain
  complex is
  \[
    H_n(C_*, \partial) = \frac{\ker \partial_n}{\im \partial_{n + 1}}.
  \]
\end{definition}

\begin{example}
  The singular chain groups and the boundary
  map $\partial$ is a chain complex.
\end{example}

\begin{definition}
  Given two chain complexes $(C_*, \partial)$ and
  $(C_*', \partial')$, a \emph{chain map} is a
  sequence of group homomorphisms $f_n : C_n \to C_n'$
  such that $\partial_n' \circ f_n = f_{n - 1} \circ \partial_{n}$, i.e. we have
  \begin{center}
    \begin{tikzcd}
      \cdots \ar[r] & C_n \ar[r, "\partial_n"] \ar[d, "f_n"] & C_{n - 1} \ar[r] \ar[d, "f_{n - 1}"] & \cdots \\
      \cdots \ar[r] & C_n' \ar[r, "\partial_n'"] & C_{n - 1}' \ar[r] & \cdots
    \end{tikzcd}
  \end{center}
\end{definition}

\begin{lemma}\label{lemma:chain-map-induced}
  A chain map $\{f_n\} : (C_*, \partial) \to (C_*', \partial')$
  induces homomorphisms
  \[(f_n)_* : H_n(C_*, \partial) \to H_n(C_*', \partial').\]
\end{lemma}

\begin{proof}
  If $[h] \in H_n(C_*, \partial) = \ker \partial_n / {\im \partial_{n + 1}}$,
  then $h \in C_n$ and $\partial_n h = 0$. Then
  \[
    \partial_n'(f_n(h)) = f_{n - 1}(\partial_n h) = f_{n - 1}(0) = 0,
  \]
  so we can define $(f_n)_*([h]) = [f_n(h)]$.
  To see that this is well-defined, suppose
  $[h] = [h'] \in H_n(C_*, \partial)$. Then
  there exists $k \in C_{n + 1}$ such that
  $h' = h + \partial_{n + 1} k$. We can write
  \[
    f_n(h')
    = f_n(h + \partial_{n + 1} k)
    = f_n(h) + f_n(\partial_{n + 1} k)
    = f_n(h) + \partial_{n + 1}'(f_{n + 1}(k)),
  \]
  i.e. $[f_n(h')] = [f_n(h)]$. Finally,
  check as an exercise that
  the $(f_n)_*$ are homomorphisms.
\end{proof}

\section{Induced Homomorphisms in Homology}
\begin{remark}
  If $f : X \to Y$ is a continuous map, then we can
  define maps
  \begin{align*}
    f_n : C_n(X) &\longrightarrow C_n(Y) \\
    \sum n_i \sigma_i &\longmapsto \sum n_i (f \circ \sigma_i).
  \end{align*}
  Note that we have
  \[
    f_{n - 1}(\partial_n \sigma)
    = f_{n - 1} \left(\sum_{i = 0}^n (-1)^i \sigma^{(i)}\right)
    = \sum_{i = 0}^n (-1)^i (f \circ \sigma^{(i)})
    = \sum_{i = 0}^n (-1)^i (f \circ \sigma)^{(i)}
    = \partial_n(f_n(\sigma)),
  \]
  In particular, we see that $\{f_n\}$ is a chain map.
  So by Lemma \ref{lemma:chain-map-induced}, we get
  a map $f_* : H_n(X) \to H_n(Y)$
  induced on singular homology for each $n$.
\end{remark}

\begin{exercise}
  Check the following:
  \begin{enumerate}
    \item $(f \circ g)_* = f_* \circ g_*$;
    \item $(\id_X)_* = \id_{H_n(X)}$.
  \end{enumerate}
\end{exercise}

\begin{definition}
  Given chain complexes
  $(C_*, \partial)$ and $(C_*', \partial')$
  and chain maps $\{f_n\}$ and $\{g_n\}$
  from $C_*$ to $C_*'$, we say there is a
  \emph{chain homotopy} between them if there is a
  sequence of homomorphisms
  \[
    p_n : C_n \to C_{n + 1}' \quad \text{such that} \quad
    \partial_{n + 1}' \circ p_n + p_{n - 1} \circ \partial_n = f_n - g_n.
  \]
  In other words, we have the diagram
  \begin{center}
    \begin{tikzcd}[sep=large]
      \cdots \ar[r] & C_{n + 1} \ar[r, "\partial_{n + 1}"] \ar[d, "g_{n + 1}"] \ar[d, swap, bend right, "f_{n + 1}"] & \ar[dl, "p_n"] C_n \ar[r, "\partial_n"] \ar[d, swap, bend right=15, "f_{n}"] \ar[d, bend left=15, "g_n"] & \ar[dl, "p_n"] C_{n - 1} \ar[d, swap, "f_{n - 1}"] \ar[d, bend left, "g_{n - 1}"] \ar[r] & \cdots \\
      \cdots \ar[r] & C_{n + 1}' \ar[r, swap, "\partial_{n + 1}'"] & C_n' \ar[r, swap, "\partial_n'"]& C_{n - 1}' \ar[r] & \cdots
    \end{tikzcd}
  \end{center}
\end{definition}

\begin{lemma}
  If there is a chain homotopy between
  $\{f_n\}$ and $\{g_n\}$, then
  \[(f_n)_* = (g_n)_* : H_n(C_*, \partial) \to H_n(C_*', \partial').\]
\end{lemma}

\begin{proof}
  Let $h \in C_n$ and $\partial_n h = 0$, so
  $h$ represents an element of $H_n(C_*, \partial)$.
  Then
  \[
    f_n(h) - g_n(h)
    = \partial_{n + 1}'(p_n(h)) + p_{n - 1}(\partial_n (h))
    = \partial_{n + 1}'(p_n(h)),
  \]
  so $[f_n(h)] = [g_n(h)]$ in $H_n(C_*', \partial')$.
\end{proof}

\begin{theorem}
  Let $f, g : X \to Y$ be homotopic maps.
  Then $\{f_n\}$ and $\{g_n\}$ are chain homotopic
  maps $(C_*(X), \partial) \to (C_*(Y), \partial')$
  and thus $f_* = g_* : H_n(X) \to H_n(Y)$.
\end{theorem}

\begin{proof}
  Let $H : X \times [0, 1] \to Y$ be the homotopy, so
  \[H(x, 0) = f(x) \quad \text{and} \quad H(x, 1) = g(x)\]
  Given a simplex $\sigma^n \to X$, define
  $p_n(\sigma)$ by
  \begin{align*}
    H \circ (\sigma \times \id_{[0, 1]})
    : \Delta^n \times [0, 1]
    &\longmapsto Y \\
    (x, t) &\longmapsto H(\sigma(x), t).
  \end{align*}
  To make sense of this, we need to see how to write
  $H \circ (\sigma \times \id_{[0, 1]})$ as a sum of
  singular $(n + 1)$-simplices in $Y$.
  For $n = 0$, note that the interval $\{e_0\} \times [e_0, e_1]$ is
  already $\Delta^1$, so
  \begin{align*}
    p_0(\sigma)
    &= H \circ (\sigma \times \id_{[0, 1]})
    = H(e_0, t) \\
    \partial_1 (p_0(\sigma))
    &= \partial(H(e_0, t))
    = (H(e_0, t))^{(0)} - (H(e_0, t))^{(1)} \\
    &= H(e_0, 1) - H(e_0, 0)
    = g(e_0) - f(e_0) = g \circ \sigma - f \circ \sigma.
  \end{align*}
  Letting $p_{-1} = 0$, we have
  $g_0 - f_0 = \partial_1 \circ p_0 + p_{-1} \circ \partial_0$ as
  desired.
  For $n = 1$, note that $[e_0, f_0] \times [e_0, e_1]$
  (the last corner is $f_1$) is a square, but we
  can divide it into two $2$-simplices by
  \[
    \tau_1 = [e_0, e_1, f_1] \quad \text{and} \quad
    \tau_0 = [e_0, f_0, f_1].
  \]
  Let $H_\sigma = H \circ (\sigma \times \id_{[0, 1]})$,
  then we have
  \begin{align*}
    p_1(\sigma)
    &= H_\sigma|_{\tau_0} - H_\sigma|_{\tau_1} \\
    \partial_2 (p_1(\sigma))
    &= (H_\sigma|_{[f_0, f_1]} - H_\sigma|_{[e_0, f_1]} + H_\sigma|_{[e_0, f_0]})
    - (H_\sigma|_{[e_1, f_1]} - H_\sigma|_{[e_0, f_1]} + H_\sigma|_{[e_0, e_1]}) \\
    &= H_\sigma|_{[f_0, f_1]}
    - H_\sigma|_{[e_0, e_1]}
    + H_\sigma|_{[e_0, f_0]}
    - H_\sigma|_{[e_1, f_1]} \\
    &= g(\sigma) - f(\sigma) - p_0(\partial \sigma)
  \end{align*}
  In general, let $e_0, \dots, e_n$ be the
  vertices of $\Delta^n$ in $\R^{n + 1}$, where
  we think of $\R^{n + 1} \subseteq \R^{n + 2}$
  with $x_{n + 2} = 0$. Let $f_0, \dots, f_n$
  be the points in $\R^{n + 2}$ above
  $e_0, \dots, e_n$ with $x_{n + 2} = 1$.
  So $[e_0, \dots, e_n]$ describes
  $\Delta^n \times \{0\}$ and $[f_0, \dots, f_n]$
  describes $\Delta^n \times \{1\}$.
  Show as an exercise that $D^n \times [0, 1]$
  is the union of $(n + 1)$-simplices
  \[
    [e_0, \dots, e_i, f_i, \dots, f_n].
  \]
  Now we can define $p_n : C_n(X) \to C_{n + 1}(Y)$
  by
  \begin{align*}
    p_n(\sigma)
    &= \sum_{i = 0}^n (-1)^i (H \circ (\sigma \times \id_{[0, 1]}))|_{[e_0, \dots, e_i, f_i, \dots, f_n]} \\
    \partial_{n + 1}(p_n(\sigma))
    &=
    \sum_{j \le i} (-1)^i (-1)^j (H \circ (\sigma \times \id_{[0, 1]}))|_{[e_0, \dots, \widehat{e}_j, \dots, e_i, f_i, \dots, f_n]} \\
    & \quad \quad + \sum_{i \le j} (-1)^i (-1)^{j + 1} (H \circ (\sigma \times \id_{[0, 1]}))|_{[e_0, \dots, e_i, f_i, \dots, \widehat{f}_j, \dots, f_n]}.
  \end{align*}
  The $i = j$ terms all cancel except for
  \[
    [\widehat{e}_0, f_0, \dots, f_n] = [f_0, \dots, f_n] \quad \text{and} \quad
    [e_0, \dots, e_n, \widehat{f}_n] = [e_0, \dots, e_n],
  \]
  which give $g \circ \sigma$ and
  $-(f \circ \sigma)$, respectively. Check
  as an exercise that many other terms cancel,
  and those which remain are precisely those
  which give $p_{n - 1}(\partial_n \sigma)$.
  This would prove the result.
\end{proof}

\begin{corollary}
  If $f : X \to Y$ is a homotopy equivalence,
  then
  \[f_* : H_n(X) \to H_n(Y)\]
  is an isomorphism for all $n$.
\end{corollary}

\begin{proof}
  There is a homotopy inverse $g : Y \to X$
  such that $(f \circ g) \simeq \id_Y$
  and $(g \circ f) \simeq \id_X$. Then
  \[
    f_* \circ g_* = (f \circ g)_* = (\id_Y)_* = \id_{H_n(Y)}
    \quad \text{and} \quad
    g_* \circ f_* = (g \circ f)_* = (\id_{X})_* = \id_{H_n(X)}.
  \]
  The first says that $f_*$ is surjective and the
  second says $f_*$ is injective, so $f_*$ is
  an isomorphism.
\end{proof}

\begin{remark}
  The corollary shows that if $X$ is contractible,
  then
  \[
    H_n(X)
    \cong
    \begin{cases}
      \Z & \text{if } n = 0, \\
      0 & \text{if } n \ne 0.
    \end{cases}
  \]
\end{remark}

\section{Relative Homology}
\begin{remark}
  Let $A$ be a subspace of $X$. Then $C_n(A) \le C_n(X)$,
  and note that $C_n(A)$ lies in the kernel of
  \begin{center}
    \begin{tikzcd}
      C_n(X) \ar[r, "\partial_n"] & C_{n - 1}(X) \ar[r, "\text{quotient}"] & C_{n - 1}(X) / C_{n - 1}(A)
    \end{tikzcd}
  \end{center}
  So $\partial_n$ induces a homomorphism
  $\partial_n : C_n(X) / C_n(A) \to C_{n - 1}(X) / C_{n - 1}(A)$.
\end{remark}

\begin{definition}
  Define $C_n(X, A) = C_n(X) / C_n(A)$ and
  $\partial_n : C_n(X, A) \to C_{n - 1}(X, A)$
  We still have $\partial_{n - 1} \circ \partial_n = 0$
  for all $n$, so this is a chain complex.
  The \emph{relative homology} of $(X, A)$ is
  \[
    H_n(X, A) = H_n(C_*(X, A), \partial).
  \]
\end{definition}
