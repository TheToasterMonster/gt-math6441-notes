\chapter{Feb.~24 --- Singular Homology, Part 2}

\section{Simple Computations in Homology}

\begin{lemma}
  If $X$ is a one-point space, then we have
  \[
    H_p(X) =
    \begin{cases}
      \Z & \text{if $p = 0$}, \\
      0 & \text{otherwise}.
    \end{cases}
  \]
\end{lemma}

\begin{proof}
  Since $X$ is a one-point space, for each $p \ge 0$
  there is a unique map $\sigma_p : \Delta^p \to X$.
  So $C_p(X) \cong \Z$, which is generated by $\sigma_p$.
  We can calculate the boundary of $\sigma_p$:
  \[
    \partial_p \sigma_p \sum_{i = 1}^p (-1)^i \sigma_p^{(i)}
    = \sum_{i = 0}^p (-1)^i \sigma_{p - 1}
    =
    \begin{cases}
      0 & \text{if $p$ is odd or $p = 0$}, \\
      \sigma_{p - 1} & \text{if $p$ is even and $p > 0$}.
    \end{cases}
  \]
  So if $p$ is odd, then $H_p(X) = \ker \partial_p / {\im \partial_{p + 1}} \cong \Z / \Z \cong \{0\}$.
  If $p$ is even and $p > 0$, then
  \[H_p(X) = \frac{\ker \partial_p}{\im \partial_{p + 1}} \cong \frac{\{0\}}{\{0\}} \cong \{0\}.\]
  Finally, we have $H_0(X) = \ker \partial_0 / {\im \partial_1} \cong \Z / \{0\} \cong \Z$, which proves
  the claim.
\end{proof}

\begin{theorem}
  We have $H_0(X) \cong \bigoplus_n \Z$,
  where $n$ is the number of path components of $X$.
\end{theorem}

\begin{proof}
  A singular $0$-simplex $\sigma_0 : \Delta^0 = \{e_0\} \to X$
  is determined by its image, i.e. by a point in $X$.
  So an element $\sigma \in C_0(X)$ is
  $c = \sum_{i = 0}^k n_i x_i$, where
  $n_i \in \Z$ and $x_i \in X$. Define the map
  \begin{align*}
    \varepsilon : C_0(X) &\longrightarrow \Z \\
    \sum_{i = 0}^k n_i x_i &\longmapsto \sum_{i = 0}^k n_i.
  \end{align*}
  It is easy to see that $\varepsilon$ is a homomorphism.
  If $\sigma$ is a singular $1$-chain (with image a
  curve), then
  \[
    \partial \sigma = \sum_{i = 0}^1 (-1)^i \sigma^{(i)}
    = \sigma^{(0)} - \sigma^{(1)}
    = \sigma(1) - \sigma(0).
  \]
  Thus $\varepsilon (\partial \sigma) = 0$.
  More generally, if $c \in C_1(X)$ with
  $c = \sum_{i = 0}^k n_i \sigma_i$, then
  \[
    \partial c = \sum_{i = 0}^k n_i \partial \sigma_i
    = \sum_{i = 0}^k n_i (\sigma_i(1) - \sigma_i(0)),
  \]
  so $\varepsilon (\partial c) = 0$. This shows that
  $\im \partial_1 \subseteq \ker \varepsilon$.
  Recalling that
  \[
    H_0(X) = \frac{\ker \partial_0}{\im \partial_1}
    = \frac{C_0(X)}{\im \partial_1},
  \]
  we see that $\varepsilon$ induces a map
  $\varepsilon_* : H_0(X) \to \Z$ by
  $[c] \mapsto \varepsilon(c)$, which is
  called an \emph{augmentation}.

  We now claim that if $X$ is path connected, then
  $\varepsilon_*$ is an isomorphism. Clearly
  $\varepsilon_*$ is onto, since
  $\varepsilon_*([x]) = 1$ and $1$ is a generator
  for $\Z$. Now fix a point $x_0 \in X$. Then
  for any $x \in X$ let $\lambda_x : [0, 1] \to X$
  be a path from $x_0$ to $x$, which exists since
  $X$ is path connected. So $\lambda_x$ is a singular
  $1$-chain, and
  \[
    \partial_1 \lambda_x = x - x_0.
  \]
  Now given $[c] \in H_0(X)$ such that
  $\varepsilon_*([c]) = 0$, we can write
  $c = \sum_{i = 0}^k n_i x_i$. Then
  \[
    \partial_1 \left(\sum_{i = 0}^k n_i \lambda_{x_i}\right)
    = \sum_{i = 0}^k n_i (x_i - x_0)
    = \sum_{i = 0}^k n_i x_i - \left(\sum_{i = 0}^k n_i\right) x_0
    = \sum_{i = 0}^k n_i x_i
    = c
  \]
  since $\varepsilon_*([c]) = 0$, which means that
  $[c] = 0$ in $H_0(X)$. So $\varepsilon_*$ is
  injective, and hence an isomorphism.

  Finally, in the general case, check as an exercise
  that if the path components of $X$ are $X_{\alpha}$
  for $\alpha \in A$, then
  $C_p(X) = \bigoplus_{\alpha \in A} C_p(X_{\alpha})$
  and $H_p(X) = \bigoplus_{\alpha \in A} H_p(X_{\alpha})$.\footnote{The main idea is that $\Delta^p$ is connected, so $\sigma(\Delta^p)$ must land in a single path component $X_\alpha$, so $C_p(X)$ decomposes. Similarly, the $\partial$ sends an element of $C_{p + 1}(X_\alpha)$ to an element of $C_{p}(X_\alpha)$, so $H_p(X)$ decomposes.} This suffices to prove the theorem.
\end{proof}

\begin{remark}
  If we set $\widetilde{\partial}_0 = \varepsilon$
  and $\widetilde{\partial}_i = \partial_i$ for $i > 0$,
  then the proof above shows $\widetilde{\partial}_i \circ \widetilde{\partial}_{i + 1} = 0$ for each $i$.
  So we can take the homology and
  define the \emph{reduced homology groups}
  $\widetilde{H}_i(X)$ via
  \[
    \widetilde{H}_i(X) = \frac{\ker \widetilde{\partial}_i}{\im \widetilde{\partial}_{i + 1}}.
  \]
  Clearly $\widetilde{H}_i(X) \cong H_i(X)$ for $i \ge 1$,
  but $H_0(X) = \widetilde{H}_0(X) \oplus \Z$. In this
  language, we can write
  \[
    \widetilde{H}_i(\{\text{pt}\}) = 0, \quad \text{for all $i$}.
  \]
\end{remark}

\section{Homology and the Fundamental Group}

\begin{remark}
  If $\gamma : [0, 1] \to X$ is a loop based at
  $x_0$, then $\gamma$ is a singular $1$-chain
  with $\partial_1 \gamma = x_0 - x_0 = 0$, so
  that $[\gamma] \in H_1(X)$. This gives a map
  $\phi : \pi_1(X, x_0) \to H_1(X)$
  called the \emph{Hurewitz map}. We will see that it is
  well-defined and a homomorphism in the next theorem.
\end{remark}

\begin{definition}
  The \emph{abelianization} $G^{\mathrm{ab}}$ of a group $G$
  is the largest abelian quotient of $G$. This means
  that if $A$ is any abelian group and
  $f : G \to A$, then $f$ factors through
  $G^{\mathrm{ab}}$, i.e.
  \begin{center}
    \begin{tikzcd}
      G \ar[r, "f"] \ar[dr, swap, "q"] & A \\
      & G^{\mathrm{ab}} \ar[u, dashed, swap, "\widetilde{f}"]
    \end{tikzcd}
  \end{center}
  where $q : G \to G^{\mathrm{ab}}$ is the quotient map.
\end{definition}

\begin{exercise}
  Show that $G^{\mathrm{ab}} \cong G / [G, G]$,
  where $[G, G]$ is the \emph{commutator subgroup} of
  $G$, i.e. the smallest normal subgroup of $G$
  containing $\{xyx^{-1}y^{-1} \mid x, y \in G\}$.
\end{exercise}

\begin{theorem}
  If $X$ is path connected, then the Hurewitz map
  induces an isomorphism
  \[
    \phi_* : (\pi_1(X, x_0))^{\mathrm{ab}} \to H_1(X),
  \]
  where $(\pi_1(X, x_0))^{\mathrm{ab}}$ is the abelianization of $\pi_1(X, x_0)$.
\end{theorem}

\begin{proof}
  Denote an equivalence class in $\pi_1(X, x_0)$
  by $[\gamma]$ and one in $H_1(X)$
  by $\llbracket \gamma \rrbracket$. Note that:
  \begin{enumerate}
    \item If $\gamma, \eta$ are paths in
      $X$ with $\gamma(1) = \eta(0)$, then
      the $1$-chain $\gamma * \eta - \gamma - \eta$
      is a boundary.

      To see this, define $\sigma : \Delta^2 \to X$
      on $[e_0, e_1, e_2]$ by
      $\gamma$ on the edge $[e_0, e_1]$ and $\eta$
      on the edge $[e_1, e_2]$. Extend $\sigma$ to
      be constant on parallel lines perpendicular
      to $[e_0, e_2]$, where $\sigma$ takes the
      value of the intersection point of the line with
      one of the other edges $[e_0, e_1]$ or $[e_1, e_2]$.
      Note that this defines $\sigma$ on $[e_0, e_1]$
      to be $\gamma * \eta$. Thus we can compute that
      \[
        \partial(-\sigma) = -\sigma^{(0)} + \sigma^{(1)} - \sigma^{(2)}
        = - \eta + \gamma * \eta - \gamma.
      \]
    \item If $\gamma$ is a path in $X$, then
      $\gamma + \overline{\gamma}$ is a boundary, and
      a constant path is a boundary.

      First if $\gamma$ is a constant path, then we
      can simply let $\sigma$ be a singular $2$-simplex
      that is the same constant. Then
      $\partial \sigma = \sigma^{(0)} - \sigma^{(1)} + \sigma^{(2)} = \gamma - \gamma + \gamma = \gamma$.

      Now given any $\gamma$, let
      $\sigma$ be the singular $2$-simplex
      which is defined on $[e_0, e_1]$ to be $\gamma$.
      Let $\sigma$ be constant on lines perpendicular
      to $[e_1, e_2]$. Note that this defines $\sigma$
      to be some constant $c$ on $[e_0, e_2]$ and
      $\overline{\gamma}$ on $[e_1, e_2]$. Let
      $\sigma'$ be a singular $2$-simplex that is
      constantly $c$, then
      \[
        \partial(\sigma + \sigma')
        = (\sigma^{(0)} - \sigma^{(1)} + \sigma^{(2)}) + c
        = (\overline{\gamma} - c + \gamma) + c
        = \overline{\gamma} + \gamma.
      \]
    \item If $\gamma$ and $\eta$ are homotopic
      rel endpoints, then $\gamma - \eta$ is a boundary.

      Let $H : [0, 1] \times [0, 1] \to X$ be the
      homotopy from $\gamma$ to $\eta$. View
      $H$ on the unit square as $\gamma$ on the
      bottom, $\eta$ on the top, and constant
      on the sides. Since $H$ is constant on the
      left edge, we can collapse the left edge to a
      point to get a $2$-simplex, so $H$ descends to
      a map $\sigma_H : \Delta^2 \to X$. Note that
      $\sigma_H$ is $\gamma$ on $[e_0, e_1]$,
      $\eta$ on $[e_0, e_2]$, and a constant $c$
      on $[e_1, e_2]$. Then
      \[
        \partial \sigma_H = \sigma_H^{(0)} - \sigma_H^{(1)} + \sigma_H^{(2)}
        = c - \eta + \gamma.
      \]
      Since the constant path $c$ is a boundary,
      the above implies that $\gamma - \eta$ also is.
  \end{enumerate}
  Note that $(3)$ implies
  $\phi : \pi_1(X, x_0) \to H_1(X)$ is well-defined, and
  $(1)$ implies $\phi$ is a homomorphism
  since
  \[
    \phi([\gamma] \cdot [\eta])
    = \phi([\gamma * \eta])
    = \llbracket \gamma * \eta \rrbracket
    \overset{(1)}{=} \llbracket \gamma \rrbracket + \llbracket \eta \rrbracket
    = \phi([\gamma]) + \phi([\eta]).
  \]
  Since $H_1(X)$ is abelian, $\phi$ induces
  a homomorphism
  $\phi_* : (\pi_1(X, x_0))^{\mathrm{ab}} \to H_1(X)$.

  To see that $\phi_*$ is a bijection, we construct
  an inverse to $\phi_*$. For each $x \in X$, let
  $\gamma_x$ be a path from $x_0$ to $x$ (if $x = x_0$,
  we let $\gamma_x$ be the constant path). Given a
  singular $1$-simplex $\sigma$, let
  $\widehat{\sigma} = \gamma_{\sigma(0)} * \sigma * \overline{\gamma}_{\sigma(1)}$, and
  define $\psi(\sigma) = [\widehat{\sigma}]$.
  Since $(\pi_1(X, x_0))^{\mathrm{ab}}$ is abelian
  and $C_1(X)$ is a free abelian group, this defines
  a homomorphism $\psi : C_1(X) \to (\pi_1(X, x_0))^{\mathrm{ab}}$. Note that
  \[
    \psi \circ \phi_*([\gamma]) = [e_{x_0} * \gamma * \overline{e_{x_0}}] = [\gamma],
  \]
  where $e_{x_0}$ is the constant path at $x_0$.
  If $\sigma$ is a singular $2$-simplex with vertices
  $y_0, y_1, y_2$, then
  \begin{align*}
    \psi(\partial_2 \sigma)
    &= \psi(\sigma^{(0)} - \sigma^{(1)} + \sigma^{(2)})
    = \psi(\sigma^{(2)}) \psi(\sigma^{(0)}) \psi(\sigma^{(1)})^{-1} \\
    &= [\gamma_{y_0} * \sigma^{(2)} * \overline{\gamma}_{y_1} * \gamma_{y_1} * \overline{\sigma}^{(1)} * \overline{\gamma}_{y_2} * \gamma_{y_2} * \sigma^{(0)} * \overline{\gamma}_{y_0}] \\
    &= [\gamma_{y_0} * \sigma^{(2)} * \sigma^{(1)} * \overline{\sigma}^{(0)} * \overline{\gamma}_{y_0}]
    = [e_{x_0}]
  \end{align*}
  since $\gamma_{y_0} * \sigma^{(2)} * \sigma^{(1)} * \overline{\sigma}^{(0)} * \overline{\gamma}_{y_0}$
  bounds a disk. Thus $\im \partial_2 \subseteq \ker \psi$,
  and so $\psi$ induces a homomorphism
  $\psi_* : H_1(X) \to (\pi_1(X, x_0))^{\mathrm{ab}}$.
  From above, we know that $\psi_* \circ \phi_* = \id$,
  and one can show as an exercise that
  $\phi_* \circ \psi_* = \id$. This shows that
  $\phi_*$ is an isomorphism.
\end{proof}
