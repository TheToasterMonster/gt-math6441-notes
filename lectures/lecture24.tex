\chapter{Apr.~9 --- Poincar\'e Duality}

\section{More Products}

\begin{remark}
  Recall that we have a map
  $C_p(X) \times C^p(X; R) \to R$ by
  $(\beta, \alpha) \mapsto \alpha(\beta)$.
  Write $\alpha(\beta) = \langle \alpha, \beta)$.
\end{remark}

\begin{exercise}
  Show that the pairing $\langle \cdot, \cdot \rangle$
  is \emph{non-degenerate}, i.e.
  if $\alpha \in C^p(X; R)$ and
  $\langle \alpha, \beta \rangle = 0$
  for all $\beta \in C_p(X)$, then
  $\alpha = 0$, and similarly,
  if $\beta \in C_p(X)$ and
  $\langle \alpha, \beta \rangle = 0$
  for all $\alpha \in C^p(X; R)$, then
  $\beta = 0$.
\end{exercise}

\begin{definition}
  Given a non-degenerate pairing, we can
  look at the \emph{adjoint} of $U$
  with respect to this pairing, i.e. the
  map $\cap : C_{p + q}(X; R) \times C^p(X; R) \to C_q(X; R)$
  such that for $\alpha \in C^p(X; R)$ and
  $\beta \in C_{p + q}(X; R)$, the element
  $\beta \cap \alpha$ is the unique element
  in $C_q(X; R)$ such that
  \[
    \langle \beta \cap \alpha, \gamma \rangle
    = \langle \beta, \alpha \cup \gamma \rangle
    \quad \text{for all $\gamma \in C^q(X; R)$}
  \]

  Call this adjoint $\cap$ to be the
  \emph{cap product}.
\end{definition}

\begin{remark}
  If we think of $\alpha \cup \cdot$ as
  a map $C^q(X; R) \to C^{p + q}(X; R)$,
  then $\cdot \cap \alpha$ is the adjoint
  of this map with respect to the pairing.
  In this perspective, we can define $\cap$ as
  follows:
  \[
    \beta \cap \alpha = \alpha({}_p \beta) \beta_q.
  \]
\end{remark}

\begin{exercise}
  Check that $\cdot \cap \alpha$ is the
  adjoint of $\alpha \cup \cdot$.
\end{exercise}

\begin{exercise}
  Show that $\delta : C^{n - 1}(X) \to C^n(X)$
  is the adjoint of $\partial : C_n(X) \to C_{n - 1}(X)$, i.e.
  \[
    \langle \beta, \delta \alpha \rangle
    = \langle \partial \beta, \alpha \rangle.
  \]
\end{exercise}

\begin{lemma}
  $C_*(X; R)$ is a unitary $C^*(X; R)$-module
  using $\cap$.
\end{lemma}

\begin{proof}
  We need to see that $\beta \cap (\alpha \cup \gamma) = (\beta \cap \alpha) \cap \gamma$.
  We have ($\beta \in C_{p + q + r}(X)$)
  \[
    \beta \cap (\alpha \cup \gamma)
    = (\alpha \cup \gamma) ({}_{p + q} \beta) \beta_r
    = \alpha ({}_p \beta) \gamma(({}_{p + q} \beta)_q) \beta_r.
  \]
  We also have
  \[
    (\beta \cap \alpha) \cap \gamma
    = (\alpha ({}_p) \beta_{q + r}) \cap \gamma
    = \alpha({}_p \beta) \gamma({}_q(\beta_{q + r})) \beta_r,
  \]
  which are the same since
  $({}_{p + q} \beta)_q = {}_q(\beta_{q + r})$.
  Check the rest (module axioms) as an exercise.
\end{proof}

\begin{lemma}
  If $\beta \in C_{p + q}(X; R)$ and
  $\alpha \in C^p(X; R)$, then
  \[\partial(\beta \cap \alpha)
    = (-1)^p (\partial \beta \cap \alpha - \beta \cap \delta \alpha).\]
\end{lemma}

\begin{proof}
  By non-degenerateness, it suffices to see
  that the equality
  holds when we pair with all possible
  elements in $C^{q - 1}(X; R)$. For the
  right-hand side, we have
  \begin{align*}
    (-1)^p \langle \partial \beta \cap \alpha - \beta \cap \delta \alpha, \gamma \rangle
    &= (-1)^p(\langle \partial \beta \cap \alpha, \gamma) \rangle - \langle \beta \cap \delta \alpha, \gamma \rangle) \\
    &= (-1)^p(\langle\partial \beta, \alpha \cup \gamma \rangle - \langle \beta, \delta \alpha \cup \gamma \rangle) \\
    &= (-1)^p(\langle \beta, \delta(\alpha \cup \gamma) \rangle - \langle \beta, \delta \alpha \cup \gamma) \\
    &= (-1)^p (\langle \beta, \delta \alpha \cup \gamma \rangle + (-1)^p \langle \beta, \alpha \cup \delta \gamma \rangle - \langle \beta, \delta \alpha \cup \gamma) \\
    &= \langle \beta, \alpha \cup \delta \gamma \rangle
    = \langle \beta \cap \alpha, \delta \gamma \rangle
    = \langle \partial(\beta \cap \alpha), \gamma \rangle,
  \end{align*}
  where we used that $\cap$ is the
  adjoint of $\cup$ and $\delta$ is the adjoint
  of $\partial$. This proves the result.
\end{proof}

\begin{exercise}
  Use the lemma to show that $\cap$ is
  well-defined on cohomology and homology.
\end{exercise}

\begin{lemma}
  Let $f : X \to Y$. Then for
  any $\beta \in H_{p + q}(X; R)$ and
  $\alpha \in H^p(Y; R)$,
  \[
    f_*(\beta \cap f^* (\alpha))
    = f_*(\beta) \cap \alpha
  \]
\end{lemma}

\begin{proof}
  Check this as an exercise.
\end{proof}

\begin{remark}
  We can also get a relative cap product
  \begin{align*}
    H_{p + q}(X, A; R) \times H^p(X, A; R)
    &\longrightarrow H_q(X; R) \\
    H_{p + q}(X, A; R) \times H^p(X; R)
    &\longrightarrow H_q(X, A; R)
  \end{align*}
  We will check that the first map is
  well-defined, the second is similar.

  To check this, let
  $[\beta] \in H_{p + q}(X, A; R)$
  and $[\alpha] \in H^p(X, A; R)$, so
  $\beta \in C_{p + q}(X, A; R)$ and
  $\alpha \in C^p(X; R)$, where
  $\alpha$ vanishes on $C_p(A)$.
  We need to see that $\partial (\beta \cap \alpha)$
  is zero in $C^{q - 1}(X)$ and not just
  in $C_{q - 1}(X, A)$.
  Write $\partial(\beta \cap \alpha) = (-1)^p (\partial \beta \cap \alpha - \beta \cap \delta \alpha)$, and note that
  \begin{enumerate}
    \item $\langle \partial \beta \cap \alpha, \gamma \rangle = \langle \partial \beta, \alpha \cup \gamma \rangle = 0$
      since $\partial \beta \in C_{p + q - 1}(A)$,
      so $\partial \beta \cap \alpha = 0$.
    \item $\langle \beta \cap \delta \alpha, \gamma \rangle = \langle \beta, \delta \alpha \cup \gamma \rangle = (\delta \alpha \cup \gamma)(\beta) = \delta \alpha({}_{p + 1}(\beta)) \cdot \gamma(\beta_{q - 1}) = \alpha(\partial_{p + 1} \beta) \cdot \gamma(\beta_{q - 1}) = 0$
      since we have
      $\partial_{p + 1} \beta \in C_{p + 1}(A)$
      and
      $\delta \alpha$ vanishes on elements
      of $C_{p + 1}(A; R)$
  \end{enumerate}
  This shows that $\partial(\beta \cap \alpha) = 0$
  in $C_{q - 1}(X)$.
\end{remark}

\section{Poincar\'e Duality}

\begin{definition}
  A \emph{manifold of dimension $n$} is a
  topological space $M$ that is Hausdorff
  and \emph{locally Euclidean}, i.e. every
  point in $M$ has an open neighborhood
  $U$ which is homeomorphic to
  $\R^n$ (we call such an open neighborhood $U$ a \emph{coordinate chart}).
\end{definition}

\begin{remark}
  Many definitions ask for $M$ to be second
  countable, but we do not need that.
\end{remark}

\begin{definition}
  A \emph{manifold with boundary of dimension $n$}
  is a topological space $M$ that is Hausdorff
  and such that each point in $M$ has an
  open neighborhoods homeomorphic
  to $\R^n$ or
  \[
    \R^n_{\ge 0} = \{ (x_1, \ldots, x_n) \in \R^n : x_n \ge 0 \}.
  \]
  Define $\partial M$ to be the points in
  $M$ that do not have a neighborhood
  homeomorphic to $\R^n$, and define $\Int M$
  to be those points which do
  have a neighborhood homeomorphic to $\R^n$.
\end{definition}

\begin{exercise}
  Check the following properties:
  \begin{enumerate}
    \item $\partial (\partial M) = \varnothing$;
    \item $\partial(\Int M) = \partial M$;
    \item $\partial(\Int M) = \varnothing$;
    \item $\partial M$ is an $(n - 1)$-manifold.
  \end{enumerate}
\end{exercise}

\begin{definition}
  A manifold $M$ is \emph{closed} if
  $M$ is compact and $\partial M = \varnothing$.
\end{definition}

\begin{example}
  The following are examples of manifolds:
  \begin{enumerate}
    \item Surfaces are $2$-manifolds.
    \item $S^n \subseteq \R^{n + 1}$.
    \item Products of manifolds are manifolds,
      e.g. $S^n \times S^m$ is a manifold.
    \item $\R P^n$ is an $n$-manifold, and
      $\C P^n$ is a $2n$-manifold.\footnote{Recall that $\R P^n = (\R^{n + 1} - \{0\}) / (\R - \{0\})$ and $\C P^n = (\C^{n + 1} - \{0\}) / (\C - \{0\})$.}
  \end{enumerate}
\end{example}

\begin{theorem}
  Let $R$ be a ring. Then:
  \begin{enumerate}
    \item Let $M$ be a closed connected
      manifold of dimension $n$. Then
      $M$ is $R$-orientable if and only if
      $H_n(M; R) \cong R$.
    \item Let $M$ be a compact connected
      manifold with boundary of dimension $n$.
      Then $M$ is $R$-orientable if and only
      if $H_n(M, \partial M; R) \cong R$.
  \end{enumerate}
\end{theorem}

\begin{remark}
  We will define $R$-orientability later,
  but for now, here are some facts:
  \begin{enumerate}
    \item Any manifold is $\Z / 2\Z$-orientable.
    \item The ``standard'' definition of
      orientable (e.g. in differential
      topology) is equivalent to
      $\Z$-orientability.
    \item A choice of generator for $H_n(M; R)$
      is called a \emph{fundamental class}
      of $M$, denoted by $[M]$, and $[M]$
      ``determines'' an $R$-orientation on
      $M$ (similarly for $[M, \partial M]$
      and
      a generator for $H_n(M, \partial M; R)$.
  \end{enumerate}
\end{remark}

\begin{theorem}
  We have the following:
  \begin{enumerate}
    \item \emph{(Poincar\'e duality)}
      If $M$ is a closed connected
      $R$-orientable $n$-manifold with
      fundamental class $[M]$, then the
      map $[M] \cap \cdot : H^p(M; R) \to H_{n - p}(M; R)$
      is an isomorphism.
    \item \emph{(Poincar\'e-Lefschetz duality)}
      If $M$ is a compact connected
      $R$-orientable $n$-manifold with boundary
      and $[M, \partial M]$ is a fundamental class,
      then $\partial[M, \partial M] = [\partial M]$, where
      \[
        \partial : H_n(M, \partial M; R) \to H_{n - 1}(\partial M; R)
      \]
      is the map in the long exact sequence of
      the pair $(M, \partial M)$. Moreover,
      \begin{center}
        \hspace{-1em}
        \begin{tikzcd}
          \cdots \ar[r] & H^{p - 1}(M) \ar[r] \ar[d, "{[M, \partial M] \cap \cdot}"] & H^{p - 1}(\partial M) \ar[d, "{[\partial M] \cap \cdot}"] \ar[d, swap, "\cong"]
          \ar[r] & H^p(M, \partial M) \ar[d, "{[M, \partial M] \cap \cdot}"]
          \ar[r] & H^p(M) \ar[r] \ar[d, "{[M, \partial M] \cap \cdot}"]& \cdots \\
          \cdots \ar[r] & H_{n - p - 1}(M, \partial M)
          \ar[r] & H_{n - p}(\partial M)
          \ar[r] & H_{n - p}(M)
          \ar[r] & H_{n - p}(M, \partial M) \ar[r] & \cdots
        \end{tikzcd}
      \end{center}
  \end{enumerate}
  where all vertical arrows are isomorphisms
  and the diagram commutes up to sign.
\end{theorem}

\begin{remark}
  We will prove these theorems, but we
  look at some applications of Poincar\'e duality
  for now.
\end{remark}

\begin{corollary}
  Let $M$ be a closed oriented $n$-fold.
  Then the cup product pairing
  \begin{align*}
    \frac{H^p(M)}{\mathrm{tors}}
    \times \frac{H^{n - p}(M)}{\mathrm{tors}}
    &\longrightarrow \Z \\
    (\alpha, \beta)
    &\longmapsto \alpha \cup \beta([M])
  \end{align*}
  is non-degenerate, and if $\alpha$ is a
  generator of  $H^p(M) / \mathrm{tors}$, then there
  exists $\beta \in H^{n - p}(M) / \mathrm{tors}$ such that
  $\alpha \cup \beta$ is a generator of
  $\Z \cong H^n(M)$.
\end{corollary}
