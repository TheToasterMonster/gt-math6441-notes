\chapter{Feb.~19 --- Singular Homology}

\section{Covering Spaces and Group Actions}

\begin{definition}
  A \emph{group action} on a topological space $X$ is
  a pair $(G, \rho)$ where $G$ is a group and
  $\rho : G \to \Homeo(X)$ is a group homomorphism.\footnote{Here $\Homeo(X)$ is the group of homeomorphisms of $X$.}
\end{definition}

\begin{remark}
  If $G$ acts on $X$, then we can form the quotient
  space $X / G$ where two points $x_1, x_2 \in X$ are
  identified if there exists $g \in G$ such that
  $\rho(g)(x_1) = x_2$. Then $X / G$ is called the
  \emph{orbit space} of the action.
\end{remark}

\begin{theorem}
  Let $G$ be a group acting on $X$ such that $(*)$
  for all $x \in X$, there is a neighborhood
  $U$ of $x$ such that $g_1 U \cap g_2 U \ne \varnothing$
  implies $g_1 = g_2$. Then:
  \begin{enumerate}
    \item $\rho : X \to X / G$ is a normal covering
      space;
    \item $G \cong G(X \to X / G)$ if $X$ is
      path-connected ($G(X \to X / G)$ is
      the group of deck transformations);
    \item $G \cong \pi_1(X / G) / \rho_*(\pi_1(X))$
      if $X$ is path connected and locally path
      connected.
  \end{enumerate}
\end{theorem}

\begin{proof}
  This is left as an exercise (or see Hatcher). Note
  that the hypothesis implies each point in
  $X / G$ has an evenly covered neighborhood, so
  $(X, \rho)$ is a covering space. Also, any two
  preimages of a point in $X / G$ are related by the
  group action, i.e. is a deck transformation
  of $X$, so $(X, \rho)$ is normal.
\end{proof}

\begin{exercise}
  Show that if $G$ is a finite group acting freely on a
  Hausdorff space $X$ (i.e. no element of $G$ has a
  fixed point), then the action satisfies $(*)$.
\end{exercise}

\begin{example}
  Let $X$ be the four-holed torus, arranged in a
  triangular shape. Then rotation by $2\pi / 3$
  is an action of $G = \Z / 3\Z$ on $X$ with no fixed
  points. Then $X / G$ is a two-holed torus, and
  by the above theorem
  $X$ is a three-fold cover of $X / G$.
\end{example}

\begin{example}
  The group $G = \Z / 2\Z$ acts on $S^n$ by sending
  $x \mapsto -x$, so $S^n$ is a covering space
  of $S^n / G = \R P^n$. Note that we have already
  seen this before without appealing to the above
  theorem.
\end{example}

\section{Introduction to Homology}

\begin{remark}
  There are many types of (ordinary) homology, i.e.
  singular, simplicial, cubical, cellular, etc.
  All of these are isomorphic on CW-complexes.
  We will define the most general homology theory,
  called \emph{singular homology}, and then derive
  \emph{cellular homology}, which is easier to compute.
  
  There are also \emph{generalized homology} theories,
  such as \emph{bordism theory}, \emph{$K$-theory}, etc.
  These are in many ways similar to ordinary homology,
  but they do not satisfy the dimension axiom, which
  can give them wildly different behavior. We will
  briefly mention bordism theory later on in the class.

  Underlying these homology theories is
  \emph{homological algebra}, which is the study of
  \emph{chain complexes}. Such objects show up in
  many places and they give rise to
  more ``homology theories,'' such as
  \emph{Floer homology}, \emph{Heegaard-Floer homology}.
  These are \emph{not} ``real homology theories''
  in the sense of algebraic topology.
\end{remark}

\section{Singular Homology}

\begin{definition}
  The \emph{standard $p$-simplex} is
  \[
    \Delta^p = \left\{
      \sum_{i = 0}^p t_i e_i \in \R^{p + 1}
      : \sum_{i = 0}^p t_i = 1, t_i \ge 0
    \right\},
  \]
  where $\{e_0, \dots, e_p\}$ are the standard basis
  vectors for $\R^{p + 1}$.
\end{definition}

\begin{example}
  The following are examples of $p$-simplices:
  \begin{enumerate}
    \item For $p = 0$, we have
      $\Delta^0 = \{t_1 e_1 \in \R : t_1 = 1, t_1 \ge 0\} = e_1$,
      which is a point.
    \item For $p = 1$, we have
      $\Delta^1 = \{t_0 e_0 + t_1 e_1 \in \R^2 : t_0 + t_1, t_i \ge 0\}$
      is the convex hull of the points $e_0$ and $e_1$
      in $\R^2$, which is a line segment.
    \item For $p = 2$, the standard $2$-simplex
      $\Delta^2$ is
      the convex hull of $e_0, e_1, e_2 \in \R^3$,
      which is a triangle.
  \end{enumerate}
\end{example}

\begin{definition}
  Given any vectors $v_0, \dots, v_p \in \R^n$, by
  $[v_0, \dots, v_p]$ we denote the map
  \begin{align*}
    \Delta^p &\longrightarrow \R^n \\
    (t_0, \dots, t_p) &\longmapsto \sum_{i = 0}^p t_i v_i
  \end{align*}
\end{definition}

\begin{example}
  Consider the following examples:
  \begin{enumerate}
    \item $[e_0, \dots, e_p]$ parametrizes the
      standard $p$-simplex.
    \item $[e_0, \dots, \widehat{e}_i, \dots, e_p]$
      (where the hat means to remove $e_i$)
      is the $i$th \emph{face map}, which
      parametrizes the $i$th \emph{face} of $\Delta^p$.
      Denote this map by $F_i^p = [e_0, \dots, \widehat{e}_i, \dots, e_p]$.
  \end{enumerate}
\end{example}

\begin{remark}
  If $i > j$, then
  \[
    F_i^p \circ F_j^{p - 1} = [e_0, \dots, \widehat{e}_j, \dots, \widehat{e}_i, \dots, e_p],
  \]
  so $F_2^2 \circ F_1^1$ identifies $[e_0]$ with
  $[e_0]$ in $\Delta^2$.
  On the other hand, if $j \ge i$, then
  \[
    F_i^p \circ F_j^{p - 1} = [e_0, \dots, \widehat{e}_i, \dots, \widehat{e}_{j + 1}, \dots, e_p].
  \]
  For instance, $F_0^3 \circ F_1^2$ identifies
  $[e_0, e_1]$ with $[e_1, e_3]$ in $\Delta^3$.
\end{remark}

\begin{definition}
  A \emph{singular $p$-simplex} in a space $X$ is a
  continuous map $\sigma : \Delta^p \to X$.
  The \emph{singular group of $p$-chains} in $X$, denoted
  $C_p(X)$, is the free abelian group generated by singular $p$-simplices.
\end{definition}

\begin{remark}
  An element in $C_p(X)$ is a
  finite formal sum $\sum_{i = 1}^k n_i \sigma_i$,
  where $n_i \in \Z$ and $\sigma_i$ is a singular
  $p$-simplex.
  There is an obvious way to add elements in
  $C(X)$, and this makes $C(X)$ into a group
\end{remark}

\begin{definition}
  An element in $C_p(X)$ is called a \emph{singular $p$-chain}.
  Given a singular $p$-simplex $\sigma$, the
  $i$th \emph{face} of $\sigma$ is
  $\sigma^{(i)} = \sigma \circ F_i^p$, and the
  \emph{boundary} of $\sigma$ is
  \[
    \partial \sigma = \sum_{i = 0}^p (-1)^i \sigma^{(i)}.
  \]
\end{definition}

\begin{remark}
  If $\sigma$ is a singular $p$-simplex, then
  $\partial \sigma \in C_{p - 1}(X)$.
\end{remark}

\begin{example}
  Let $\Delta^2 = [e_0, e_1, e_2]$ and
  $\sigma : \Delta^2 \to X$. Then
  $\partial \sigma = \sigma^{(0)} - \sigma^{(1)} + \sigma^{(2)}$
  traces the edges $[\sigma(e_0), \sigma(e_1)]$,
  $[\sigma(e_1), \sigma(e_2)]$, and
  $[\sigma(e_2), \sigma(e_0)]$, which forms the
  boundary of
  $\sigma(\Delta^2) \subseteq X$.
\end{example}

\begin{definition}
  Define the $p$th \emph{boundary map} to be
  \begin{align*}
    \partial_p : C_p(X) &\longrightarrow C_{p - 1}(X) \\
    \sum_{i = 1}^k n_i \sigma_i &\longmapsto \sum_{i = 1}^k n_i (\partial \sigma_i).
  \end{align*}
\end{definition}

\begin{lemma}
  We have $\partial_{p - 1} \circ \partial_p = 0$,
  i.e. $\partial^2 = 0$.
\end{lemma}

\begin{proof}
  We can calculate
  \begin{align*}
    \partial_{p - 1} \circ \partial_p \sigma
    &= \partial_{p - 1} \left( \sum_{i = 0}^p (-1)^i (-1)^i \sigma \circ F_i^p \right) \\
    &= \sum_{i = 0}^p (-1)^i \left( \sum_{j = 0}^{p - 1} (-1)^j \sigma \circ F_i^p \circ F_j^{p - 1} \right) \\
    &= \sum_{0 \le j < i \le p} (-1)^{i + j} \sigma \circ F_i^p \circ F_j^{p - 1}
    + \sum_{0 \le i \le j \le p - 1} (-1)^{i + j} \sigma \circ F_i^p \circ F_j^{p - 1} \\
    &= \sum_{0 \le j < i \le p} (-1)^{i + j} \sigma \circ [e_0, \dots, \widehat{e}_j, \dots, \widehat{e}_i, \dots, e_p] \\
    &\quad \quad+ \sum_{0 \le i \le j \le p - 1} (-1)^{i + j} \sigma \circ [e_0, \dots, \widehat{e}_i, \dots, \widehat{e}_{j + 1}, \dots, e_p].
  \end{align*}
  Relabel $k = j + 1$ in the second sum, we find that
  \begin{align*}
    \partial_{p - 1} \circ \partial_p \sigma
    &= \sum_{0 \le j < i \le p} (-1)^{i + j} \sigma \circ [e_0, \dots, \widehat{e}_j, \dots, \widehat{e}_i, \dots, e_p] \\
    &\quad \quad + \sum_{0 \le i < k \le p - 1} (-1)^{i + k + 1} \sigma \circ [e_0, \dots, \widehat{e}_i, \dots, \widehat{e}_{k}, \dots, e_p] = 0
  \end{align*}
  since these are the same but with an extra $-1$ on
  the second sum, so they cancel out to $0$.
\end{proof}

\begin{remark}
  The lemma implies that $\im \partial_{p + 1} \subseteq \ker \partial_p$.
\end{remark}

\begin{definition}
  We define the \emph{$p$th (singular) homology group} of $X$ to be
  \[
    H_p(X) = \frac{\ker \partial_p}{\im \partial_{p + 1}}.
  \]
  An element of $\ker \partial_p$ is called a
  \emph{$p$-cycle}, and an element of $\im \partial_{p + 1}$
  is called a \emph{$p$-boundary}. We say that
  two chains $c_1, c_2 \in C(X)$ are \emph{homologous}
  if there exists some $d \in C_{p + 1}(X)$ such that
  $c_1 - c_2 = \partial d$.

\end{definition}

\begin{remark}
  Writing $c_1 \sim c_2$ if $c_1, c_2$ are homologous
  and
  letting the equivalence classes be denoted
  by $[c]$,
  we can also write $H_p$ as
  \[
    H_p(X) = \{[c] : \text{$c$ is a $p$-cycle}\}.
  \]
\end{remark}

\begin{remark}
  There are no singular $p$-simplices for $p < 0$,
  so $C_p(X) = \{0\}$ for $p < 0$.
\end{remark}
