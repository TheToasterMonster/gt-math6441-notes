\chapter{Mar.~31 --- Cohomology}

\section{Homology with Coefficients}

\begin{definition}
  Given any abelian group $G$ and a space $X$,
  let
  \[
    C_n(X; G) = \left\{
      \sum_{i = 1}^k g_i \sigma_i :
      g_i \in G, \text{$\sigma_i$ are singular $n$-simplices}
    \right\}
  \]
  and define
  \[
    \partial_n\left(\sum_{i = 1}^k g_i \sigma_i\right)
    = \sum_{i = 1}^k g_i \partial_n \sigma_i
    = \sum_{i = 1}^k \sum_{j = 0}^n g_i (-1)^j \sigma_i^{(j)}.
  \]
  Just like before, $\partial_n \circ \partial_{n + 1} = 0$, so we
  can define the \emph{homology of $X$ with
  coefficients in $G$} to be
  \[
    H_n(X; G) = \frac{\ker \partial_n}{\im \partial_{n + 1}}.
  \]
  If $G = \Z$, then this is the usual singular homology.
\end{definition}

\begin{remark}
  All theorems about the original definition of
  singular homology work in this setting as well.
  We can also define the \emph{cellular homology with coefficients in $G$}
  in a similar way.
\end{remark}

\begin{example}
  Consider $\R P^2$. Using $\Z$-coefficients,
  we get
  \begin{center}
    \begin{tikzcd}[row sep=small]
      0 \ar[r] & C_2^\CW(\R P^2, \Z) \ar[r] & C_1^\CW(\R P^2, \Z) \ar[r] & C_0^\CW(\R P^2, \Z) \ar[r] & 0 \\
               & \Z \ar[u, equal] \ar[r, "\times 2"] & \Z \ar[u, equal] \ar[r, "0"] & \Z \ar[u, equal]
    \end{tikzcd}
  \end{center}
  so we find that
  \[
    H_k(\R P^2; \Z) \cong
    \begin{cases}
      \Z & \text{if $k = 0$}, \\
      \Z / 2\Z & \text{if $k = 1$}, \\
      0 & \text{otherwise}.
    \end{cases}
  \]
  Using $\Z / 2\Z$-coefficients, however, we get
  \begin{center}
    \begin{tikzcd}[row sep=small]
      0 \ar[r] & C_2^\CW(\R P^2, \Z / 2\Z) \ar[r] & C_1^\CW(\R P^2, \Z / 2\Z) \ar[r] & C_0^\CW(\R P^2, \Z / 2\Z) \ar[r] & 0 \\
               & \Z / 2\Z \ar[u, equal] \ar[r, "0"] & \Z / 2 \Z \ar[u, equal] \ar[r, "0"] & \Z / 2\Z \ar[u, equal]
    \end{tikzcd}
  \end{center}
  since multiplication by $2$ in $\Z / 2\Z$ is
  the zero map. Thus we have
  \[
    H_k(\R P^2; \Z / 2\Z)
    \cong
    \begin{cases}
      \Z / 2\Z & \text{if $k = 0, 1, 2$}, \\
      0 & \text{otherwise}.
    \end{cases}
  \]
\end{example}

\begin{remark}
  We will see later that $H_k(X; \Z)$ determines
  $H_k(X; G)$ for any other abelian group $G$.
  However, $H_k(X; G)$ may be sufficient for
  some purposes and may be easier to work
  with.
\end{remark}

\section{Cohomology}

\begin{definition}
  A sequence of abelian groups $C^*$ and maps
  $\delta_n : C^n \to C^{n + 1}$ is called a
  \emph{cochain complex} if $\delta_{n + 1} \circ \delta_n = 0$.
  for all $n$. The ``homology'' of this complex
  is called the \emph{cohomology} of
  $(C^*, \delta)$:
  \[
    H^n(C^*, \delta) = \frac{\ker \delta_n}{\im \delta_{n - 1}}.
  \]
\end{definition}

\begin{definition}
  If $(C_*, \partial)$ is a chain complex and
  $G$ is any abelian group, then let
  \[
    C^n = \Hom(C_n, G) =
    \{\text{homomorphisms } C_n \to G\}
  \]
  and set $\delta_n = \partial_{n + 1}^* : C^n \to C^{n + 1}$,
  i.e. $\delta_n(\tau) = \tau \circ \partial_{n + 1}$ (this is the same dual map from linear algebra).
\end{definition}

\begin{remark}
  Note that we have
  \[
    (\partial_{n + 1} \circ \delta_n(\tau))(\sigma)
    = (\delta_n(\tau))(\partial_{n + 1} \sigma)
    = \tau(\partial_{n + 1} \circ \partial_{n + 1} (\sigma))
    = 0,
  \]
  so $\delta_{n + 1} \circ \delta_n = 0$
  and $(C^*, \delta)$ is a cochain complex.
\end{remark}

\begin{definition}
  Call $H^n(C_*; G) = \ker \delta_n / {\im \delta_{n - 1}}$
  the \emph{cohomology of $(C_*, \partial)$ with coefficients in $G$}. In the case that $G = \Z$,
  we usually abbreviate this via
  $H^n(C_*) = H^n(C_*; \Z)$.
\end{definition}

\begin{remark}
  As groups, $H^n(C_*; G)$ is
  determined by (and also determines)
  $H_n(C_*; G)$. However, we will see that we can
  put a ring structure on $\bigoplus_{n = 0}^\infty H^n(C_*; G)$,
  which is a stronger invariant.
\end{remark}

\begin{definition}
  If $(A_*, \partial)$, $(B_*, \partial')$ are
  two chain complexes and
  $\alpha : (A_*, \partial) \to (B_*, \partial')$
  is a chain map, then $\alpha^* : B^* \to A^*$
  given by $\beta \to \beta \circ \alpha$ is a
  \emph{cochain map}, i.e.
  $\delta \circ \alpha^* = \alpha^* \circ \delta'$.
  Hence, the map $\alpha$ induces a homomorphism
  $\alpha^* : H^n(B_*; G) \to H^n(A_*; G)$.
\end{definition}

\begin{exercise}
  Show the following:
  \begin{enumerate}
    \item If $\alpha : (A_*, \partial) \to (B_*, \partial')$
      and $\beta : (B_*, \partial') \to (C_*, \partial'')$
      are chain maps, then
      $(\beta \circ \alpha)^* = \alpha^* \circ \beta^*$.
    \item $\id^* = \id$ and
      $0^* = 0$, where $\id$ is the identity map
      and $0$ is the zero map.
  \end{enumerate}
\end{exercise}

\begin{remark}
  As mentioned above, $H_n(C_*; \partial)$
  determines $H^n(C_*, \partial)$, but this is not
  obvious.
\end{remark}

\begin{example}
  Consider the chain complex $(C_*, \partial)$:
  \begin{center}
    \begin{tikzcd}[row sep=small]
      0 \ar[r] & \Z \ar[r, "0"] & \Z \ar[r, "\times 2"] & \Z \ar[r, "0"] & \Z \ar[r] & 0 \\
               & C_3 \ar[u, equal] & C_2 \ar[u, equal] & C_1 \ar[u, equal] & C_0 \ar[u, equal]
    \end{tikzcd}
  \end{center}
  so we have $H_3(C_*) \cong H_0(C_*) \cong \Z$,
  $H_2(C_*) \cong 0$ and $H_1(C_*) \cong \Z / 2\Z$.
  After dualizing, we have
  \begin{center}
    \begin{tikzcd}[row sep=small]
               & C_3 \ar[d, equal] & C_2 \ar[d, equal] & C_1 \ar[d, equal] & C_0 \ar[d, equal] \\
      0 & \ar[l] \Z & \ar[l, "0"] \Z & \ar[l, "\times 2"] \Z & \ar[l, "0"] \Z & \ar[l] 0
    \end{tikzcd}
  \end{center}
  so we get
  $H^3(C_*) \cong H^0(C_*) \cong \Z$,
  $H_2(C_*) \cong \Z / 2\Z$ and $H_1(C_*) \cong 0$.
  In particular, we see that
  $H^n$ is not something simple
  like $\Hom(H_n, \Z)$ in general.
\end{example}

\begin{remark}
  There is a natural map from the cohomology and
  homology groups to $G$:
  \begin{align*}
    H^n(C_*; G) \times H_n(C_*; G) &\longrightarrow G, \\
    ([\alpha], [\beta]) &\longmapsto \alpha(\beta).
  \end{align*}
  (Check that this is well-defined.)
  So there is a ``natural'' map $\Phi : H^n(C_*; G) \to \Hom(H_n(C_*, \partial), G)$ by
  \begin{align*}
    [\alpha] \longmapsto (\phi_{[\alpha]} : H_n(C_*, \partial) &\longrightarrow G), \\
    [\beta] &\longmapsto \alpha(\beta).
  \end{align*}
  We want to understand $\Phi$ better: If
  $A$ is an abelian group, then there exist free
  abelian groups $F, R$ and homomorphisms such that
  the following sequence:
  \begin{center}
    \begin{tikzcd}
      0 \ar[r] & R \ar[r, "f"] & F \ar[r, "g"] & A \ar[r] & 0
    \end{tikzcd}
  \end{center}
  is exact (this follows by the first isomorphism
  theorem, since a subgroup of a free group is
  free).
\end{remark}

\begin{exercise}
  Show that functor
  $\Hom(\cdot, G)$ is \emph{left exact},
  i.e. given an exact sequence of the form
  \begin{center}
    \begin{tikzcd}
      G_1 \ar[r, "\alpha"] & G_2 \ar[r, "\beta"] & G_3 \ar[r] & 0
    \end{tikzcd}
  \end{center}
  then the following sequence:
  \begin{center}
    \begin{tikzcd}
      0 \ar[r] & \Hom(G_3, G) \ar[r, "\beta^*"] & \Hom(G_2, G) \ar[r, "\alpha^*"] & \Hom(G_1, G)
    \end{tikzcd}
  \end{center}
  is also exact, but given an exact sequence of
  the form
  \begin{center}
    \begin{tikzcd}
      0 \ar[r] & G_1 \ar[r, "\alpha"] & G_2
    \end{tikzcd}
  \end{center}
  then it is not always true that $\alpha^*$ is
  surjective, i.e. we can lose exactness on the
  right.
\end{exercise}

\begin{definition}
  Define the functor
  $\Ext(A, G) = \Hom(R, G) / {\im f^*}$.\footnote{The name ``$\Ext$'' comes from ``extension''.}
\end{definition}

\begin{remark}
  Note that the sequence
  \begin{center}
    \begin{tikzcd}
      0 \ar[r] & \Hom(A, G) \ar[r, "g^*"] & \Hom(F, G) \ar[r, "f^*"] & \Hom(R, G)
      \ar[r] & \Ext(A, G) \ar[r] & 0
    \end{tikzcd}
  \end{center}
  is exact, so $\Ext(A, G)$ is exactly the
  obstruction to $f^*$ being surjective.
\end{remark}

\begin{example}
  If $A = \Z$, then we have
  \begin{center}
    \begin{tikzcd}[row sep=small]
      & R \ar[d, equal] & F \ar[d, equal] & G \ar[d, equal] \\
      0 \ar[r] & 0 \ar[r] & \Z \ar[r, "\id"] & \Z \ar[r] & 0
    \end{tikzcd}
  \end{center}
  so $\Ext(\Z, G) = \Hom(R, G) / {\im f^*} = 0 / {\im f^*} = 0$.
  For $\Z / n\Z$, This is
  \begin{center}
    \begin{tikzcd}[row sep=small]
      & R \ar[d, equal] & F \ar[d, equal] & G \ar[d, equal] \\
      0 \ar[r] & \Z \ar[r, swap, "f"] \ar[r, "\times n"] & \Z \ar[r, swap, "g"] & \Z / n\Z \ar[r] & 0
    \end{tikzcd}
  \end{center}
  so we have $\Ext(\Z / n\Z, G) = \Hom(\Z, G) / {\im f^*} \cong G / n G$.
\end{example}

\begin{exercise}
  Show that $\Ext(\Z / n\Z, \Z / m\Z)$ =
  $\Z / d \Z$, where $d = \gcd(m, n)$.
\end{exercise}

\begin{exercise}
  Show the following:
  \begin{enumerate}
    \item $\Ext(A, G)$ is independent of $F, R, f, g$.
    \item $\Ext(H \oplus H', G) \cong \Ext(H, G) \oplus \Ext(H', G)$.
    \item $\Ext(H, G) = 0$ if $H$ is free.
    \item $\Ext(Z / n\Z, G) = G / nG$
    \item From above, we can compute
      $\Ext(H, G)$ for any finitely generated
      abelian groups $H, G$.
    \item $\Ext(G, \Q) = 0$ for any $G$.
  \end{enumerate}
\end{exercise}

\section{Universal Coefficient Theorem}

\begin{theorem}[Universal coefficient theorem]
  The following sequence:
  \begin{center}
    \begin{tikzcd}
      0 \ar[r] & \Ext(H_{n - 1}(C_*), G) \ar[r] & H^n(C_*; G) \ar[r, "\Phi"] & \Hom(H_n(C_*), G) \ar[r] & 0
    \end{tikzcd}
  \end{center}
  is exact, splits (i.e.
  $H^n(C_*; G) = \Ext(H_{n - 1}(C_*), G) \oplus \Hom(H_n(C_*), G)$ but not in a natural way), and is
  natural with respect
  to chain maps. In particular, the
  cohomology groups are determined by homology.
\end{theorem}

\begin{proof}
  This is purely a fact of algebra, see Hatcher.
\end{proof}

\begin{corollary}
  If $F_n$ is the free part of $H_n(C_*)$ and
  $T_n$ is the torsion part of $H_n(C_*)$, then
  \[H^n(C_*; \Z) \cong F_n \oplus T_{n - 1}.\]
\end{corollary}

\begin{proof}
  This follows from the universal coefficient theorem
  and the exercises.
\end{proof}

\begin{example}
  Suppose that we have
  \[
    H_n(C_n, \partial) =
    \begin{cases}
      \Z & \text{if $n = 0$}, \\
      \Z / 2\Z & \text{if $n$ is odd}, \\
      0 & \text{otherwise}.
    \end{cases}
  \]
  Then we can compute that
  \[
    H^n(C_*; \Z) =
    \begin{cases}
      \Z & \text{if $n = 0$}, \\
      0 & \text{if $n$ is odd}, \\
      \Z / 2\Z & \text{otherwise}.
    \end{cases}
  \]
  This follows directly from the corollary.

  For $\Z / 2\Z$-coefficients however, we need to
  use the universal coefficient theorem:
  \begin{align*}
    H^n(C_*; \Z / 2\Z)
    &=
    \begin{cases}
      \Hom(\Z; \Z / 2\Z) & \text{if $n = 0$}, \\
      \Hom(\Z / 2\Z; \Z / 2\Z) \oplus \Ext(0; \Z / 2\Z) & \text{if $n$ is odd}, \\
      \Hom(0; \Z / 2\Z) \oplus \Ext(\Z / 2\Z; \Z / 2\Z) & \text{otherwise}
    \end{cases} \\
    &= \Z / 2\Z \quad \text{for all $n \ge 0$}
  \end{align*}
  since $\Ext(0; \Z / 2\Z) = \Hom(0; \Z / 2\Z) = 0$.
\end{example}

\begin{corollary}\label{cor:iso-cohom}
  If a chain map induces an isomorphism on
  all homology groups, then it induces an
  isomorphism on all cohomology groups.
\end{corollary}

\begin{proof}
  If $\alpha : (C_*, \partial) \to (C_*', \partial')$
  induces isomorphisms on all homology groups,
  then
  \begin{center}
    \begin{tikzcd}
      0 \ar[r] & \Ext(H_{n - 1}(C_*), G) \ar[r] & H^n(C_*; G) \ar[r] & \Hom(H_n(C_*), G) \ar[r] & 0 \\
      0 \ar[r] & \Ext(H_{n - 1}(C_*'), G) \ar[u, swap, "\alpha_*"] \ar[r] & H^n(C_*'; G) \ar[r] \ar[u, swap, "\alpha_*"] & \Hom(H_n(C_*'), G) \ar[u, swap, "\alpha_*"] \ar[r] & 0
    \end{tikzcd}
  \end{center}
  The $\alpha^*$ on the ends are isomorphisms, so
  the $\alpha^*$ in the middle is also an
  isomorphism.
\end{proof}
