\chapter{Mar.~24 --- Degree}

\section{Degree}

\begin{definition}
  Given a map $f : S^n \to S^n$, we get an induced
  map $f_* : \widetilde{H}_n(S^n) \to \widetilde{H}_n(S^n)$ (recall that $\widetilde{H}_n(S^n) \cong \Z$).
  Then the \emph{degree} of $f$, denoted
  $\deg f$, is $f_*(1) \in \Z$.
\end{definition}

\begin{remark}
  Note the following:
  \begin{enumerate}
    \item $\deg (\id_{S^n}) = 1$.
    \item $\deg f$ only depends on $f$ up to homotopy
      (since $f_* = g_*$ if $f \simeq g$).
    \item If $f$ is not surjective, then $\deg f = 0$.

      To see this, note that if
      $f$ is not surjective, then there exists
      $p \in S^n$ such that $p \notin \im f$. Then
      \begin{center}
        \begin{tikzcd}
          S^n \ar[r, "f"] \ar[dr, swap, "\widetilde{f}"] & S^n \\
          & S^n \setminus \{f(p)\} \ar[u, hook, swap, "i"]
        \end{tikzcd}
        \quad
        \begin{tikzcd}
          \widetilde{H}_n(S^n) \ar[r, "f_*"] \ar[dr, swap, "\widetilde{f}_*"] & \widetilde{H}_n(S^n) \\
                                                                   & \widetilde{H}_n(S^n \setminus \{f(p)\}) \ar[u, hook, swap, "i_*"]
        \end{tikzcd}
      \end{center}
      Since $\widetilde{H}_n(S^n \setminus \{f(p)\}) = 0$, we have
      $f_*(1) = i_*(\widetilde{f}_*(1)) = 0$.
    \item If $f$ is a reflection (e.g.
      $f(x_0, \dots, x_n) = (-x_0, x_1, \dots, x_n)$, then $\deg f = -1$.

      To see this, we induct on $n$. For
      $n = 0$, we have $S^0 = \{\pm 1\}$ and
      \[
        H_0(S^0) = H_0(\{1\}) \oplus H_0(\{-1\}) = \Z \oplus \Z,
      \]
      where $f_*(a, b) = (b, a)$. Recall that the
      reduced homology is computed with
      \begin{center}
        \begin{tikzcd}
          C_1(S^0) \ar[r, "\partial_0"] & C_0(S^0) \ar[r, "\varepsilon"] & \Z
        \end{tikzcd}
      \end{center}
      where $\varepsilon$ is given by
      $\sum m_i x_i \mapsto \sum m_i$. Then we
      defined $\widetilde{H}_0(S^0) = \ker \varepsilon / {\im \partial_0}$.
      Note that we have
      \begin{align*}
        \varepsilon : \Z \oplus \Z &\longrightarrow \Z \\
        (a, b) &\longmapsto a + b
      \end{align*}
      so $\ker \varepsilon$ (hence also
      $\widetilde{H}_0(S^0) \cong \Z$) is generated by
      $(1, -1)$. Thus
      $f_*(1, -1) = (-1, 1) = -(1, -1)$, so
      $\deg f = -1$. Now assume the
      result is true for $S^k$ with $k < n$. Let
      \[
        D^n_{\pm} = \{(x_0, \dots, x_n) \in S^n : \pm x_n \ge 0\}
      \]
      and note that $f$ preserves $D^n_{\pm}$.
      Since $(S^n, D^n_+)$ is a good pair,
      $\widetilde{H}_n(S^n, D^n_+) \cong \widetilde{H}_n(S^n / D^n_+)$,
      and
      \begin{center}
        \begin{tikzcd}
          \widetilde{H}_n(S^n) \ar[r, "\cong"] \ar[d, "f_*"] &
          H_n(S^n, D^n_+) \ar[r, "\cong"] \ar[d, "f_*"] & H_n(D^n_-, \partial D^n_-) \ar[r, dashed, "\cong"] \ar[d, "f_*"] & H_{n - 1}(\partial D^n_-) \ar[d, "f_*"] \\
          \widetilde{H}_n(S^n) \ar[r, "\cong"] &
          H_n(S^n, D^n_+) \ar[r, "\cong"] & H_n(D^n_-, \partial D^n_-) \ar[r, dashed, "\cong"] & H_{n - 1}(\partial D^n_-)
        \end{tikzcd}
      \end{center}
      where $\partial D^n_- \cong S^{n - 1}$.
      To see that $H_n(D^n_-, \partial D^n_-) \cong H_{n - 1}(\partial D^n_-)$, note that for $n > 1$,
      \begin{center}
        \begin{tikzcd}[row sep=small]
          H_n(D^n_-) \ar[r] & H_n(D^n_-, \partial D^n_-) \ar[r, "\partial"] & H_{n - 1}(\partial D^n_-) \ar[r] & H_{n - 1}(D^n_-) \\
          0 \ar[u, equal] & & & 0 \ar[u, equal]
        \end{tikzcd}
      \end{center}
      is exact, so $\partial$ is an isomorphism.
      Check the $n = 1$ case as an exercise.

      This completes the proof by induction since
      we already know the result for $\partial D^n_- \cong S^{n - 1}$.
    \item If $f$ is the antipodal map (i.e.
      $f(x_0, \dots, x_n) = (-x_0, \dots, -x_n)$), then $\deg f = (-1)^{n + 1}$.

      This follows from the previous result, since the
      antipodal map is a composition of reflections.
  \end{enumerate}
\end{remark}

\begin{lemma}
  Let $f, g : X \to S^n \subseteq \R^{n + 1}$.
  If $f(x) \ne -g(x)$ for all$ x \in X$, then $f \simeq g$.
\end{lemma}

\begin{proof}
  Define the homotopy $H : X \times [0, 1] \to S^n$ by
  \[
    (x, t) \longmapsto \frac{t f(x) + (1 - t) g(x)}{\|t f(x) + (1 - t) g(x)\|},
  \]
  which is well-defined since $g(x) \ne -f(x)$.
\end{proof}

\begin{corollary}
  Let $f : S^n \to S^n$. Then
  \begin{enumerate}
    \item if $f$ has no fixed points, then $\deg f = (-1)^{n + 1}$;
    \item if there is no point $x \in S^n$ such that
      $f(x) = -x$, then $\deg f = 1$.
  \end{enumerate}
\end{corollary}

\begin{proof}
  Apply the lemma to $f$ and the antipodal map
  (resp. identity map).
\end{proof}

\begin{corollary}
  If $n$ is even, then any map $f : S^n \to S^n$
  has a fixed point or an antipodal point
  (i.e. $f(x) = -x)$.
\end{corollary}

\begin{proof}
  If $f$ has neither, then $1 = \deg f = -1$, a
  contradiction.
\end{proof}

\begin{corollary}
  The sphere $S^n$ has a non-zero vector field if
  and only if $n$ is odd.
\end{corollary}

\begin{proof}
  $(\Rightarrow)$ We prove the contrapositive.
  If $n$ is even,
  consider a vector field $v$, so $v(x)$ is a
  vector in $\langle x \rangle^\perp$ for each
  $x \in S^n$. If $v(x) \ne 0$ for all $x$, then
  $f : S^n \to S^n$ given by
  \[
    v(x) \mapsto \frac{v(x)}{\|v(x)\|}
  \]
  is a function with no fixed points or
  antipodal points, a contradiction.

  $(\Leftarrow)$ If $n$ is odd, then
  $v(x_0, x_1, \dots, x_{n - 1}, x_n) = (x_1, -x_0, \dots, x_n, -x_{n - 1})$ is a non-zero vector field.
\end{proof}

\begin{remark}
  The case $n = 2$ is the \emph{hairy ball theorem}.
\end{remark}

\begin{remark}
  It is true that $f, g : S^n \to S^n$ are homotopic
  if and only if $\deg f = \deg g$.
\end{remark}

\section{Computing the Degree}

\begin{definition}
  Let $n > 0$ and $f : S^n \to S^n$.
  Take a point $y \in S^n$ such
  that $f^{-1}(\{y\})$ consists of finitely many
  points $x_1, \dots, x_k$. Then for $n \ne 1$ (check
  $n = 1$ as an exercise), we
  have the exact sequence
  \begin{center}
    \begin{tikzcd}[row sep=small]
      H_n(S^n - \{y\}) \ar[r] & H_n(S^n) \ar[r, "i_*"] & H_n(S^n, S^n - \{y\}) \ar[r] & H_{n - 1}(S^n - \{y\}) \\
      0 \ar[u, equal] & & & 0 \ar[u, equal]
    \end{tikzcd}
  \end{center}
  So $i_*$ is an isomorphism $H_n(S^n) \to H_n(S^n, S^n - \{y\})$, and similarly
  $j_* : H_n(S^n) \to H_n(S^n, S^n - \{x_i\})$
  is also an isomorphism. Let $V$ be a neighborhood
  of $y$ and $U_i$ be a neighborhood of $x_i$ such
  that $f(U_i) \subseteq V$ and $x_j \notin U_i$
  for $i \ne j$. Then by excision, we have an
  isomorphism
  \[
    H_n(S^n)
    \cong H_n(S^n, S^n - \{y\})
    \cong H_n(S^n - (S^n - V), (S^n - \{y\}) - (S^n - V))
    \cong H_n(V, V - \{y\})
  \]
  and similarly for $H_n(U_i, U_i - \{x_i\})$. Now
  $f$ gives a map $f_* : H_n(U_i, U_i - \{x_i\}) \to H_n(V, V - \{y\})$, which we can view
  as a map $f_* : \Z \to \Z$. Define the
  \emph{local degree of $f$ at $x_i$} to be
  $\deg (f, x_i) = f_*(1) \in \Z$ (choose the generators
  using the isomorphism with $H_n(S^n)$).
\end{definition}

\begin{remark}
  Note the following:
  \begin{itemize}
    \item If we change $V$ or $U_i$, we get the same
      number (as long as $f(U_i) \subseteq V)$).
    \item If $f|_{U_i} : U_i \to f(U_i)$ is a
      homeomorphism, then take $V = f(U_i)$ and
      we see that
      \[
        (f|_{U_i})_* : H_n(U_i, U_i - \{x_i\}) \to H_n(V, V - \{y\})
      \]
      is an isomorphism,
      so if we view $(f|_{U_i})_*$ as a map
      $\Z \to \Z$, then $1 \mapsto \pm 1$.
      So $\deg(f, x_i) = \pm 1$.
  \end{itemize}
\end{remark}

\begin{lemma}
  With $f : S^n \to S^n$, $y$, $x_1, \dots, x_k$
  as above, $\deg f = \sum_{i = 1}^k \deg(f, x_i)$.
\end{lemma}

\begin{proof}
  Choose the $U_i$ to be disjoint and set
  $Z = S^n - \bigcup_{i = 1}^k U_i$. Then by
  excision,
  \begin{align*}
    H_n(S^n, S^n - f^{-1}(\{y\}))
    &= H_n(S^n, S^n - \{x_1, \dots, x_k\})
    \cong H_n(S^n - Z, (S^n - \{x_1, \dots, x_k\}) - Z) \\
    &= H_n\left(\bigcup_{i = 1}^k U_i,\, \bigcup_{i = 1}^k (U_i - \{x_i\})\right)
    \cong \bigoplus_{i = 1}^k H_n(U_i, U_i - \{x_i\}).
  \end{align*}
  So we have the diagram (let $g$ be the composition
  map $H_n(S^n) \to \bigoplus_{i = 1}^k H_n(U_i, U_i - \{x_i\})$)
  \begin{center}
    \begin{tikzcd}
      H_n(S^n) \ar[r, "f_*"] \ar[d, "i_*"] & H_n(S^n) \ar[d, "i_*"] \ar[d, swap, "\cong"] \\
      H_n(S^n, S^n - f^{-1}(\{y\})) \ar[d, "\cong"] & H_n(S^n, S^n - \{y\}) \ar[d, "\cong"] \\
      \bigoplus_{i = 1}^k H_n(U_i, U_i - \{x_i\}) \ar[r, swap, "\bigoplus (f|_{U_i})_*"] & H_n(V, V - \{y\})
    \end{tikzcd}
  \end{center}
  Note that $H_n(S^n) \to H_n(U_i, U_i - \{x_i\})$
  (viewed as $\Z \to \Z$) sends $1 \mapsto 1$,
  so $g(1) = (1, 1, \dots, 1)$ and
  \[
    \deg f = f_*(1) = \left(\bigoplus (f|_{U_i})_*\right)
    \circ g_*(1)
    = \bigoplus(f|_{U_i})_*(1)
    = \sum_{i = 1}^k \deg(f, x_i),
  \]
  which is the desired result.
\end{proof}

\begin{example}
  Consider the following:
  \begin{enumerate}
    \item Let $n > 0$ and $f_n : S^1 \to S^1$
      be the map $z \mapsto z^n$, where $S^1$ is viewed
      as the unit circle in $\C$. Let $y = 1$ and
      pick $x_1, \dots, x_n$ evenly spaced around the
      circle with $x_1 = y$. Choose the $U_i$
      such that $f|_{U_i} : U_i \to V$ is a
      homeomorphism, so $\deg(f_n, x_i) = \pm 1$.
      We can extend $f|_{U_i} : U_i \to V$ to a
      homeomorphism $g_i : S^1 \to S^1$. One can
      shows as an exercise that $g_i$ must be homotopic
      to $\id_{S^1}$. Since $n > 0$, we have
      \[
        1 = \deg g_i = \deg(g_i, x_i) = \deg(f, x_i),
      \]
      so all local degrees are $1$ and $\deg f = n$.
      Now if $n < 0$, let $f_n = f_{-n} \circ r$ for
      some reflection $r$, so
      \[\deg f_n = (\deg f_{-n}) (\deg r) = (-n) (-1) = n.\]
      Thus we see that $\deg f_n = n$ for all $n \ne 0$.
    \item Now let $D_1, \dots, D_k$ be disjoint disks
      in $S^n$ for $n > 1$. Let $C = S^n - \bigcup_{i = 1}^k D_i$, then
      \[
        X = S^n / C \cong \bigvee_{i = 1}^k S^n.
      \]
      Let $q : S^n \to X$ be the quotient map.
      Let $U = X - \{\text{wedge point}\}$ and
      $V$ be a neighborhood of the wedge point, and
      note that $U \cap V \cong \bigsqcup_{i = 1}^n (S^{n - 1} \times (0, 1))$.
      Mayer-Vietoris gives the exact sequence
      \begin{center}
        \begin{tikzcd}[row sep=small]
          H_n(U) \oplus H_n(V) \ar[r] & H_n(X) \ar[r] & H_{n - 1}(U \cap V) \ar[r] & H_{n - 1}(U) \oplus H_{n - 1}(V) \\
          0 \ar[u, equal] & & \bigoplus_{i = 1}^k \Z \ar[u, equal] & 0 \ar[u, equal]
        \end{tikzcd}
      \end{center}
      so we see that $H_n(X) \cong \bigoplus_{i = 1}^k \Z$.
      Now let $f_i : X \to S^n$ collapse all but the
      $i$th copy of $S^n$ to a point. Then we claim that
      $(f_i)_* : H_n(X) \to H_n(S^n)$ maps
      $(m_1, \dots, m_k) \mapsto m_i$. To see this,
      consider the map
      $j_i : S^n \to X$ which includes $S^n$ into the
      $i$th $S^n$ in $X$. Then
      \[
        (f_\ell \circ j_i)_*(1) = 0 \quad \text{if $i \ne \ell$},
      \]
      whereas $f_i \circ j_i$ is the identity on
      $S^n$, so $(f_i \circ j_i)_*(1) = 1$. So we
      see that
      \[
        (j_i)_*(1) = (0, \dots, 0, 1, 0, \dots, 1)
        = 0 \quad \text{and} \quad
        (f_i)_*(0, \dots, 0, 1, 0, \dots, 0) = 1,
      \]
      where the $1$ appears in the $i$th slot.
      Then let $f : X \to S^n$ be $f_n$ on the $i$th
      sphere, so $f(1, \dots, 1) = k$
      and we see that $(f_* \circ q_*)(1) = k$, i.e.
      the degree is $k$.
  \end{enumerate}
\end{example}
