\documentclass[12pt, letterpaper, oneside]{book}
\usepackage[margin={0.6in, 0.75in}]{geometry}
\usepackage{microtype}
% \usepackage{kpfonts}
\usepackage{amsmath, amssymb, amsthm}
\usepackage{parskip}
\usepackage[many]{tcolorbox}
\usepackage{footnote}
\usepackage{cancel}
\usepackage{titlesec}
\usepackage{pgffor}
\usepackage[shortlabels, inline]{enumitem}
\usepackage{hyperref}
\usepackage{tikz-cd}

\usepackage[overload]{textcase}

\renewcommand{\chaptername}{Lecture}
\newtheorem{axiom}{Axiom}[chapter]
\newtheorem{theorem}{Theorem}[chapter]
\newtheorem{prop}{Proposition}[chapter]
\newtheorem{corollary}{Corollary}[theorem]
\newtheorem{lemma}{Lemma}[chapter]
\theoremstyle{definition}
\newtheorem{definition}{Definition}[chapter]
\newtheorem{exercise}{Exercise}[chapter]
\newtheorem{example}{Example}[definition]
\newtheorem*{remark}{Remark}

\tcbset{sharp corners, breakable, enhanced, parbox=false}
\newtcolorbox{mybox}[3][]
{
  colframe = #2!150,
  colback  = #2!5,
  coltitle = #2!0!white,  
  title    = {#3},
  #1,
}

\titleformat{\chapter}[display]
    {\normalfont\huge\bfseries}{\chaptertitlename\ \thechapter}{20pt}{\Huge}
\titlespacing*{\chapter}{0pt}{0pt}{40pt}

\newcommand{\R}{\mathbb{R}}
\newcommand{\N}{\mathbb{N}}
\newcommand{\Z}{\mathbb{Z}}
\newcommand{\C}{\mathbb{C}}
\newcommand{\Q}{\mathbb{Q}}
\newcommand{\F}{\mathbb{F}}
\newcommand{\sphere}{\mathbb{S}}
\newcommand{\ZF}{\mathsf{ZF}}
\newcommand{\ZFC}{\mathsf{ZFC}}
\newcommand{\AC}{\mathsf{AC}}

\newcommand{\T}{\mathcal{T}}
\newcommand{\B}{\mathcal{B}}

\DeclareMathOperator{\Vol}{Vol}
\DeclareMathOperator{\Int}{int}
\DeclareMathOperator{\area}{area}
\DeclareMathOperator{\curl}{curl}
\DeclareMathOperator{\maps}{maps}
\DeclareMathOperator{\id}{id}
\DeclareMathOperator{\im}{im}
\DeclareMathOperator{\diam}{diam}
\DeclareMathOperator{\con}{\#}

\title{MATH 6441: Algebraic Topology I}
\author{Frank Qiang\\Instructor: John Etnyre}
\date{Georgia Institute of Technology\\Spring 2025}

\begin{document}
  \maketitle

  \begingroup
  \let\cleardoublepage\clearpage
  \tableofcontents
  \endgroup

  % \foreach \i in {00, 01, 02, 03, 04, ..., 50} {%
  %   \edef\FileName{lectures/lecture\i.tex}%     The % here are necessary to eliminate any
  %   \IfFileExists{\FileName}{%  spurious spaces that may get inserted
  %      \input{\FileName}%       at these points
  %   }
  % }
  \chapter{Jan.~6 --- CW-Complexes}

\section{Introduction and Motivation}
Algebraic topology builds ``functions'' (actually
\emph{functors})
\[
  \{
    \text{topological spaces, continuous maps}
  \}
  \longrightarrow
  \{
    \text{algebraic things, algebraic maps}
  \},
\]
where ``algebraic things'' can be groups, vector
spaces, etc. The main objective of algebraic topology is
to \emph{distinguish topological spaces}, e.g.
showing that $\R^n \ncong \R^m$ for $n \ne m$.
More applications are:
\begin{enumerate}
  \item Studying maps between spaces.\footnote{All maps and functions in this class are continuous unless otherwise specified.}
    \begin{itemize}
      \item Does a given space $M$ embed in $N$? For
        instance, for what $m$ does $\R P^n$ embed
        in $\R^m$? (This is still not known in general.)
        Here $\R P^n$ is the real projective space.
      \item Lifting maps, i.e. given
        $f : A \to B$ and $g : E \to B$, does
        there exist a map $\widetilde{f} : A \to E$
        such that $g \circ \widetilde{f} = f$?
        In other words, is there a map $\widetilde{f}$
        such that the following diagram commutes:
        \begin{center}
        \begin{tikzcd}
          & E \arrow[d, "g"] \\
          A \arrow[ur, "\widetilde{f}", dashed] \arrow[r, "f", swap] & B
        \end{tikzcd}
        \end{center}
      \item Fixed point problems: Given $f : X \to X$,
        does $f$ have a fixed point, i.e. $x \in X$
        such that $f(x) = x$? Such theorems are used
        to prove the existence of solutions to
        ordinary differential equations, for instance.
    \end{itemize}
  \item Group actions, e.g. which finite groups act
    freely on $S^n$?
  \item Group theory, e.g. showing that every subgroup of a free
    group is free.
    Another example is that if $F_n$ is
    the free group on $n$ generators, then
    its \emph{commutator} $[F_n, F_n]$ is not
    finitely generated.
  \item Algebra, e.g. proving the fundamental theorem
    of algebra.
\end{enumerate}

This course will cover the following topics:
\begin{enumerate}
  \item The \emph{fundamental group}
    $\pi_1(X, x_0)$ of a space $X$ for $x_0 \in X$,
    and \emph{covering spaces}.
  \item The \emph{homology groups} $H_k(X)$ for
    $k = 0, 1, 2, \dots$. These groups are abelian.
  \item The \emph{cohomology ring}
    $H^*(X) = \bigoplus_{k = 0}^\infty H^k(X)$.
\end{enumerate}
But before getting to this, we need to develop some
important ideas.

\section{CW-Complexes}

\begin{definition}
  Let $D^n \subseteq \R^n$ be the unit disk and
  $S^{n - 1} = \partial D^n$. Given a topological
  space $Y$ and a continuous map $a : S^{n - 1} \to Y$,
  the space obtained from $Y$ by \emph{attaching} an
  \emph{$n$-cell} (\emph{via $a$}) is
  \[
    Y \cup_a D^n = (Y \sqcup D^n) / {\sim},
  \]
  where the equivalence relation $\sim$ is given by
  $x \sim a(x)$ for $x \in \partial D^n$.\footnote{Here $A \sqcup B$ denotes the \emph{disjoint union} of A and B.}
\end{definition}

\begin{definition}
  An \emph{$n$-complex} or an \emph{$n$-dimensional CW-complex}
  is defined inductively by:
  \begin{itemize}
    \item A $(-1)$-complex is the empty set $\varnothing$.
    \item An $n$-complex $X^n$ is a space obtained
      from an $(n - 1)$-complex by attaching $n$-cells.
  \end{itemize}
  An $n$-complex is \emph{finite} if it involves
  only a finite number of cells. The \emph{$k$-skeleton}
  of $X$ is the union of all $n$-cells in $X$
  with $n \le k$.
\end{definition}

\begin{remark}
  Any CW-complex is Hausdorff. See Hatcher for a proof.
\end{remark}

\begin{example}
  Here are some examples of CW-complexes:
  \begin{itemize}
    \item A $0$-complex is a union of points.
      This is because $D^0 = \{\text{pt}\}$ and
      $\partial D^0 = \varnothing$.
    \item A $1$-complex is a graph (points and lines
      connecting them).
    \item The torus $T$ (a square with opposite sides
      identified) is a $2$-complex. Here
      the $0$-skeleton $T^{(0)}$ is the common corner
      on the square and the $1$-skeleton $T^{(1)}$
      is two sides of the square after taking the
      quotient. The $2$-skeleton $T = T^{(2)}$ is the
      entire torus.
    \item Another example of a $2$-complex is the
      two-holed torus, which is obtained by identifying
      the edges of an octagon (pairs of every other
      edge identified with opposite orientation).
    \item A third example of a $2$-complex is
      $X^{(1)} \cup_a D^2$ given an attaching map
      $D^2 \to X^{(1)}$.
    \item Consider the unit sphere $S^n \subseteq \R^{n + 1}$.
      One way to give $S^2$ a CW-complex structure is
      to see the sphere as two disks $D^2$ glued together,
      resulting in one $0$-cell, one $1$-cell, and two
      $2$-cells. Another way is to start with
      two points, attach a $1$-cell to get a circle,
      and then attaching two disks to get $S^2$. This
      results in two $0$-cell, two $1$-cell, and two
      $2$-cells.

      The second idea generalizes to $S^n$. We can
      write
      \[
        S^n = S^{n - 1} \cup_{a_1} D^n \cup_{a_2} D^n,
      \]
      where $S^{n - 1}$ inductively has a CW-complex
      structure. This yields two $k$-cells for each
      $k \le n$.

      Another way to put a CW-complex structure
      on $S^n$ is to attach $D^n$ to a point with
      $\partial D^n \to \{\text{pt}\}$.
      In particular, notice that a space
      can in general have several different CW-complex
      structures.
    \item Consider the \emph{$n$-dimensional real projective space}
      \[
        \R P^n = \{\text{lines through the origin in } \R^{n + 1}\}.
      \]
      Since each line through the origin passes through
      $S^n$ twice, we can equivalently think of $\R P^n$
      as the unit sphere $S^n$ with antipodal points
      identified.

      We can also think of this as $D^n$ with
      antipodal points on $\partial D^n$ identified.
      Since $\partial D^n = S^{n - 1}$, this is
      simply $\R P^{n - 1} \cup_{a} D^n$, where
      $a : \partial D^n \to \R P^{n - 1}$ is the
      quotient map. This gives $\R P^n$ a CW-complex
      structure with one $k$-cell for each $k \le n$.
    \item The complex projective space
      $\C P^n$ has a similar CW-complex structure with
      one $k$-cell for each even $k \le 2n$.
      One can verify this as an exercise.
    \item Any smooth manifold has a CW-complex
      structure. See Hatcher.
  \end{itemize}
\end{example}

\begin{exercise}
  Show the product of CW-complexes is a CW-complex.
\end{exercise}

\begin{definition}
  A \emph{subcomplex} of a CW-complex $X$ is a closed
  subset $A \subseteq X$ that is a union of cells in $X$.
  In particular, $A$ is also a CW-complex and
  $(X, A)$ is called a \emph{CW-pair}.
\end{definition}

\section{Homotopy}

\begin{definition}
  Let $X$ and $Y$ be topological spaces. Two maps
  $f, g : X \to Y$ are \emph{homotopic}, denoted
  $f \sim g$, if there exists a continuous map
  $\Phi : X \times [0, 1] \to Y$ such that
  \[
    \Phi(x, 0) = f(x) \quad \text{and} \quad \Phi(x, 1) = g(x).
  \]
  In this case, $\Phi$ is called a \emph{homotopy}
  between $f$ and $g$.
\end{definition}

\begin{remark}
  We note the following:
  \begin{itemize}
    \item A homotopy $\Phi$ gives a family of maps
      $\Phi_{t_0} : X \to Y$ given by
      $x \mapsto \Phi(x, t_0)$ which is continuous
      in $t_0$. So maps are homotopic if there is a
      continuous family of maps between them.
    \item If $A \subseteq X$ then we say that
      $f$ is \emph{homotopic to $g$ rel $A$},
      denoted $f \sim_A g$, if there exists
      $\Phi$ as above with the additional property
      that $\Phi(x, t) = f(x)$ for all $x \in A$,
      i.e. points in $A$ are fixed.
    \item If $A \subseteq X$ and $B \subseteq Y$,
      then the notation $f : (X, A) \to (X, B)$
      means that $f : X \to Y$ and
      $f(a) \in B$ for each $a \in A$.
      We say that $f$ is a \emph{map of pairs}.
      If $f, g : (X, A) \to (Y, B)$, then $f, g$
      are homotopic as maps of pairs if each
      $\Phi_t$ is a map of pairs.
  \end{itemize}
\end{remark}

  \chapter{Jan.~8 --- Homotopy}

\section{More on Homotopy}

\begin{example}
  For any space $X$, any map $f : X \to [0, 1]$ is
  homotopic to the map $g : X \to [0, 1]$ given by
  $x \mapsto 0$. To see this, we have the homotopy
  $\Phi : X \times [0, 1] \times [0, 1]$ defined by
  \[
    (x, t) \mapsto (1 - t) f(x).
  \]
  We can see that $\Phi(x, 0) = f(x)$ and
  $\Phi(x, 1) = 0 = g(x)$.
\end{example}

\begin{exercise}
  Show that homotopy is an equivalence relation on
  maps $X \to Y$.
\end{exercise}

\begin{definition}
  Let $C(X, Y) = \{\text{continuous maps from $X$ to $Y$}\}$.
  Let $[X, Y] = C(X, Y) / {\sim}$, i.e. homotopic
  maps are identified with each other.
\end{definition}

\begin{example}
  We have the following:
  \begin{enumerate}
    \item $[X, [0, 1]] = \{g\}$ for any space $X$, where
      $g$ is the map $x \mapsto 0$ as above.
    \item $[\{*\}, X] = \{\text{path components of $X$}\}$.
  \end{enumerate}
\end{example}

\section{Homotopy Groups}
\begin{definition}
  We call a space $X$ \emph{pointed} if there
  is a designated ``base point'' $x_0 \in X$.
  Given two pointed spaces $(X, x_0)$ and
  $(Y, y_0)$, we define
  \[
    [X, Y]_0 = \{\text{homotopy classes of maps of pairs
    $(X, \{x_0\}) \to (Y, \{y_0\})$}\}.
  \]
\end{definition}

\begin{definition}
  Let $y_0$ be the north pole in $S^n$, i.e.
  $S^n \subseteq \R^{n + 1}$ is the unit sphere and
  $y_0 = (0, \dots, 0, 1)$. The
  \emph{$n$th homotopy group}
  of a pointed space $(X, x_0)$ is
  $\pi_n(X, x_0) = [S^n, X]_0$.
\end{definition}

\begin{remark}
  The homotopy group $\pi_n(X, x_0)$ is in fact a
  group. We will study $\pi_1(X, x_0)$ next and it is
  called the \emph{fundamental group} of $(X, x_0)$.
\end{remark}

\begin{remark}
  For which $(Y, y_0)$ is $[Y, X]_0$ ``naturally'' a
  group for all $(X, x_0)$? Similarly, for which
  $(Y, y_0)$ is $[X, Y]_0$ a group for all $(X, x_0)$?
  Here, given a map $f : (X_1, x_1) \to (X_2, x_2)$,
  there is an obvious \emph{induced map}
  $f_* : [Y, X_1]_0 \to [Y, X_2]_0$ given by
  $[g] \mapsto [f \circ g]$. Similarly, there is a map
  $f^* : [X_2, Y]_0 \to [X_1, Y]_0$ given by
  $[g] \mapsto [g \circ f]$. In the questions above,
  ``naturally'' means that $f_*$ and $f^*$ are
  homomorphisms.
  for any $(X_1, x_1)$ and $(X_2, x_2)$.
  The (perhaps unsatisfying) answer is that
  a space satisfying the first condition is called
  an \emph{$H$-space}, and a space satisfying the second
  is called an \emph{$H'$-space}.
\end{remark}

\section{Homotopy Equivalence}
\begin{definition}
  We say that $f : X \to Y$ is the \emph{homotopy inverse}
  to a function $g : Y \to X$ if
  $f \circ g \sim \id_Y$ and $g \circ f \sim \id_X$,
  where $\id_X$ and $\id_Y$ are the identity maps
  on $X$ and $Y$.
  If $g$ has a homotopy inverse, then
  we call $g$ a \emph{homotopy equivalence} from $Y$ to
  $X$ and we call $X, Y$ \emph{homotopy equivalent}.\footnote{We will denote homotopy equivalence by $X \simeq Y$ or simply $X \sim Y$.}
\end{definition}

\begin{exercise}
  Show that homotopy equivalence is an equivalence
  relation.
\end{exercise}

\begin{lemma}
  The following are equivalent:
  \begin{enumerate}
    \item $X$ and $Y$ are homotopy equivalent.
    \item For any space $Z$, there is a one-to-one
      correspondence $\phi_Z : [X, Z] \to [Y, Z]$
      such that for all continuous maps
      $h : Z \to Z'$, the following diagram commutes:
      \begin{center}
        \begin{tikzcd}
          {[X, Z]} \ar[r, "\phi_Z"] \ar[d, "h_*"] & {[Y, Z]} \ar[d, "h_*"] \\
          {[X, Z']} \ar[r, "\phi_{Z'}"] & {[Y, Z']}
        \end{tikzcd}
      \end{center}
    \item For any space $Z$, there is a one-to-one
      correspondence $\phi^Z : [Z, X] \to [Z, Y]$
      such for that all continuous maps $h : Z \to Z'$,
      the following diagram commutes:
      \begin{center}
        \begin{tikzcd}
          {[Z', X]} \ar[r, "\phi^{Z'}"] \ar[d, "h^*"] & {[Z', Y]} \ar[d, "h^*"] \\
          {[Z, X]} \ar[r, "\phi^{Z}"] & {[Z, Y]}
        \end{tikzcd}
      \end{center}
  \end{enumerate}
\end{lemma}

\begin{proof}
  This is left as an exercise.
\end{proof}

\begin{remark}
  Based on the previous lemma, two spaces are
  homotopy equivalent if and only if homotopy classes
  of maps to and from the space are ``naturally
  equivalent.''
\end{remark}

\begin{example}
  We have the following:
  \begin{itemize}
    \item Homeomorphic spaces are homotopy equivalent.
    \item Let $X = S^1$ and $Y = S^1 \times [0, 1]$.
      We claim that $X$ is homotopy equivalent to $Y$.

      Define the maps $f : S^1 \to S^1 \times [0, 1]$ by
      $x \mapsto (x, 0)$ and $g : S^1 \times [0, 1] \to S^1$
      by $(x, t) \mapsto x$. Then we can see
      that $g \circ f : S^1 \to S^1$ maps
      $x \mapsto x$, so $g \circ f = \id_{S^1}$. On the
      other hand, the composition
      $f \circ g : S^1 \times [0, 1] \to
      S^1 \times [0, 1]$ maps $(x, t) \mapsto (x, 0)$.
      Now $f \circ g \sim \id_{S^1 \times [0, 1]}$ by
      homotopy. For instance, define
      $\Phi : (S^1 \times [0, 1]) \times [0, 1] \to
      (S^1 \times [0, 1])$ by
      $((x, t), s) \mapsto (x, st)$, so
      \[
        \Phi((x, t), 1) = (x, t) = \id_{S^1 \times [0, 1]}(x, t)
        \quad \text{and} \quad
        \Phi((x, t), 0) = (x, 0) = f \circ g.
      \]
      Thus $f$ is a homotopy equivalence from $S^1$ to
      $S^1 \times [0, 1]$. Note that
      $S^1 \times [0, 1]$ is the annulus.
  \end{itemize}
\end{example}

\begin{definition}
  A space is called \emph{contractible} if it is
  homotopy equivalent to a point.
\end{definition}

\begin{example}
  The spaces $[0, 1]$ and $\R^n$ are contractible
  (exercise).
\end{example}

\begin{definition}
  If $A \subseteq X$, then a \emph{retraction of $X$ to $A$}
  is a map $r : X \to A$ such that $r(a) = a$ for
  every $a \in A$. A \emph{deformation retraction} of
  $X$ to $A$ is a retraction $r : X \to A$
  that is homotopic rel $A$ to the identity map $\id_X$,
  i.e. we can find $\phi_t : X \to X$ for
  $t \in [0, 1]$ such that
  $\phi_0(x) = x$ and $\phi_1(X) \subseteq A$ and
  $\phi_t(x) = x$ for all $x \in A$ and $t \in [0, 1]$.
\end{definition}

\begin{remark}
  If $X$ deformation retracts to $A$, then $X$ is
  homotopy equivalent to $A$. To see this, suppose we
  have a homotopy $\phi_t : X \to X$ as above, and let
  $i : A \to X$ be the inclusion map. Then
  $\phi_1 \circ i = \id_A$ and $i \circ \phi_1 = \phi_1 \sim \phi_0 = \id_X$, so
  $\phi_1$ is a homotopy equivalence from $X$ to $A$.
\end{remark}

\begin{definition}
  Given two spaces $X, Y$ and a map $f : X \to Y$,
  the \emph{mapping cylinder} of $f$ is the space
  \[
    M_f = ((X \times [0, 1]) \cup Y) / {\sim},
  \]
  where the equivalence relation $\sim$ is given by
  by $(x, 1) \sim f(x)$ for $x \in X$.
\end{definition}

\begin{remark}
  The mapping cylinder $M_f$ deformation retracts to $Y$.
  To see this, consider the map $\widetilde{\phi}_t$
  given by
  $(x, s) \mapsto (x, (1 - t)s + t)$ on
  $X \times [0, 1]$ and $y \mapsto y$ on $Y$. Since
  $\widetilde{\phi}_t$ respects the equivalence
  relation, it descends to a map $\phi_t : M_f \to M_f$
  on the quotient space. Note that
  $\phi_0 = \id_{M_f}$ and $\phi_1(M_f) = Y \subseteq M_f$,
  and $\phi_t = \id_Y$ for all $t$. Thus $\phi_1$ is a
  deformation retraction. In particular,
  this means that $M_f \simeq Y$.
\end{remark}

\begin{remark}
  There are obvious inclusions
  $i : X \to M_f$ given by $x \mapsto (x, 0)$
  and $j : Y \to M_f$ given by $y \mapsto y$.
  Note that $\phi_1$ defined above is the homotopy
  inverse to $j$. Now we have the diagram
  \[
    \begin{tikzcd}
      X \ar[r, "f"] \ar[dr, "i"] & Y \ar[d, "j"] \\
      & M_f
    \end{tikzcd}
  \]
  where $j$ is a homotopy equivalence and
  $j \circ f \sim i$ (exercise).
\end{remark}

\begin{remark}
  The above remark shows the following ``slogan''
  of algebraic topology:
  \begin{quote}
    Any map is an inclusion up to homotopy.
  \end{quote}
\end{remark}

\begin{example}
  Let $X$ be three circles with two enclosed in a third
  bigger one, and let $Y$ be two circles enclosing the
  inner two circles of $X$
  connected by a line segment. Let $Z$ be the region
  inside by the outer circle of $X$ and outside
  the inner two circles of $X$.

  Define $f : X \to Y$
  to be the map which sends $x \in X$ to the point in
  $Y$ at the other end of an interval (points on the
  inner circles of $X$ are mapped by radial lines to the
  circles in $Y$, and points on the outer circle of $X$
  are mapped radially to either the circles or
  the line segment in $Y$). One can write an explicit
  formula for $f$ as an exercise.

  Then $Z$ is homeomorphic to $M_f$, and in particular
  $Z \simeq Y$. Similarly, $M_f$ is homotopy equivalent
  to two circles joined at a point, or a circle
  with a diameter. Thus by transitivity, these two spaces
  and $Z$ are all homotopy equivalent to each other.
\end{example}

  \chapter{Jan.~13 --- Homotopy, Part 2}

\section{More on Homotopy Equivalence}
\begin{lemma}\label{lem:lemma-2}
  If $(X, A)$ is a CW pair and $A$ is contractible,
  then $X / A \simeq X$.\footnote{Here $X / A$ denotes
  the quotient of $X$ obtained by collapsing all of $A$
  to a single point.}
\end{lemma}

\begin{exercise}
  The following are some applications of this lemma:
  \begin{enumerate}
    \item Let $X$ be a connected graph (i.e. a
      $1$-complex), and let $A$ be an edge in $X$
      connecting distinct vertices. Then $A$ is
      contractible, so $X / A \simeq X$.
      Continuing this process, let $A$ be a maximal
      tree in $X$, which will also be contractible.
      Then $X \simeq X / A$, so any connected graph is
      homotopy equivalent to a \emph{wedge of circles}
      (with number of circles equal to the number of
      self-loops in the graph).\footnote{A \emph{wedge} of pointed spaces $(X, x_0) \lor (Y, y_0)$ is the space obtained from $X \sqcup Y$ by identifying $x_0$ and $y_0$.}
    \item Consider the space $X$ obtained by attaching
      a $1$-cell $A$ connecting the north and south
      poles on a sphere. Let $B$ be half of a great
      circle
      connecting the endpoints of $A$. Clearly $A$ and
      $B$ are both contractible. After collapsing
      $A$, we see that $X \simeq X / A$, which is
      $S^2$ with the north and south poles identified.
      On the other hand, by contracting $B$ instead
      we see that $X \simeq X / B$, which is
      $S^2 \lor S^1$.

      Note that after contracting $A$,
      the subset $B$ is actually no longer contractible.
    \item Let $X$ be a torus with attached disks
      $A_1, A_2, A_3$ in the tube of the torus.
      Then
      \[X \simeq ((X / A_1) / A_2) / A_3,\]
      which is three spheres lying in a circle, each
      attached to the next one at a single point.

      We can also obtain this space by considering
      the space $Y$ of
      three spheres attached in a line with an extra
      $1$-cell $B$ attached at the ends of the chain
      of circles. Also let $A$ be the union of halves
      of great circles going through the chain of
      circles, with the same endpoints as $B$. Then
      we can see that this creates the same space as
      before, so that $X \simeq Y / B \simeq Y$.
      On the other hand, by contracting $A$, we see
      that $Y \simeq Y / A = S^2 \lor S^2 \lor S^2 \lor S^1$.

      Of course, all of these spaces are then homotopy
      equivalent to each other by transitivity.
  \end{enumerate}
\end{exercise}

\begin{lemma}\label{lem:lemma-3}
  Let $(X, A)$ be a CW pair and $f, g : A \to Y$ 
  be homotopic maps. Then $X \cup_f Y \simeq X \cup_g Y$.
\end{lemma}

\begin{example}
  Let $Y = S^2$, and $X = D^2$, and $A = \partial D^2$.
  Let $g : A \to Y$ map $A$ to a great circle, and
  let $f : A \to Y$ map $A$ to the north pole.
  One can show as an exercise that $f \sim g$ (e.g. by
  pulling the equator towards the north pole).
  So the lemma says that $X \cup_g Y$, which is a
  sphere with a disk glued along its equator, is homotopy
  equivalent to $X \cup_f Y = S^2 \lor S^2$.
\end{example}

\section{Homotopy Extension Property}
\begin{remark}
  To prove both of these lemmas, we need the
  \emph{homotopy extension property} (HEP).
\end{remark}

\begin{definition}
  A space $X$ and a subspace $A \subseteq X$ have the
  \emph{homotopy extension property} if given
  $F_0 : X \to Y$ (for any $Y$ and $F_0$) and a
  homotopy $f_t : A \to Y$ such that $f_0 = F_0|_A$,
  then there is a homotopy $F_t : X \to Y$ such that
  $F_t|_A = f_t$ for every $t$.
\end{definition}

\begin{lemma}\label{lem:lemma-4}
  A pair $(X, A)$ has the homotopy extension property
  if and only if
  \[
    (X \times \{0\}) \cup (A \times [0, 1])
  \]
  is a retract of $X \times [0, 1]$.
\end{lemma}

\begin{proof}
  $(\Leftarrow)$ We will assume that $A$ is closed
  (not necessarily but makes the proof easier, and almost
  all examples satisfy this). By assumption, we
  have a retraction
  \[
    r : (X \times [0, 1]) \to (X \times \{0\}) \cup (A \times [0, 1]).
  \]
  Given $F_0 : X \to Y$ and $f_t : A \to Y$
  such that $f_0 = F_0|_A$, we can define a map
  \[
    \widetilde{F} : (X \times \{0\}) \cup (A \times [0, 1]) \to Y
  \]
  by $x \mapsto F_0(x)$ on $X \times \{0\}$ and
  $(a, t) \mapsto f_t(a)$ on $A \times [0, 1]$.
  This map $\widetilde{F}$ is continuous since the definitions of
  $\widetilde{F}$ agree on the intersection and the
  intersection $A \times \{0\}$ is closed. Now define
  \[F : X \times [0, 1] \to Y\]
  by $F = \widetilde{F} \circ r$,
  which is a homotopy of $F_0$ that extends $f_t$.

  $(\Rightarrow)$ Let $Y = (X \times [0, 1]) \cup (A \times [0, 1])$. Let
  \[F_0 : X \to (X \times \{0\}) \cup (A \times [0, 1])\]
  be given by $x \mapsto (x, 0)$, and
  \[
    f_t : A \mapsto (X \times \{0\}) \cup (A \times [0, 1])
  \]
  be given by $a \mapsto (a, t)$. Then the homotopy
  extension property yields an extension
  \[F : X \times [0, 1] \to (X \times \{0\}) \cup (A \times [0, 1]),\]
  which is a retraction, as desired.
\end{proof}

\begin{lemma}\label{lem:lemma-5}
  If $(X, A)$ is a CW pair, then
  $(X \times \{0\}) \cup (A \times [0, 1])$ is a
  (deformation) retract of $X \times [0, 1]$.
  In particular, $(X, A)$ satisfies the homotopy
  extension property.
\end{lemma}

\begin{proof}
  The main idea is that for any disk $D^n$, the space
  $(D^n \times \{0\}) \cup (\partial D^n \times [0, 1])$
  is a deformation retract of $D^n \times [0, 1]$.
  To see this, let $D^n \subseteq \R^n$ be the unit
  disk and $D^n \times [0, 1] \subseteq \R^{n + 1}$.
  Let
  \[
    p = (0, \dots, 0, 2).
  \]
  For any $x \in D^n \times [0, 1]$, let
  $\ell_x$ be the line through $p$ and $x$. Note that
  \[
    \ell_x \cap ((D^n \times [0, 1] \cup (\partial D^n \times [0, 1]))
  \]
  is a unique point. Define $\widetilde{r}(x)$ to be this point,
  which yields a map
  \[\widetilde{r} : D^n \times [0, 1] \to (D^n \times \{0\}) \cup (\partial D^n \times [0, 1])\]
  Note that for $x \in (D^n \times \{0\}) \cup (\partial D^n \times [0, 1])$,
  then $\widetilde{r}(x) = x$ since the point of intersection is
  unique and $x$ is already in the intersection.
  Show as an exercise that $\widetilde{r}$ is continuous.
  Then setting
  \[
    \widetilde{r}_t = t \widetilde{r} + (1 - t) \id_{D^n \times [0, 1]}
  \]
  gives a deformation retraction
  of $D^n \times [0, 1]$ onto
  $(D^n \times \{0\}) \cup (\partial D^n \times [0, 1])$.

  To show the general case that
  $X \times [0, 1]$ retracts to
  $(X \times \{0\}) \cup (A \times [0, 1])$, we induct
  on the dimension of cells. Define
  $r$ on $X^{(0)} \times [0, 1] \to (X \times \{0\}) \cup (A \times [0, 1])$
  by the following:
  \begin{itemize}
    \item if a $0$-cell $D^0 \subseteq A$, then
      let $r$ be the identity on $D^0 \times [0, 1]$;
    \item if a $0$-cell $D^0$ is not in $A$, then
      send $D^0 \times [0, 1]$ to
      $D^0 \subseteq X \times \{0\}$.
  \end{itemize}
  Now inductively assume $r$ is defined on
  $X^{(k - 1)} \times [0, 1] \to (X \times \{0\}) \cup (A \times [0, 1])$.
  For each $k$-cell $D^k$,
  \begin{itemize}
    \item if $D^k \subseteq A$, then
      let $r$ be the identity on $D^k \times [0, 1]$;
    \item if $D^k$ is not in $A$, then note that
      $\partial D^k \times [0, 1] \to X^{(k - 1)} \times [0, 1]$
      is defined by induction, and
      we have an ``inclusion'' (here
      $a : \partial D^k \to X^{(k - 1)}$ is the attaching map for $D^k$)
      \begin{center}
        \begin{tikzcd}
          {D^k} \arrow[r, "i"] \arrow[rr, bend right=10, swap, "j"] & {X^{(k - 1)} \cup D^k}
          \arrow[r, "q"] & {(X^{(k - 1)} \cup D^k) / \{(x \in \partial D^k) \sim (a(x) \in X^{k - 1})\}}
        \end{tikzcd}
      \end{center}
    So we let $D^k \times \{0\} \to (X \times \{0\}) \cup (A \times [0, 1])$
    be the map $j$ into $X \times \{0\}$. This
    defines $r$ on
    \[
      (D^k \times [0, 1]) \cup (\partial D^k \times [0, 1]),
    \]
    which extends to $D^k \times [0, 1]$ by composing
    with the map $\widetilde{r}$ from above.
  \end{itemize}
  This inductively defines the retraction
  $r$ on all of $X \times [0, 1]$.
  The last claim follows by Lemma
  \ref{lem:lemma-4}.
\end{proof}

\begin{remark}
  Now we can prove the first two lemmas from the
  beginning of the day.
\end{remark}

\begin{proof}[Proof of Lemma \ref{lem:lemma-2}]
  We prove that the result holds for any $(X, A)$
  which satisfies the homotopy exttension property.
  We show that the quotient map $q : X \to X / A$
  has a homotopy inverse. Since $A$ is contractible,
  we know there exists a homotopy $f_t : A \to A \subseteq X$,
  such that $f_0 = \id_A$ and $f_1$ is constant.
  Let $F_0 : X \to X$ be the identity (note that
  $F_0|_A = \id_A$), so that the homotopy extension
  property gives a homotopy
  $F_t : X \to X$ extending $f_t$.
  Since $F_t(A) \subseteq A$ for every $t$, we get maps
  \[
    \widetilde{F}_t : X / A \to X / A,
  \]
  which are well defined since points in $A$ are mapped
  to points in $A$. Furthermore, $F_1(A) = \{\text{pt}\}$,
  so $F_1$ factors through $X / A$ to give a map
  $h : X / A \to X$ satisfying $F_1 = h \circ q$. This
  gives the diagram:
  \begin{center}
    \begin{tikzcd}
      X \arrow[d, swap, "q"] \arrow[r, "F_1"] & X \arrow[d, "q"] \\
      X / A \arrow [ru, "h"] \arrow[r, swap, "\overline{F}_1"] & X / A
    \end{tikzcd}
  \end{center}
  It is easy to check that $h \circ q = F_1$ and
  $q \circ h = \overline{F}_1$, so that the
  diagram commutes. But then
  \[
    h \circ q = F_1 \sim F_0 = \id_X \quad \text{and} \quad
    q \circ h = \overline{F}_1 \sim \overline{F}_0 = \id_{X / A},
  \]
  so $q$ is a homotopy equivalence.
\end{proof}

\begin{proof}[Proof of Lemma \ref{lem:lemma-3}]
  Let $F : A \times [0, 1] \to Y$ be a homotopy,
  which extends to $F : X \times [0, 1] \to Y$
  by the homotopy extension property. Consider
  the mapping cylinder
  \[
    M_F = (X \times [0, 1]) \cup_F Y,
  \]
  and one can show that
  $M_F \simeq X \cup_g Y \simeq X \cup_f Y$.
\end{proof}

  \chapter{Jan.~15 --- Fundamental Group}

\section{Fundamental Group}

\begin{remark}
  The basic idea of the \emph{fundamental group} is to
  study the topology of a space with loops mapped into
  the space. For instance, intuitively, any loop
  in $S^2$ can be ``pulled back'' to (i.e. is homotopic
  to) a constant loop. On the other hand, a loop
  wrapping around the hole in $T^2$ gets stuck and
  cannot be ``pulled back'' to a constant loop. The
  same issue happens for a loop in $T^2$ around the
  cylindrical part.
  The fundamental group is a way to make this intuition
  precise and measure the ``holes'' in a space.
\end{remark}

\begin{definition}
  The \emph{fundamental group} of a based space
  $(X, x_0)$ is
  \[
    \pi_1(X, x_0) = [(S^1, n), (X, x_0)]_0,
  \]
  i.e. the homotopy classes of loops in $X$ based
  at $x_0$. Here $n = (0, 1)$ is the north pole of
  $S^1$.
\end{definition}

\begin{exercise}
  Let $S^1 \subseteq \R^2$ be the unit circle and
  let $p : [0, 1] \to S^1$ be given by
  \[
    t \mapsto (\cos 2\pi t, \sin 2\pi t).
  \]
  Show that $p$ is a quotient map, so we can think of
  $S^1$ as $[0, 1]$ with $0, 1$ identified. Moreover,
  show that there is a one-to-one correspondence
  between maps of the form\footnote{Such a loop $\gamma$ is called a \emph{based loop}.}
  \[
    \gamma : ([0, 1], \{0, 1\}) \to (X, x_0)
    \quad \text{and} \quad
    \widetilde{\gamma} : (S^1, \{(1, 0)\}) \to (X, x_0)
  \]
  given by $\widetilde{\gamma} \mapsto \widetilde{\gamma} \circ p = \gamma$,
  and that homotopies of $\widetilde{\gamma}$
  rel $\{0, 1\}$ correspond to homotopies of $S^1$
  rel $\{(1, 0)\}$.
\end{exercise}

\begin{remark}
  Using the above exercise, we can think of
  $\pi_1(X, x_0) = [S^1, X]_0$ instead as
  \[
    \pi_1(X, x_0) = [([0, 1], \{0, 1\}), (X, x_0)]_0.
  \]
  Given a based loop $\gamma : [0, 1] \to X$,
  we denote its equivalence class in $\pi_1(X, x_0)$
  by $[\gamma]$.
\end{remark}

\begin{definition}
  If $\gamma_1, \gamma_2$ are loops in $X$ based
  at $x_0 \in X$, their \emph{concatenation}
  $\gamma_1 * \gamma_2 : [0, 1] \to X$ is
  \[
    t \mapsto
    \begin{cases}
      \gamma_1(2t) & \text{if } 0 \le t \le 1 / 2, \\
      \gamma_2(2t - 1) & \text{if } 1 / 2 \le t \le 1.
    \end{cases}
  \]
\end{definition}

\begin{remark}
  Concatenation of loops indeed yields another loop
  since $\gamma_1 * \gamma_2(0) = \gamma_1 * \gamma_2(1) = x_0$
  and $\gamma_1 * \gamma_2$ is continuous since
  the definitions agree on the closed
  set $\{1 / 2\}$.
\end{remark}

\begin{remark}
  We can clearly see that $\gamma_1 * \gamma_2$ is
  well-defined given $\gamma_1$ and $\gamma_2$,
  but can we define $[\gamma_1] * [\gamma_2]$
  for homotopy classes of loops in a well-defined
  manner? We need to check
  that if $\gamma_1 \sim \gamma_2$ and $\delta_1 \sim \delta_2$
  (i.e. $\gamma_1, \gamma_2 \in [\gamma_1]$ and
  $\delta_1, \delta_2 \in [\delta_1]$), then we
  also have
  $\gamma_1 * \delta_1 \sim \gamma_2 * \delta_2$.

  To do this, let $H : [0, 1] \times [0, 1] \to X$
  be the homotopy from $\gamma_1$ to $\gamma_2$ and
  $G : [0, 1] \times [0, 1] \to X$ be the homotopy
  from $\delta_1$ to $\delta_2$. We need to construct
  a homotopy $\widetilde{H} : [0, 1] \times [0, 1] \to X$
  from $\gamma_1 * \delta_1$ to $\gamma_2 * \delta_2$.

  Note that if we think of $[0, 1] \times [0, 1]$
  as the unit square, then $\widetilde{H}$ is
  already defined on the boundary: the left and right
  sides are constantly $x_0$, the top side is
  $\gamma_2 * \delta_2$, and the bottom side is
  $\gamma_1 * \delta_1$. So we only need
  to define it on the interior.
  For this, note that the vertical line in
  the middle of the square is also constantly
  $x_0$ by construction: This creates two rectangles
  on each half, which we can fill with $H$ and $G$.

  More formally, we can construct the homotopy
  $\widetilde{H} : [0, 1] \times [0, 1] \to X$
  explicitly via
  \[
    (t, s) \mapsto
    \begin{cases}
      H(2t, s) & \text{if } 0 \le t \le 1 / 2, \\
      G(2t - 1, s) & \text{if } 1 / 2 \le t \le 1.
    \end{cases}
  \]
  This is continuous since the definitions agree
  on the closed set $\{t = 1 / 2\}$. Thus, setting
  \[
    [\gamma_1] * [\delta_1] = [\gamma_1 * \delta_1]
  \]
  gives a well-defined binary operation by the above
  discussion.
\end{remark}

\begin{lemma}
  The pair $(\pi_1(X, x_0), *)$ is
  a group.
\end{lemma}

\begin{proof}
  For the identity, let $e : [0, 1] \to X$ be the
  constant loop $t \mapsto x_0$. We will show that
  \[
    [e] * [\gamma] = [\gamma] = [\gamma] * [e].
  \]
  The picture is that $[0, 1] \times [0, 1]$ has
  $\gamma$ on the top side and $\gamma * e$ on the
  bottom. By drawing a line from the midpoint of
  the bottom side and the top-right corner, we see
  that we can fill the right portion with just
  $x_0$ and the left portion with $\gamma$.
  The equation of this line is $s = 2t - 1$, so
  $t = (s + 1) / 2$.
  Thus from the picture, we can write the explicit
  homotopy $H : [0, 1] \times [0, 1] \to X$ via
  \[
    H(t, s) =
    \begin{cases}
      \gamma(2 / (s + 1), t) & \text{if } 0 \le t \le (s + 1) / 2, \\
      x_0 & \text{if } (s + 1) / 2 \le t \le 1.
    \end{cases}
  \]
  One can use a similar construction to show that
  $[e] * [\gamma] = [\gamma]$, so that
  $[e]$ is an identity element.

  Now we show the existence of inverses. Given
  a loop $\gamma$, define
  $\overline{\gamma}$ via $\overline{\gamma}(t) = \gamma(1 - t)$, i.e.
  $\gamma$ backwards. Set
  $\gamma_s(t) = \gamma(st)$. Note that
  as $t$ goes from $0$ to $1$, $\gamma_s$ goes
  from $\gamma(0)$ to $\gamma(s)$, and also that
  $\overline{\gamma}_s(t) = \gamma(s - st)$. So
  we can write the homotopy
  $H : [0, 1] \times [0, 1] \to X$ between
  $\gamma * \overline{\gamma}$ and $e$ by
  \[
    H(t, s) =
    \begin{cases}
      \gamma_s(2t) & \text{if } 0 \le t \le 1 / 2, \\
      \overline{\gamma}_s(2t - 1) & \text{if } 1 / 2 \le t \le 1
    \end{cases}
    =
    \begin{cases}
      \gamma(2st) & \text{if } 0 \le t \le 1 / 2, \\
      \gamma(s - s(2t - 1)) & \text{if } 1 / 2 \le t \le 1.
    \end{cases}
  \]
  Thus setting $[\gamma]^{-1} = [\overline{\gamma}]$
  gives us inverses.

  Finally, for associativity, we need to see that
  $(\gamma_1 * \gamma_2) * \gamma_3 \sim \gamma_1 * (\gamma_2 * \gamma_3)$. Again by drawing a picture, we
  see that we can draw two diagonal lines connecting the
  starting points of $\gamma_2$ on top and bottom and
  the ending points of $\gamma_1$ on top and bottom.
  Write an explicit formula for the homotopy
  as an exercise.
\end{proof}

\begin{remark}
  If $f : X \to Y$ and $x_0 \in X$, let
  $y_0 = f(x_0)$. Then given a based loop
  $\gamma : [0, 1] \to X$, note that the composition
  $f \circ \gamma : [0, 1] \to Y$ is a based loop
  in $Y$. Also, if $\gamma \sim \delta$, then
  $f \circ \gamma \sim f \circ \delta$ (if
  $H$ is a homotopy from $\gamma \sim \delta$, then
  $f \circ H$ is a homotopy from $f \circ \gamma$
  to $f \circ \delta$). In particular,
  $f$ induces a map
  \[f_* : \pi_1(X, x_0) \to \pi_1(Y, y_0).\]
\end{remark}

\begin{lemma}
  The induced map $f_* : \pi_1(X, x_0) \to \pi_1(Y, y_0)$ is a homomorphism.
\end{lemma}

\begin{proof}
  Note that
  \[
    \gamma_1 * \gamma_2(t) =
    \begin{cases}
      \gamma_1(2t) & \text{if } 0 \le t \le 1 / 2, \\
      \gamma_2(2t - 1) & \text{if } 1 / 2 \le t \le 1
    \end{cases}
  \]
  and that
  \[
    (f \circ \gamma_1) * (f \circ \gamma_2)(t) =
    \begin{cases}
      f(\gamma_1(2t)) & \text{if } 0 \le t \le 1 / 2, \\
      f(\gamma_2(2t - 1)) & \text{if } 1 / 2 \le t \le 1,
    \end{cases}
  \]
  so $f \circ (\gamma_1 * \gamma_2) = (f \circ \gamma_1) * (f \circ \gamma_2)$, which implies
  $f_*([\gamma_1 * \gamma_2]) = f_*([\gamma_1]) * f_*([\gamma_2])$.
\end{proof}

\begin{exercise}
  Check the following as an exercise:
  \begin{enumerate}
    \item $(f \circ g)_* = f_* \circ g_*$;
    \item if $f : X \to Y$ is homotopic rel
      base point to $g : X \to Y$, then
      $f_* = g_* : \pi_1(X, x_0) \to \pi_1(Y, y_0)$.
  \end{enumerate}
\end{exercise}

\begin{remark}
  How does $\pi_1$ depend on the base point? Let
  $x_0, x_1 \in X$, and suppose that there
  exists a path $h : [0, 1] \to X$ with
  $h(0) = x_0$ and $h(1) = x_1$. Then if $\gamma$ is a
  loop based at $x_1$, we can get a loop based at $x_0$
  by going from $x_0$ to $x_1$ along $h$, taking
  $\gamma$, and then going back to $x_0$. More
  explicitly, this is
  \[
    h * \gamma * \overline{h}(t) =
    \begin{cases}
      h(3t) & \text{if } 0 \le t \le 1 / 3, \\
      \gamma(3t - 1) & \text{if } 1 / 3 \le t \le 2 / 3, \\
      \overline{h}(3t - 2) & \text{if } 2 / 3 \le t \le 1.
    \end{cases}
  \]
\end{remark}

\begin{lemma}
  The path $h$ induces an isomorphism
  $\phi_h : \pi_1(X, x_1) \to \pi_1(X, x_0)$
  by $[\gamma] \mapsto [h * \gamma * \overline{h}]$.
\end{lemma}

\begin{proof}
  Check as an exercise that $\phi_h$ is a well-defined
  homomorphism.
  To show that $\phi_h$ is an isomorphism, we claim
  that $\phi_{\overline{h}}$ is an inverse of $\phi_h$.
  To see this, let $[\gamma] \in \pi_1(X, x_0)$. Then
  \[
    \phi_h \circ \phi_{\overline{h}}([\gamma])
    = [h * \overline{h} * \gamma * h * \overline{h}]
    = [h * \overline{h}] * [\gamma] * [h * \overline{h}]
    = [e] * [\gamma] * [e]
    = [\gamma],
  \]
  where the second equality follows by the same
  proof for associativity of $*$. This proves the result.
\end{proof}

\begin{remark}
  Note the following based on the above lemma:
  \begin{enumerate}
    \item The isomorphism class of $\pi_1(X, x_0)$
      only depends on the path component of $X$
      containing $x_0$.
    \item The isomorphism \emph{depends on} $h$.
      One needs to be careful about using
      the correct identification.
  \end{enumerate}
\end{remark}

\begin{theorem}
  If $f : X \to Y$ is a homotopy equivalence, then
  the induced map
  \[f_* : \pi_1(X, x_0) \to \pi_1(Y, y_0)\] is an
  isomorphism, where $y_0 = f(x_0)$.
\end{theorem}

\begin{proof}
  We will prove this next class.
\end{proof}

  \chapter{Jan.~22 --- Simple Computations}

\section{Fundamental Groups and Homotopy Equivalence}
\begin{lemma}
  Suppose $f_0, f_1 : X \to Y$ are homotopic by
  the homotopy $H : X \times [0, 1] \to Y$. Let
  $x_0 \in X$ be a basepoint and define
  $h : [0, 1] \to Y$ by $t \mapsto H(x_0, t)$. Then
  the following diagram commutes:
  \begin{center}
    \begin{tikzcd}
      \pi_1(X, x_0) \ar[r, "(f_0)_*"] \ar[dr, swap, "(f_1)_*"] & \pi_1(Y, f_0(x_0)) \\
      & \pi_1(Y, f_1(x_0)) \ar[u, swap, "\phi_h"]
    \end{tikzcd}
  \end{center}
\end{lemma}

\begin{proof}
  Fix an arbitrary $[\gamma] \in \pi_1(X, x_0)$, and we construct a
  homotopy from $h * (f_1 \circ \gamma) * \overline{h}$
  to $f_0 \circ \gamma$. The picture is the following:
  we have $h * (f_1 \circ \gamma) * \overline{h}$
  at the top and $f_0 \circ \gamma$ at the bottom,
  where $t$ parametrizes the horizontal direction and
  $s$ parametrizes the vertical direction.
  Draw a trapezoidal shape by connecting the
  middle two points on the top edge to the two bottom
  corners.

  Define the homotopy
  as follows: For a fixed $s$, define
  $H'(t, s)$ first by $h_s(t)$ in the first third, $H(\gamma(t), s)$ in the second third,
  and then $\overline{h}_s(t)$ in the last third.
  Construct the explicit homotopy as an exercise.
\end{proof}

\begin{theorem}
  If $f : X \to Y$ is a homotopy equivalence, then
  the induced map
  \[f_* : \pi_1(X, x_0) \to \pi_1(Y, f(x_0))\]
  on fundamental groups is an isomorphism.
\end{theorem}

\begin{proof}
  Let $g$ be a homotopy inverse to $f$, so we
  have the composition:
  \begin{center}
    \begin{tikzcd}
      \pi_1(X, x_0) \ar[r, "f_*"] & \pi_1(Y, f(x_0)) \ar[r, "g_*"] & \pi_1(X, g(f(x_0)))
    \end{tikzcd}
  \end{center}
  But we know $g \circ f = \id_X$, so by the lemma
  there exists a path $h : X \to X$ from $x_0$
  to $g(f(x_0))$ such that $g_* \circ f_* = \phi_h$
  is an isomorphism. So $f_*$ is injective.
  Similarly, $f_* \circ g_*$ is an isomorphism,
  and therefore $f_*$ is surjective. Since
  we already know $f_*$ is a homomorphism, this
  shows that $f_*$ is an isomorphism.
\end{proof}

\begin{remark}
  Recall that we have defined a ``functor''
  \[
    \{\text{pointed topological spaces, pointed maps}\}
    \to \{\text{groups, homomorphisms}\},
  \]
  where homotopy equivalent spaces are mapped to
  isomorphic
  groups and homotopic maps give rise to the ``same''
  homomorphism. We will finally make some
  computations of fundamental groups next.
\end{remark}

\section{Simple Computations of Fundamental Groups}

\begin{lemma}
  If $X$ is contractible, then
  $\pi_1(X, x_0) = \{1\}$ for all $x_0 \in X$, where
  $\{1\}$ is the trivial group.
\end{lemma}

\begin{proof}
  If $Y = \{y_0\}$ is a one-point space, then there
  exists a unique loop $\gamma : [0, 1] \to X$
  given by $t \mapsto y_0$. So $\pi_1(Y, y_0) = \{1\}$.
  Since $X$ is contractible, it is homotopy
  equivalent to $Y$ and so $\pi_1(X, x_0) = \{1\}$.
\end{proof}

\begin{definition}
  We say that a space $X$ is \emph{simply connected} if
  \begin{enumerate}
    \item $X$ is path-connected, and
    \item $\pi_1(X, x_0) = \{1\}$
      for some $x_0 \in X$.
  \end{enumerate}
\end{definition}

\begin{remark}
  Simply connected means that ``points in $X$ are
  connected in a very simple way.''
\end{remark}

\begin{lemma}
  A space $X$ is simply connected if and only if
  every two points in $X$ are connected by a unique
  homotopy class of paths in $X$.
\end{lemma}

\begin{proof}
  $(\Leftarrow)$ Clearly $X$ is path-connected.
  Furthermore, any loop based at $x_0$ is a path
  from $x_0$ to itself, and the constant is as well.
  Thus any loop is homotopic to the constant loop,
  i.e. $\pi_1(X, x_0) = \{1\}$.

  $(\Rightarrow)$ For any $a, b \in X$, there exists a
  path from $a$ to $b$ (since $X$ is path-connected).
  Now suppose $\gamma, \delta : [0, 1] \to X$
  are paths from $a$ to $b$. By
  $\pi_1(X, a) = \{1\}$ we know that
  $\gamma * \overline{\delta} \sim e_a$, so
  \[
    \gamma \sim \gamma * (\overline{\delta} * \delta)
    \sim (\gamma * \overline{\delta}) * \delta
    \sim e_a * \delta
    \sim \delta,
  \]
  i.e. $\gamma$ and $\delta$ are in the same
  homotopy class.
\end{proof}

\begin{lemma}
  Let $X = A \cup B$, where $A, B, A \cap B$
  are open and path-connected. Let $x_0 \in A \cap B$.
  Then any loop $\gamma : [0, 1] \to X$ based at
  $x_0$ can be written as
  \[
    \gamma \sim \gamma_1 * \gamma_2 * \dots * \gamma_n,
  \]
  where each $\gamma_i$ is a loop in $A$ or $B$
  based at $x_0$.
\end{lemma}

\begin{proof}
  We first claim that there exist
  $0 = t_0 < t_1 < \dots < t_n = 1$ such that
  $\im \gamma|_{[t_{i - 1}, t_i]} \subseteq A$
  or $B$, and $\gamma(t_i) \in A \cap B$ for every $i$.
  The proof of this will use the following
  topology fact:
  \begin{quote}
    \textbf{Lemma} (Lebesgue number lemma)\textbf{.}
    Let $X$ be a compact
    metric space and $\{U_\alpha\}_{\alpha \in A}$ be
    an open cover. Then there exists a \emph{Lebesgue number}
    $\delta > 0$ such that for all sets $S$ with
    $\diam(S) = \sup_{x, y \in S} d(x, y) < \delta$,
    there exists $\alpha \in A$ such that $S \subseteq U_\alpha$.
  \end{quote}
  To prove the claim, let $U_1 = \gamma^{-1}(A)$
  and $U_2 = \gamma^{-1}(B)$, which is an open
  cover of $[0, 1]$. So there exists $\delta > 0$
  such that if $|b - a| < \delta$, then
  $[a, b] \subseteq U_i$ for $i = 1$ or $2$. Thus
  $\gamma([a, b]) \subseteq A$ or $B$. Now let
  $n$ be a positive integer such that $1 / n < \delta$,
  so that for each $i$ we have
  \[\im \gamma|_{[i / n, (i + 1) / n]} \subseteq A \text{ or } B.\]
  So we start with $t_i = 1 / n$ for $i = 0, \dots, n$.
  Now if $\gamma|_{[t_{i - 1}, t_i]}$ and
  $\gamma|_{[t_i, t_{i + 1}]}$ both have image in
  $A$ (or both in $B$), then throw out $t_i$. Then
  $\gamma|_{[t_{i - 1}, t_{i}]}$ will have image in
  $A$ or $B$ and $\gamma(t_{i}) \in A \cap B$,
  as desired.

  Now given the claim, let
  $\delta_i : [0, 1] \to A \cap B$ connect
  $x_0$ to $\gamma(t_1)$, and set
  $\gamma_i = \gamma|_{[t_{i - 1}, t_i]}$. Then
  \[
    \gamma \sim \gamma_1 * \gamma_2 * \dots * \gamma_n
    \sim (\gamma_1 * \overline{\delta}_1)
    * (\delta_1 * \gamma_2 * \overline{\delta}_2)
    * \dots
    * (\delta_{n - 1} * \gamma_n),
  \]
  where each of the above loops is either in
  $A$ or $B$.
\end{proof}

\begin{theorem}
  We have $\pi_1(S^n, x_0) = \{1\}$ for all $n \ge 2$.
\end{theorem}

\begin{proof}
  We have $\pi_1(S^n, x_0) = \{1\}$ for all $n \ge 2$.
  Let $A = S^n \setminus \{(0, \dots, 0, 1)\}$ and
  $B = S^n \setminus \{(0, \dots, 0, -1)\}$.
  Note that $A \cong B \cong \R^n$, so they are path-connected.
  Furthermore, $A \cap B = S^{n - 1} \times \R$,
  which is also path-connected if $n \ge 2$.\footnote{Note that this does not work for $n = 1$: the intersection $S^0 \times \R = \{\pm 1\} \times \R$ is not
  path-connected.}
  Now take any $x_0 \in A \cap B$. Then any
  $[\gamma] \in \pi_1(S^n, x_0)$ can be written as
  \[
    [\gamma] = [\gamma_1] * [\gamma_2] * \dots * [\gamma_n],
  \]
  where $[\gamma_i] \in \pi_1(A, x_0)$ or
  $\pi_1(B, x_0)$ by the lemma. But we have
  \[
    \pi_1(A, x_0) = \pi_1(B, x_0) = \{1\},
  \]
  so $[\gamma] = [e_{x_0}]$ and hence
  $\pi_1(S^n, x_0) = \{1\}$.
\end{proof}

\begin{theorem}
  Given two spaces $X$ and $Y$, we have
  $\pi_1(X \times Y, (x_0, y_0)) \cong \pi_1(X, x_0) \times \pi_1(Y, y_0)$.
\end{theorem}

\begin{proof}
  The map $\Phi : \pi_1(X, x_0) \times \pi_1(Y, y_0) \to \pi_1(X \times Y, (x_0, y_0))$
  given by
  \[
    ([\gamma], [\delta]) \mapsto [\gamma \times \delta],
  \]
  where $(\gamma \times \delta)(t) = (\gamma(t), \delta(t))$,
  is an isomorphism. Check as an exercise that
  $\Phi$ is well-defined and a bijection (for the
  second part, consider the projections $p_X : X \times Y \to X$
  and $p_Y : X \times Y \to Y$).
\end{proof}

\section{Fundamental Group of the Circle}

Our next objective is the following computation:
\begin{theorem}\label{thm:fund-group-circle}
  We have $\pi_1(S^1, (1, 0)) \cong \Z$. In particular,
  the map sending $n \in \Z$ to
  \[
    \gamma_n : [0, 1] \to S^1 : t \mapsto (\cos 2\pi nt, \sin 2\pi nt)
  \]
  is an isomorphism $\Z \to \pi_1(S^1, (1, 0))$.
\end{theorem}

\begin{remark}
  The proof is an example of a very important
  technique that we will see again soon. The proof
  involves studying the map
  \[
    p : \R \to S^1 : t \mapsto (\cos 2\pi t, \sin 2\pi t).
  \]
  Note that $p^{-1}((1, 0)) = \Z$. This is a
  particular example of a \emph{covering map}, which we
  will study later.
\end{remark}

\begin{definition}
  If $\gamma : [0, 1] \to S^1$ is a path based at
  the point $(1, 0)$, then a
  \emph{lift of $\gamma$ based at}
  $n \in \Z$ is a map $\widetilde{\gamma}_n : [0, 1] \to \R$
  such that $\widetilde{\gamma}_n(0) = n$ and
  $p \circ \widetilde{\gamma}_n = \gamma$.
\end{definition}

\begin{lemma}\label{lem:lifting}
  We have the following:
  \begin{enumerate}
    \item For each $n \in \Z$, each loop
      $\gamma : [0, 1] \to S^1$ based at $(0, 1)$
      lifts to a unique path $\widetilde{\gamma}_n$
      based at $n$.
    \item If $\gamma \sim \gamma'$ are loops in
      $S^1$ based at $(0, 1)$ and
      $\widetilde{\gamma}_n, \widetilde{\gamma}'_n$
      are their lifts based at $n$, then
      $\widetilde{\gamma}_n \sim \widetilde{\gamma}'_n$
      rel $\{0, 1\}$.
  \end{enumerate}
  The above properties are called \emph{path lifting}
  and \emph{homotopy lifting}.
\end{lemma}

\begin{proof}
  We will prove this next class.
\end{proof}

\begin{proof}[Proof of Theorem \ref{thm:fund-group-circle}]
  Given $[\gamma] \in \pi_1(S^1, (1, 0))$, part (a) of Lemma \ref{lem:lifting} says that there is a unique lift $\widetilde{\gamma}_0 : [0, 1] \to \R$.
  Since $\widetilde{\gamma}_0(1) \in p^{-1}((1, 0)) = \Z$,
  we can define $\Phi : \pi_1(S^1, (1, 0)) \to \Z$ by
  $[\gamma] \mapsto \widetilde{\gamma}_0(1)$.
  Part (b) of Lemma \ref{lem:lifting} says that $\Phi$ is
  well-defined. We need to show the following:
  \begin{enumerate}
    \item $\Phi$ is surjective:
      
      Let $\widetilde{\delta}^n(t) = nt$ for
      $t \in [0, 1]$ and $\delta^n = p \circ \widetilde{\delta}$.
      Then $\widetilde{\delta}^n$ is the lift of $\delta^n$ based at $0$, and
      $\Phi([\delta^n]) = n$.
    \item $\Phi$ is injective:

      Suppose $\gamma, \gamma'$ are loops in $S^1$
      such that
      \[
        \Phi([\gamma]) = \widetilde{\gamma}_0(1) = \widetilde{\gamma}'_0(1) = \Phi([\gamma']).
      \]
      Set $\widetilde{H}(s, t) = (1 - t) \widetilde{\gamma}_0(s) + t \widetilde{\gamma}'_0(s)$
      and $H = p \circ \widetilde{H}$. Then $H$ is a
      homotopy from $\gamma$ to $\gamma'$ (exercise).
    \item $\Phi$ is a homomorphism:

      We will prove this next class.
  \end{enumerate}
  Thus $\Phi$ is an isomorphism, and we have
  $\pi_1(S^1, (1, 0)) \cong \Z$.
\end{proof}

  \chapter{Jan.~27 --- Fundamental Group of the Circle}

\section{Path Lifting}

\begin{proof}[Proof of Lemma \ref{lem:lifting}]
  $(a)$ Let $A = S^1 \setminus \{(1, 0)\}$, and note that
  \[
    p^{-1}(A) = \bigcup_{i \in \Z} (i, i + 1) = \bigcup_{i \in \Z} A_i.
  \]
  Notice that each restriction
  $p|_{A_i} : A_i \to A$ is a homeomorphism.
  Now let $B = S^1 \setminus \{(-1, 0)\}$, so
  \[
    p^{-1}(B) = \bigcup_{i \in \Z} \left(i - \frac{1}{2}, i + \frac{1}{2}\right)
    = \bigcup_{i \in \Z} B_i.
  \]
  Similarly, each $p|_{B_i} : B_i \to B$ is a
  homeomorphism. Now if $\gamma : [0, 1] \to S^1$ is
  contained in $A$ (or $B$), we can choose any $i \in \Z$
  and let $\widetilde{\gamma} = (p|_{A_i})^{-1} \circ \gamma$,
  giving a lift of $\gamma$. Then for a general
  $\gamma : [0, 1] \to S^1$ with $\gamma(0) = (1, 0)$,
  the set $\{\gamma^{-1}(A), \gamma^{-1}(B)\}$ is
  an open cover of the compact metric space $[0, 1]$,
  so there exists a Lebesgue number $\delta > 0$
  such that any interval $[a, b]$ with $b - a < \delta$
  lies in either $\gamma^{-1}(A)$ or $\gamma^{-1}(B)$.
  Choose $n$ such that $1 / n < \delta$. If
  $t_n = i / n$ for $i = 0, \dots, n$, then
  \[
    \gamma([t_i, t_{i + 1}]) \subseteq A \text{ or } B
  \]
  for every $i$. Again for convenience, if
  $[t_{i - 1}, t_i]$ and $[t_i, t_{i + 1}]$ are
  both in $\gamma^{-1}(A)$ or $f^{-1}(B)$, then
  discard $t_i$. So we have a partition
  $0 = t_0 < t_1 < \dots < t_k = 1$ such that
  (note that $\gamma$ starts at $(1, 0) \notin A$)
  \[
    \gamma([t_i, t_{i + 1}]) \subseteq
    \begin{cases}
      A & \text{if } i \text{ is odd}, \\
      B & \text{if } i \text{ is even}.
    \end{cases}
  \]
  Then we want to build $\widetilde{\gamma}_n$.
  Define $\widetilde{\gamma}_n$ on $[t_0, t_1]$
  to be $(p|_{B_n})^{-1} \circ \gamma|_{[t_0, t_1]}$.
  Now $\widetilde{\gamma}_n(t_1) \in A_i$ for a
  unique $i$, so define $\widetilde{\gamma}_n$
  on $[t_1, t_2]$ by $(p|_{A_i})^{-1} \circ \gamma|_{[t_1, t_2]}$.
  Note that $\widetilde{\gamma}_n$ is continuous
  on $[t_0, t_2]$ since the two definitions agree
  at $t = t_1$.
  Inductively continue to define the lift
  $\widetilde{\gamma}_n$ on all of $[0, 1]$.

  $(b)$ The proof is very similar to path lifting.
  Given a homotopy $H : [0, 1] \times [0, 1] \to S^1$,
  we can find a Lebesgue number $\delta > 0$
  for $\{H^{-1}(A), H^{-1}(B)\}$. Pick $n$
  such that $\sqrt{2} / n < \delta$ and break
  $[0, 1] \times [0, 1]$ into $n^2$ squares
  of side length $1 / n$. The diameter of each
  square is at most $\sqrt{2} / n$, so each square
  can be lifted as above. Finish the construction
  as an exercise to lift $H$ on all of
  $[0, 1] \times [0, 1]$.
\end{proof}

\section{Applications of the Fundamental Group of \texorpdfstring{$S^1$}{S1}}

\begin{corollary}
  There is no retraction $D^2 \to \partial D^2$.
\end{corollary}

\begin{proof}
  Suppose there was a retraction $r : D^2 \to \partial D^2$,
  and let $i : S^1 \to D^2$ be the inclusion of $S^1$
  as the boundary of $D^2$. Then we have the composition:
  \begin{center}
    \begin{tikzcd}
      S^1 \ar[r, "i"] & D^2 \ar[r, "r"] & S^1
    \end{tikzcd}
  \end{center}
    Noting that $r \circ i = S^1 \to S^1$ is the
    identity, so $(r \circ i)_* : \pi_1(S^1, (1, 0)) \to \pi_1(S^1, (1, 0))$
    is the identity map. In particular,
    $r_* \circ i_* = (r \circ i)_*$ is the identity map,
    hence $i_*$ must be injective.
    But
    \[
      i_* : \pi_1(S^1, (1, 0)) \to \pi_1(D^2, (1, 0))
    \]
    where $\pi_1(S^1, (1, 0)) \cong \Z$ and
    $\pi_1(D^2, (1, 0)) = \{1\}$, so $i_*$ cannot be
    injective. Contradiction.
\end{proof}

\begin{corollary}
  Any map $f : D^2 \to D^2$ has a fixed point, i.e.
  $x \in D^2$ such that $f(x) = x$.
\end{corollary}

\begin{proof}
  Suppose otherwise that $f : D^2 \to D^2$ has no fixed
  points.
  Then for each $x \in D^2$, there is a unique ray
  $R_x$ starting at $f(x)$ and going through $x$.
  Note that $R_x \cap \partial D^2$ in a unique point
  (on the interior of $R_x$). Define
  $r : D^2 \to S^1$ by $x \mapsto R_x \cap \partial D^2$.
  Show that $r$ is continuous as an exercise (e.g.
  parametrize the line).
  But then $r$ is a retraction $D^2 \to \partial D^2$,
  a contradiction.
\end{proof}

\begin{remark}
  There are more applications such as the
  fundamental theorem of algebra, the ham sandwich
  theorem, and the Borsuk-Ulam theorem. See Hatcher
  for more details.
\end{remark}

\section{Free Products of Groups}

\begin{definition}
  Let $G_1$ and $G_2$ be groups. A \emph{word} in
  $G_1 \sqcup G_2$ is a finite sequence
  \[
    x = (x_1, x_2, \dots, x_n)
  \]
  for some $n$, where each $x_i$ is in $G_1$ or $G_2$.
  Define an equivalence relation on words in
  $G_1 \sqcup G_2$ which is generated by (show as
  an exercise that this is in fact an equivalence
  relation):
  \begin{enumerate}
    \item replace $a, b$ in a sequence by $ab$ if
      $a, b$ are in the same group (or the reverse of this), and
    \item if $e_i$ (the identity in $G_i$) is in
      a sequence, then remove it (or add it in any
      place in a sequence).
  \end{enumerate}
  Denote the equivalence class of a word $x$ by $[x]$.
  Call a word $x = (x_1, \dots, x_n)$ \emph{reduced} if
  \begin{enumerate}
    \item $x_j \ne e_i$ for any $j$ or $i$, and
    \item $x_i$ and $x_{i + 1}$ are from different
      groups.
  \end{enumerate}
  Show that each $[x]$ contains a unique reduced word
  (note: uniqueness is hard).
  The \emph{free product} of $G_1$ and $G_2$ is
  the group $G_1 * G_2$ of all equivalence classes of
  words in $G_1 \sqcup G_2$, with multiplication
  \[
    [x_1, \dots, x_n] \cdot [y_1, \dots, y_m]
    = [x_1, \dots, x_n, y_1, \dots, y_m].
  \]
\end{definition}

\begin{remark}
  Note that in $G_1 * G_2$, the identity
  $e$ is the empty word and the inverse is given by
  \[[x_1, \dots, x_n]^{-1} = [x_n^{-1}, \dots, x_1^{-1}].\]
  Check as an exercise that $G_1 * G_2$ is in
  fact a group (really only need to check associativity).
\end{remark}

\begin{prop}
  Let $j_i : G_i \to G_1 * G_2$ be the inclusion
  of $G_i$ into $G_1 * G_2$. Given any homomorphisms
  $\phi_i : G_i \to H$ where $H$ is any group, there
  exists a unique homomorphism
  $\phi : G_1 * G_2 \to H$ such that
  $\phi \circ j_i = \phi_i$, i.e.
  the following diagram commutes:
  \begin{center}
    \begin{tikzcd}
      & G_1 \ar[d, "j_1"] \ar[ddr, "\phi_1", bend left=30] \\
      G_2 \ar[r, "j_2"] \ar[drr, "\phi_2", bend right=20, swap] & G_1 * G_2 \ar[dr, dashed, "\phi"] \\
        & & H
    \end{tikzcd}
  \end{center}
\end{prop}

\begin{proof}
  If the $x_i$ are reduced and $x_1 \in G_1$, then define
  \[
    \phi_1(x_1, x_2, \dots, x_n) = \phi_1(x_1) \cdot \phi_2(x_2) .\cdot \phi_1(x_3) \cdot \dots.
  \]
  Then check the following as an exercise:
  \begin{enumerate}
    \item Show such that $\phi$ exists and is unique.
    \item Show this property \emph{defines} the
      free product, i.e. if $D$ is another group
      satisfying the property in the proposition, then
      $D \cong G_1 * G_2$.
  \end{enumerate}
  The second part above says that this is
  the \emph{universal property} of the free product.
\end{proof}

\begin{example}
  Represent $\Z$ in product notation via $\{x^n\}$, where
  $x^n x^m = x^{n + m}$. Then
  \[
    \Z * \Z = \{x^n\} * \{y^m\}
    = \{e,
      x^{n_1} y^{m_1} \dots x^{n_k},
      x^{n_1} y^{m_1} \dots y^{m_k},
      y^{m_1} x^{n_1} \dots y^{m_k},
      y^{m_1} x^{n_1} \dots x^{n_k}
    \}.
  \]
\end{example}

\end{document}
