\documentclass[12pt, letterpaper, oneside]{book}
\usepackage[margin={0.6in, 0.75in}]{geometry}
\usepackage{microtype}
% \usepackage{kpfonts}
\usepackage{amsmath, amssymb, amsthm}
\usepackage{parskip}
\usepackage[many]{tcolorbox}
\usepackage{footnote}
\usepackage{cancel}
\usepackage{titlesec}
\usepackage{pgffor}
\usepackage[shortlabels, inline]{enumitem}
\usepackage{hyperref}
\usepackage{tikz-cd}

\usepackage[overload]{textcase}

\renewcommand{\chaptername}{Lecture}
\newtheorem{axiom}{Axiom}[chapter]
\newtheorem{theorem}{Theorem}[chapter]
\newtheorem{prop}{Proposition}[chapter]
\newtheorem{corollary}{Corollary}[theorem]
\newtheorem{lemma}{Lemma}[chapter]
\theoremstyle{definition}
\newtheorem{definition}{Definition}[chapter]
\newtheorem{exercise}{Exercise}[chapter]
\newtheorem{example}{Example}[definition]
\newtheorem*{remark}{Remark}

\tcbset{sharp corners, breakable, enhanced, parbox=false}
\newtcolorbox{mybox}[3][]
{
  colframe = #2!150,
  colback  = #2!5,
  coltitle = #2!0!white,  
  title    = {#3},
  #1,
}

\titleformat{\chapter}[display]
    {\normalfont\huge\bfseries}{\chaptertitlename\ \thechapter}{20pt}{\Huge}
\titlespacing*{\chapter}{0pt}{0pt}{40pt}

\newcommand{\R}{\mathbb{R}}
\newcommand{\N}{\mathbb{N}}
\newcommand{\Z}{\mathbb{Z}}
\newcommand{\C}{\mathbb{C}}
\newcommand{\Q}{\mathbb{Q}}
\newcommand{\F}{\mathbb{F}}
\newcommand{\sphere}{\mathbb{S}}
\newcommand{\ZF}{\mathsf{ZF}}
\newcommand{\ZFC}{\mathsf{ZFC}}
\newcommand{\AC}{\mathsf{AC}}

\newcommand{\T}{\mathcal{T}}
\newcommand{\B}{\mathcal{B}}

\DeclareMathOperator{\Vol}{Vol}
\DeclareMathOperator{\Int}{int}
\DeclareMathOperator{\area}{area}
\DeclareMathOperator{\curl}{curl}
\DeclareMathOperator{\maps}{maps}
\DeclareMathOperator{\id}{id}
\DeclareMathOperator{\con}{\#}

\title{MATH 6441: Algebraic Topology I}
\author{Frank Qiang\\Instructor: John Etnyre}
\date{Georgia Institute of Technology\\Spring 2025}

\begin{document}
  \maketitle

  \begingroup
  \let\cleardoublepage\clearpage
  \tableofcontents
  \endgroup

  % \foreach \i in {00, 01, 02, 03, 04, ..., 50} {%
  %   \edef\FileName{lectures/lecture\i.tex}%     The % here are necessary to eliminate any
  %   \IfFileExists{\FileName}{%  spurious spaces that may get inserted
  %      \input{\FileName}%       at these points
  %   }
  % }
  \chapter{Jan.~6 --- CW-Complexes}

\section{Introduction and Motivation}
Algebraic topology builds ``functions'' (actually
\emph{functors})
\[
  \{
    \text{topological spaces, continuous maps}
  \}
  \longrightarrow
  \{
    \text{algebraic things, algebraic maps}
  \},
\]
where ``algebraic things'' can be groups, vector
spaces, etc. The main objective of algebraic topology is
to \emph{distinguish topological spaces}, e.g.
showing that $\R^n \ncong \R^m$ for $n \ne m$.
More applications are:
\begin{enumerate}
  \item Studying maps between spaces.\footnote{All maps and functions in this class are continuous unless otherwise specified.}
    \begin{itemize}
      \item Does a given space $M$ embed in $N$? For
        instance, for what $m$ does $\R P^n$ embed
        in $\R^m$? (This is still not known in general.)
        Here $\R P^n$ is the real projective space.
      \item Lifting maps, i.e. given
        $f : A \to B$ and $g : E \to B$, does
        there exist a map $\widetilde{f} : A \to E$
        such that $g \circ \widetilde{f} = f$?
        In other words, is there a map $\widetilde{f}$
        such that the following diagram commutes:
        \begin{center}
        \begin{tikzcd}
          & E \arrow[d, "g"] \\
          A \arrow[ur, "\widetilde{f}", dashed] \arrow[r, "f", swap] & B
        \end{tikzcd}
        \end{center}
      \item Fixed point problems: Given $f : X \to X$,
        does $f$ have a fixed point, i.e. $x \in X$
        such that $f(x) = x$? Such theorems are used
        to prove the existence of solutions to
        ordinary differential equations, for instance.
    \end{itemize}
  \item Group actions, e.g. which finite groups act
    freely on $S^n$?
  \item Group theory, e.g. showing that every subgroup of a free
    group is free.
    Another example is that if $F_n$ is
    the free group on $n$ generators, then
    its \emph{commutator} $[F_n, F_n]$ is not
    finitely generated.
  \item Algebra, e.g. proving the fundamental theorem
    of algebra.
\end{enumerate}

This course will cover the following topics:
\begin{enumerate}
  \item The \emph{fundamental group}
    $\pi_1(X, x_0)$ of a space $X$ for $x_0 \in X$,
    and \emph{covering spaces}.
  \item The \emph{homology groups} $H_k(X)$ for
    $k = 0, 1, 2, \dots$. These groups are abelian.
  \item The \emph{cohomology ring}
    $H^*(X) = \bigoplus_{k = 0}^\infty H^k(X)$.
\end{enumerate}
But before getting to this, we need to develop some
important ideas.

\section{CW-Complexes}

\begin{definition}
  Let $D^n \subseteq \R^n$ be the unit disk and
  $S^{n - 1} = \partial D^n$. Given a topological
  space $Y$ and a continuous map $a : S^{n - 1} \to Y$,
  the space obtained from $Y$ by \emph{attaching} an
  \emph{$n$-cell} (\emph{via $a$}) is
  \[
    Y \cup_a D^n = (Y \sqcup D^n) / {\sim},
  \]
  where the equivalence relation $\sim$ is given by
  $x \sim a(x)$ for $x \in \partial D^n$.\footnote{Here $A \sqcup B$ denotes the \emph{disjoint union} of A and B.}
\end{definition}

\begin{definition}
  An \emph{$n$-complex} or an \emph{$n$-dimensional CW-complex}
  is defined inductively by:
  \begin{itemize}
    \item A $(-1)$-complex is the empty set $\varnothing$.
    \item An $n$-complex $X^n$ is a space obtained
      from an $(n - 1)$-complex by attaching $n$-cells.
  \end{itemize}
  An $n$-complex is \emph{finite} if it involves
  only a finite number of cells. The \emph{$k$-skeleton}
  of $X$ is the union of all $n$-cells in $X$
  with $n \le k$.
\end{definition}

\begin{remark}
  Any CW-complex is Hausdorff. See Hatcher for a proof.
\end{remark}

\begin{example}
  Here are some examples of CW-complexes:
  \begin{itemize}
    \item A $0$-complex is a union of points.
      This is because $D^0 = \{\text{pt}\}$ and
      $\partial D^0 = \varnothing$.
    \item A $1$-complex is a graph (points and lines
      connecting them).
    \item The torus $T$ (a square with opposite sides
      identified) is a $2$-complex. Here
      the $0$-skeleton $T^{(0)}$ is the common corner
      on the square and the $1$-skeleton $T^{(1)}$
      is two sides of the square after taking the
      quotient. The $2$-skeleton $T = T^{(2)}$ is the
      entire torus.
    \item Another example of a $2$-complex is the
      two-holed torus, which is obtained by identifying
      the edges of an octagon (pairs of every other
      edge identified with opposite orientation).
    \item A third example of a $2$-complex is
      $X^{(1)} \cup_a D^2$ given an attaching map
      $D^2 \to X^{(1)}$.
    \item Consider the unit sphere $S^n \subseteq \R^{n + 1}$.
      One way to give $S^2$ a CW-complex structure is
      to see the sphere as two disks $D^2$ glued together,
      resulting in one $0$-cell, one $1$-cell, and two
      $2$-cells. Another way is to start with
      two points, attach a $1$-cell to get a circle,
      and then attaching two disks to get $S^2$. This
      results in two $0$-cell, two $1$-cell, and two
      $2$-cells.

      The second idea generalizes to $S^n$. We can
      write
      \[
        S^n = S^{n - 1} \cup_{a_1} D^n \cup_{a_2} D^n,
      \]
      where $S^{n - 1}$ inductively has a CW-complex
      structure. This yields two $k$-cells for each
      $k \le n$.

      Another way to put a CW-complex structure
      on $S^n$ is to attach $D^n$ to a point with
      $\partial D^n \to \{\text{pt}\}$.
      In particular, notice that a space
      can in general have several different CW-complex
      structures.
    \item Consider the \emph{$n$-dimensional real projective space}
      \[
        \R P^n = \{\text{lines through the origin in } \R^{n + 1}\}.
      \]
      Since each line through the origin passes through
      $S^n$ twice, we can equivalently think of $\R P^n$
      as the unit sphere $S^n$ with antipodal points
      identified.

      We can also think of this as $D^n$ with
      antipodal points on $\partial D^n$ identified.
      Since $\partial D^n = S^{n - 1}$, this is
      simply $\R P^{n - 1} \cup_{a} D^n$, where
      $a : \partial D^n \to \R P^{n - 1}$ is the
      quotient map. This gives $\R P^n$ a CW-complex
      structure with one $k$-cell for each $k \le n$.
    \item The complex projective space
      $\C P^n$ has a similar CW-complex structure with
      one $k$-cell for each even $k \le 2n$.
      One can verify this as an exercise.
    \item Any smooth manifold has a CW-complex
      structure. See Hatcher.
  \end{itemize}
\end{example}

\begin{exercise}
  Show the product of CW-complexes is a CW-complex.
\end{exercise}

\begin{definition}
  A \emph{subcomplex} of a CW-complex $X$ is a closed
  subset $A \subseteq X$ that is a union of cells in $X$.
  In particular, $A$ is also a CW-complex and
  $(X, A)$ is called a \emph{CW-pair}.
\end{definition}

\section{Homotopy}

\begin{definition}
  Let $X$ and $Y$ be topological spaces. Two maps
  $f, g : X \to Y$ are \emph{homotopic}, denoted
  $f \sim g$, if there exists a continuous map
  $\Phi : X \times [0, 1] \to Y$ such that
  \[
    \Phi(x, 0) = f(x) \quad \text{and} \quad \Phi(x, 1) = g(x).
  \]
  In this case, $\Phi$ is called a \emph{homotopy}
  between $f$ and $g$.
\end{definition}

\begin{remark}
  We note the following:
  \begin{itemize}
    \item A homotopy $\Phi$ gives a family of maps
      $\Phi_{t_0} : X \to Y$ given by
      $x \mapsto \Phi(x, t_0)$ which is continuous
      in $t_0$. So maps are homotopic if there is a
      continuous family of maps between them.
    \item If $A \subseteq X$ then we say that
      $f$ is \emph{homotopic to $g$ rel $A$},
      denoted $f \sim_A g$, if there exists
      $\Phi$ as above with the additional property
      that $\Phi(x, t) = f(x)$ for all $x \in A$,
      i.e. points in $A$ are fixed.
    \item If $A \subseteq X$ and $B \subseteq Y$,
      then the notation $f : (X, A) \to (X, B)$
      means that $f : X \to Y$ and
      $f(a) \in B$ for each $a \in A$.
      We say that $f$ is a \emph{map of pairs}.
      If $f, g : (X, A) \to (Y, B)$, then $f, g$
      are homotopic as maps of pairs if each
      $\Phi_t$ is a map of pairs.
  \end{itemize}
\end{remark}

  \chapter{Jan.~8 --- Homotopy}

\section{More on Homotopy}

\begin{example}
  For any space $X$, any map $f : X \to [0, 1]$ is
  homotopic to the map $g : X \to [0, 1]$ given by
  $x \mapsto 0$. To see this, we have the homotopy
  $\Phi : X \times [0, 1] \times [0, 1]$ defined by
  \[
    (x, t) \mapsto (1 - t) f(x).
  \]
  We can see that $\Phi(x, 0) = f(x)$ and
  $\Phi(x, 1) = 0 = g(x)$.
\end{example}

\begin{exercise}
  Show that homotopy is an equivalence relation on
  maps $X \to Y$.
\end{exercise}

\begin{definition}
  Let $C(X, Y) = \{\text{continuous maps from $X$ to $Y$}\}$.
  Let $[X, Y] = C(X, Y) / {\sim}$, i.e. homotopic
  maps are identified with each other.
\end{definition}

\begin{example}
  We have the following:
  \begin{enumerate}
    \item $[X, [0, 1]] = \{g\}$ for any space $X$, where
      $g$ is the map $x \mapsto 0$ as above.
    \item $[\{*\}, X] = \{\text{path components of $X$}\}$.
  \end{enumerate}
\end{example}

\section{Homotopy Groups}
\begin{definition}
  We call a space $X$ \emph{pointed} if there
  is a designated ``base point'' $x_0 \in X$.
  Given two pointed spaces $(X, x_0)$ and
  $(Y, y_0)$, we define
  \[
    [X, Y]_0 = \{\text{homotopy classes of maps of pairs
    $(X, \{x_0\}) \to (Y, \{y_0\})$}\}.
  \]
\end{definition}

\begin{definition}
  Let $y_0$ be the north pole in $S^n$, i.e.
  $S^n \subseteq \R^{n + 1}$ is the unit sphere and
  $y_0 = (0, \dots, 0, 1)$. The
  \emph{$n$th homotopy group}
  of a pointed space $(X, x_0)$ is
  $\pi_n(X, x_0) = [S^n, X]_0$.
\end{definition}

\begin{remark}
  The homotopy group $\pi_n(X, x_0)$ is in fact a
  group. We will study $\pi_1(X, x_0)$ next and it is
  called the \emph{fundamental group} of $(X, x_0)$.
\end{remark}

\begin{remark}
  For which $(Y, y_0)$ is $[Y, X]_0$ ``naturally'' a
  group for all $(X, x_0)$? Similarly, for which
  $(Y, y_0)$ is $[X, Y]_0$ a group for all $(X, x_0)$?
  Here, given a map $f : (X_1, x_1) \to (X_2, x_2)$,
  there is an obvious \emph{induced map}
  $f_* : [Y, X_1]_0 \to [Y, X_2]_0$ given by
  $[g] \mapsto [f \circ g]$. Similarly, there is a map
  $f^* : [X_2, Y]_0 \to [X_1, Y]_0$ given by
  $[g] \mapsto [g \circ f]$. In the questions above,
  ``naturally'' means that $f_*$ and $f^*$ are
  homomorphisms.
  for any $(X_1, x_1)$ and $(X_2, x_2)$.
  The (perhaps unsatisfying) answer is that
  a space satisfying the first condition is called
  an \emph{$H$-space}, and a space satisfying the second
  is called an \emph{$H'$-space}.
\end{remark}

\section{Homotopy Equivalence}
\begin{definition}
  We say that $f : X \to Y$ is the \emph{homotopy inverse}
  to a function $g : Y \to X$ if
  $f \circ g \sim \id_Y$ and $g \circ f \sim \id_X$,
  where $\id_X$ and $\id_Y$ are the identity maps
  on $X$ and $Y$.
  If $g$ has a homotopy inverse, then
  we call $g$ a \emph{homotopy equivalence} from $Y$ to
  $X$ and we call $X, Y$ \emph{homotopy equivalent}.\footnote{We will denote homotopy equivalence by $X \simeq Y$ or simply $X \sim Y$.}
\end{definition}

\begin{exercise}
  Show that homotopy equivalence is an equivalence
  relation.
\end{exercise}

\begin{lemma}
  The following are equivalent:
  \begin{enumerate}
    \item $X$ and $Y$ are homotopy equivalent.
    \item For any space $Z$, there is a one-to-one
      correspondence $\phi_Z : [X, Z] \to [Y, Z]$
      such that for all continuous maps
      $h : Z \to Z'$, the following diagram commutes:
      \begin{center}
        \begin{tikzcd}
          {[X, Z]} \ar[r, "\phi_Z"] \ar[d, "h_*"] & {[Y, Z]} \ar[d, "h_*"] \\
          {[X, Z']} \ar[r, "\phi_{Z'}"] & {[Y, Z']}
        \end{tikzcd}
      \end{center}
    \item For any space $Z$, there is a one-to-one
      correspondence $\phi^Z : [Z, X] \to [Z, Y]$
      such for that all continuous maps $h : Z \to Z'$,
      the following diagram commutes:
      \begin{center}
        \begin{tikzcd}
          {[Z', X]} \ar[r, "\phi^{Z'}"] \ar[d, "h^*"] & {[Z', Y]} \ar[d, "h^*"] \\
          {[Z, X]} \ar[r, "\phi^{Z}"] & {[Z, Y]}
        \end{tikzcd}
      \end{center}
  \end{enumerate}
\end{lemma}

\begin{proof}
  This is left as an exercise.
\end{proof}

\begin{remark}
  Based on the previous lemma, two spaces are
  homotopy equivalent if and only if homotopy classes
  of maps to and from the space are ``naturally
  equivalent.''
\end{remark}

\begin{example}
  We have the following:
  \begin{itemize}
    \item Homeomorphic spaces are homotopy equivalent.
    \item Let $X = S^1$ and $Y = S^1 \times [0, 1]$.
      We claim that $X$ is homotopy equivalent to $Y$.

      Define the maps $f : S^1 \to S^1 \times [0, 1]$ by
      $x \mapsto (x, 0)$ and $g : S^1 \times [0, 1] \to S^1$
      by $(x, t) \mapsto x$. Then we can see
      that $g \circ f : S^1 \to S^1$ maps
      $x \mapsto x$, so $g \circ f = \id_{S^1}$. On the
      other hand, the composition
      $f \circ g : S^1 \times [0, 1] \to
      S^1 \times [0, 1]$ maps $(x, t) \mapsto (x, 0)$.
      Now $f \circ g \sim \id_{S^1 \times [0, 1]}$ by
      homotopy. For instance, define
      $\Phi : (S^1 \times [0, 1]) \times [0, 1] \to
      (S^1 \times [0, 1])$ by
      $((x, t), s) \mapsto (x, st)$, so
      \[
        \Phi((x, t), 1) = (x, t) = \id_{S^1 \times [0, 1]}(x, t)
        \quad \text{and} \quad
        \Phi((x, t), 0) = (x, 0) = f \circ g.
      \]
      Thus $f$ is a homotopy equivalence from $S^1$ to
      $S^1 \times [0, 1]$. Note that
      $S^1 \times [0, 1]$ is the annulus.
  \end{itemize}
\end{example}

\begin{definition}
  A space is called \emph{contractible} if it is
  homotopy equivalent to a point.
\end{definition}

\begin{example}
  The spaces $[0, 1]$ and $\R^n$ are contractible
  (exercise).
\end{example}

\begin{definition}
  If $A \subseteq X$, then a \emph{retraction of $X$ to $A$}
  is a map $r : X \to A$ such that $r(a) = a$ for
  every $a \in A$. A \emph{deformation retraction} of
  $X$ to $A$ is a retraction $r : X \to A$
  that is homotopic rel $A$ to the identity map $\id_X$,
  i.e. we can find $\phi_t : X \to X$ for
  $t \in [0, 1]$ such that
  $\phi_0(x) = x$ and $\phi_1(X) \subseteq A$ and
  $\phi_t(x) = x$ for all $x \in A$ and $t \in [0, 1]$.
\end{definition}

\begin{remark}
  If $X$ deformation retracts to $A$, then $X$ is
  homotopy equivalent to $A$. To see this, suppose we
  have a homotopy $\phi_t : X \to X$ as above, and let
  $i : A \to X$ be the inclusion map. Then
  $\phi_1 \circ i = \id_A$ and $i \circ \phi_1 = \phi_1 \sim \phi_0 = \id_X$, so
  $\phi_1$ is a homotopy equivalence from $X$ to $A$.
\end{remark}

\begin{definition}
  Given two spaces $X, Y$ and a map $f : X \to Y$,
  the \emph{mapping cylinder} of $f$ is the space
  \[
    M_f = ((X \times [0, 1]) \cup Y) / {\sim},
  \]
  where the equivalence relation $\sim$ is given by
  by $(x, 1) \sim f(x)$ for $x \in X$.
\end{definition}

\begin{remark}
  The mapping cylinder $M_f$ deformation retracts to $Y$.
  To see this, consider the map $\widetilde{\phi}_t$
  given by
  $(x, s) \mapsto (x, (1 - t)s + t)$ on
  $X \times [0, 1]$ and $y \mapsto y$ on $Y$. Since
  $\widetilde{\phi}_t$ respects the equivalence
  relation, it descends to a map $\phi_t : M_f \to M_f$
  on the quotient space. Note that
  $\phi_0 = \id_{M_f}$ and $\phi_1(M_f) = Y \subseteq M_f$,
  and $\phi_t = \id_Y$ for all $t$. Thus $\phi_1$ is a
  deformation retraction. In particular,
  this means that $M_f \simeq Y$.
\end{remark}

\begin{remark}
  There are obvious inclusions
  $i : X \to M_f$ given by $x \mapsto (x, 0)$
  and $j : Y \to M_f$ given by $y \mapsto y$.
  Note that $\phi_1$ defined above is the homotopy
  inverse to $j$. Now we have the diagram
  \[
    \begin{tikzcd}
      X \ar[r, "f"] \ar[dr, "i"] & Y \ar[d, "j"] \\
      & M_f
    \end{tikzcd}
  \]
  where $j$ is a homotopy equivalence and
  $j \circ f \sim i$ (exercise).
\end{remark}

\begin{remark}
  The above remark shows the following ``slogan''
  of algebraic topology:
  \begin{quote}
    Any map is an inclusion up to homotopy.
  \end{quote}
\end{remark}

\begin{example}
  Let $X$ be three circles with two enclosed in a third
  bigger one, and let $Y$ be two circles enclosing the
  inner two circles of $X$
  connected by a line segment. Let $Z$ be the region
  inside by the outer circle of $X$ and outside
  the inner two circles of $X$.

  Define $f : X \to Y$
  to be the map which sends $x \in X$ to the point in
  $Y$ at the other end of an interval (points on the
  inner circles of $X$ are mapped by radial lines to the
  circles in $Y$, and points on the outer circle of $X$
  are mapped radially to either the circles or
  the line segment in $Y$). One can write an explicit
  formula for $f$ as an exercise.

  Then $Z$ is homeomorphic to $M_f$, and in particular
  $Z \simeq Y$. Similarly, $M_f$ is homotopy equivalent
  to two circles joined at a point, or a circle
  with a diameter. Thus by transitivity, these two spaces
  and $Z$ are all homotopy equivalent to each other.
\end{example}

\end{document}
